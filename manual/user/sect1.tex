\chapter[Introduction]{INTRODUCTION}
\label{intro}

During the last several years, the finite element method has evolved
from a linear structural analysis procedure to a general technique
for solving non-linear partial differential equations.  An extensive
literature exists on the method describing
the theory necessary to formulate solutions
to general classes of problems. It is
assumed that the reader is familiar with the finite element method as
describe in popular reference books (e.g., {\it The Finite Element Method},
7th edition, by O.C. Zienkiewicz, R.L. Taylor and J.Z. Zhu\scite{zt1n7} and
{\it The Finite Element Method for Solid \& Structural Mechanics},
7th edition, by O.C. Zienkiewicz, R.L. Taylor and D.D. Fox\scite{zt2n7}) and
desires either to solve a specific problem or to generate new solution capabilities.

The Finite Element Analysis Program - Personal Version
({\sl FEAPpv}) is a computer analysis system designed for:
\begin{enumerate}
\item Use in an instructional program to illustrate performance
of different types of elements and modeling methods; 
\item In a research, and/or applications
environment which requires frequent modifications to address new
problem areas or analysis requirements.
\end{enumerate}
The computer system has been
developed primarily for Apple, UNIX and Windows environments
and includes an integrated set of modules to perform input of data describing
a finite element model, construction of solution algorithms to address a wide
range of applications, and graphical and numerical output of solution results.

A problem solution is constructed using a command
language concept in which the solution algorithm is
written by the user.  Accordingly, with this capability, each
user may define a solution strategy which meets specific needs.
There are sufficient commands included in the system for
applications in structural or fluid mechanics, heat transfer, and
many other areas requiring solution of problems modeled by differential
equations; including those
for both steady state and transient problems.

Users also may add new features for model description and command language
statements to meet specific applications requirements. These additions 
may be used to assist users in generating meshes
for specific classes of problems or to import meshes generated by other
systems.

The {\sl FEAPpv} system includes a limited element library.
Some elements are available to model one, two and three dimensional problems
in linear and non-linear
structural and solid mechanics and for linear heat conduction problems.
Each available element uses a material
model library. A few material models are provided for elasticity,
viscoelasticity, plasticity, and heat transfer constitutive equations,
however, it is assumed that users will want to add additional models for
their specific application.
Elements provide capability to generate mass and
geometric stiffness matrices for structural problems and to compute output
quantities associated with each element (e.g., stress, strain), including
capability of projecting these quantities to nodes to permit graphical
outputs of result contours.

Users also may add an element to the system by writing and linking a single
module to the {\sl FEAPpv} system. Details on
specific requirements to add an element as well as other optional
features available are included in the
{\sl FEAP Programmers Manual}.\scite{feapp}

The next several sections of this manual describe how to use existing
capabilities in the {\sl FEAPpv} system.  The discussion
centers on three different phases of problem solution using the system:
\begin{enumerate}
\item Mesh description options;
\item Problem solution options; and
\item Graphical display options.
\end{enumerate}
The {\sl FEAP Example Manual}\scite{feape}
may be consulted for examples of some of the input and solution
options available, however, users should consult this manual to ensure that
all features needed are part of the \textsl{FEAPpv} system.
Generally, problems defined for \textsl{FEAPpv} will also work with the full version
of the program (\textsl{FEAP}), but those for \textsl{FEAP} may not work with
\textsl{FEAPpv}.
