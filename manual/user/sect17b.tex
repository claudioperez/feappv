\section{Time Dependent Loading}

The loading applied to a problem may be changed during a solution
process.  This may be achieved by specifying new nodal loads for
each time step using the commands
\begin{verbatim}
       BATCh
        ...
       LOOP,time,steps
         MESH
          ...
       NEXT
        ...
       END
       FORCe
        ...
       END
       FORCE
        ...
       END
        ...
\end{verbatim}
in which a set of new forces appears for each time step performed.  The
use of the {\tt MESH} command within a solution strategy permits the alteration
of any nodal or element data.  It is not permitted to change the size of the
problem by adding new nodes or elements (elements may be {\tt ACTI}vated
or {\tt DEAC}tivated based on region descriptions); however, nodal forces,
displacements, boundary restraint codes, etc. may be changed.
Material parameters may be changed but not the type of material model (i.e.,
it is not permitted to change a model from elastic to elasto-plastic during
the solution process).

The above form, while general in concept, requires extensive amounts of
data to describe the behavior.  {\sl FEAP} can easily treat
loading states which may be written in the form
\begin{equation}
\B{F}(t) = p_j(t) \, \B{F}_j
\end{equation}
where $p_j(t)$ is a set of time dependent (proportional loading) factors
and $\B{F}_j$ is a fixed loading pattern on a mesh.

To perform an analysis involving proportional loads, during mesh input
it is necessary to specify:
\begin{enumerate}
\item
Nodal force patterns $\B{F}_j$;
\item
Associations between force patterns and proportional loading factors
using the $FPRO$\-por\-tion\-al command during mesh generation.  This command
has the form
\begin{verbatim}
       FPROportional
         NODE NG J1 J2 ... Jndf
            ...
                         ! Termination record
\end{verbatim}
where the {\tt Ji} define the proportional load number $p_j$ assigned to
a degree of freedom.  A zero value will use the {\it sum} of all specified
proportional load factors as the multiplier for an associated force or
displacement, whereas, a non-zero value will use only the individual $p_j$
factor; and
\item
The proportional loading function $p_j(t)$.
\end{enumerate}

A proportional loading function is specified using the solution command 
\begin{verbatim}
       PROPortional,,Ji,Jj
\end{verbatim}
where {\tt Ji} and {\tt Jj} define a range of loadings to be input.
If {\tt Jj} is zero only the {\tt Ji} function is to be input.
A functional type of proportional loading is
\begin{equation}
p(t) = a_0 + a_1 t + a_2 [ \sin a_3 (t - t_{min})]^k ~~~;~~~
t_{min} \le t \le t_{max}
\end{equation}
and is input by the statements
\begin{verbatim}
       PROP,,J1
        ...
       END
       1 K T-min T-max A0 A1 A2 A3
\end{verbatim}
This is called TYpe 1 loading and requires a 1 in the first column defining
the parameters.  A blank record is considered to be a Type 1 loading with
default parameters:
\begin{equation}
t_{min} = 0 ~;~ t_{max} = 10^{6} ~;~ a_1 = 1 ~;~ k = a_0 = a_2 = a_3 = 0
\end{equation}

A piecewise linear set of values may be input using the Type 2 proportional
loading function which is specified by a {\tt PROP}ortional command  whose
data is:
\begin{center}
\begin{tabular}{l l l l l l}
  2 & n \\
$t_1$ & $p_1$ & $t_2$ & $\cdots$ & $t_n$ & $p_n$ \\
$t_{n+1}$ & $p_{n+1}$ & $\cdots$ & $\cdots$ & $t_{2n}$ & $p_{2n}$ \\
$\cdots$ & $\cdots$ \\
\end{tabular}
\end{center}
by default $n = 1$ and the values appear as time/factor pairs on each record.
Input terminates with a blank record.

\section{Continuation Methods: Arclength Solution}

Many non-linear static problems have solutions which exhibit limit load
states or other types of variations in the response which make solution
difficult.  Continuation methods may be employed to make solutions to
this class of problems easier to obtain.  {\sl FEAP} includes a version
of continuation methods based on maintaining a constant length of a specified
load-displacement path.  This solution process is commonly called an {\it
arclength} method.  To employ the arclength method in a solution the
command {\tt ARCL}ength is used.  A typical algorithm for an arclength 
solution is given by:
\begin{verbatim}
       ARCLength
       DT,,delta-t
       LOOP,time,n-steps
         TIME
         LOOP,newton,n-iters
           TANGent,,1
         NEXT,iteration
          ..... (outputs, etc.)
       NEXT,time
\end{verbatim}

\begin{description}
\item[Remark:]{It is {\it not} permitted to use a {\tt PROP}ortional loading
command with the arclength procedure.}
\end{description}

\section[Augmented Solutions]{AUGMENTED SOLUTIONS}
\label{augment}

{\sl FEAP} has options to employ penalty method solutions to enforce
CONStraints.  A penalty method is used as an option of the {\tt GAP}
element and also to enforce incompressibility constraints in some of
the continuum elements.  The use of large penalty parameter
values in some material
models and some finite deformation analyses makes a
Newton iteration loop difficult or impossible to converge.  Often
when the penalty parameter value is reduced so that acceptable convergence
of the iteration is achieved it is observed that the constraint is not
accurately captured.  In these cases it is possible to achieve a better
satisfaction of the constraint by using an {\it augmented Lagrangian}
solution strategy.  The augmented Lagrangian solution scheme implemented
in {\sl FEAP} is based on the Uzawa algorithm briefly discussed on pp 404ff
of {\it The Finite Element Method: Its Basis \& Fundamentals}, 6th ed., by
Zienkiewicz, Taylor and Zhu\scite{zt1n6}.  The command language program to perform an augmented solution
is given by:
\begin{verbatim}
       LOOP,augment,n-augm
         LOOP,newton,n-iter
           TANGent,,1
         NEXT,iter
         AUGMent
       NEXT,augment
\end{verbatim}
The number of augmented iterations ({\tt n-augm}) should be kept quite
small as convergence of the iteration process is only checked by the
{\tt TANG}ent command.  If convergence is achieved in this loop execution
passes to the {\tt AUGM}ent command and another augmentation is performed
until the {\tt n-augm} augmentation iterations are performed.

\section{Time History Plots}
\label{tplot}

The response of specific solution quantities may be saved in files during
solution using the {\tt TPLO}t command.  This permits the construction of
time history plots during or after the completion of a solution using any
program which is capable of constructing x-y plots from files (e.g., using
gnuplot or Matlab).  The {\tt TPLO}t command works only with time dependent
problems and whenever the command {\tt TIME} is executed writes
data to files with the
name designated for plots at the start of execution and an added extender. 
To recover the last computed data set it is necessary to conclude an analysis
with a {\tt TIME} command.  The {\tt TPLO}t command is given as
\begin{verbatim}
       TPLOt
        ...
       END
       type,n1,n2,x,y,z
       show (optional to force echo of data list)
       ...
                    ! Termination record
\end{verbatim}
The parameters may have the values described in Table \ref{tabtpl}.

\begin{table}
\begin{center}
\begin{tabular}{| c | c | c | c | c | c |} \hline
Type & $n_1$ & $n_2$ & $x$ & $y$ & $z$ \\ \hline
disp &  Node & dof   & $x$ & $y$ & $z$ \\
velo &  Node & dof   & $x$ & $y$ & $z$ \\
acce &  Node & dof   & $x$ & $y$ & $z$ \\
reac &  Node & dof   & $x$ & $y$ & $z$ \\
stre &  Elmt & component   & - & - & - \\
arcl &  Node & dof   & $x$ & $y$ & $z$ \\
cont &  Node & dof   & $x$ & $y$ & $z$ \\
ener &  Comp & print       & - & - & - \\ \hline 
\end{tabular}
\caption{Tplot types and parameters}
\label{tabtpl}
\end{center}
\end{table}

Indicated data may be given either by the node number, or the coordinate
of the point where the data is located (the closest node to the point
will be selected).  The energy components, if computed, should be ordered
as: 1-3: linear momentum; 4-6: angular momentum; 7: stored energy; 8: kinetic
energy; 9: dissipated energy; and 10: total energy.

\section{Viewing Solution Data: SHOW Command}
\label{show}

The {\tt SHOW} command permits users to display the problem size and values
for some of the solution parameters as well as to check the amount of data
stored in arrays allocated in the solution space.  The command is given as
\begin{verbatim}
       SHOW,option,v1,v2
\end{verbatim}
where {\tt option} can have the values {\tt DICT}ionary, {\it array name}, or 
be omitted.  When omitted the {\tt SHOW} command displays values for basic
solution parameters.  Use of the {\tt DICT}\-ion\-ary option displays the
names, type, and size for all arrays currently allocated in the solution
space.  Values stored in each array may be displayed by using the name
as the {\it array name} option.  If the array is large the {\tt vi} parameters
can be used to limit the amount of information displayed.

\section{Re-executing Commands: HISTORY Command}
\label{hist}

A useful feature of the command language for 
interactive executions is the {\tt HIST}ory command.  During the execution
of solution commands the program compiles a list of all commands 
executed (called the {\it history list}) which may be used to re-execute
one or several of the commands.  The user may also {\tt SAVE} this list
as a file and at a later time {\tt READ} the list back into the program.
At any stage of interactive execution the list may be displayed by
entering the command {\tt HIST,LIST} or {\tt HIST};
alternatively, a portion of the list
may be displayed; e.g., {\tt HIST,LIST,5,9} will display only commands
5 through 9.  A user may then re-execute commands by entering
the command numbers from the history list.  For example, {\tt HIST,,1}
(note the double commas as field separators) would re-execute
command 1, or {\tt HIST,,6,9} would re-execute commands 6 through 9
inclusive.  The history commands also may be embedded in a normal
command language {\tt LOOP-NEXT} pair;  e.g., entering the commands:
\begin{verbatim}
       LOOP,,4
         HIST,,6,9
       NEXT
\end{verbatim}
performs a loop 4 times in which during each loop commands
6 through 9 are executed.
If the history commands 6 to 9 involve a loop or next which is not
closed it is necessary to provide a closing sequence before execution
will commence.

\section{Solutions Using Procedures}
\label{proc}

Many analyses require the use of a sequence of commands which 
are then reused throughout the solution process or in subsequent
solution of problems.  For example, in
the analysis of static problems the sequence of commands:
\begin{verbatim}
       TANG,,1
       DISP,ALL
       STRE,ALL
       REAC,ALL
       STRE,NODE,1,50   !(output first 50 nodes)
\end{verbatim}
may be used.  The repeated input of this sequence is not only time
consuming but may result in user input errors.  This sequence of
commands may be defined as a {\tt PROC}edure and saved for use during
the current analysis or during any subsequent analysis.  A procedure
only may be defined during an interactive solution; however, it may
be used in either a batch or interactive solution (including
the solution in which the procedure is defined).  A procedure is 
saved in the current directory in a file with the extender {\tt .pcd}.

A procedure is created during an interactive analysis by entering the command:
\begin{verbatim}
       PROCedure,name,v1,v2,v3
\end{verbatim}
The name {\it procedure} may be abbreviated by the first four (or more)
characters, {\it name} is any 1-8 character alphanumeric identifier
which specifies the procedure name ({\it the first 4 characters must
not be the same as an existing command name}), {\tt v1,v2,v3} are
any 1 to 4 alphanumeric parameter names for the procedure.  The
parameters are optional.  For example the procedure for
a static analysis may be given as:
\begin{verbatim}
       PROCedure,STATIC,NODE
\end{verbatim}
After entering a procedure name and its parameters,
prompts to furnish the commands
for the procedure are given.  These are normal execution commands
and may not contain calls to other procedures or {\tt HIST} commands.
The parameter names
defined in the procedure (e.g., {\tt NODE} in the above {\tt STATIC}
command) may be used in place of any numerical
entries in commands.  A procedure is terminated
using an {\tt END} command.  As an example the complete
static analysis procedure would read:
\begin{verbatim}
       PROCedure,STATIC,NODE
         TANG,,1
         DISP,ALL
         STRE,ALL
         REAC,ALL
         STRE,NODE,1,NODE
       END
\end{verbatim}
Note that in the {\it nodal stress} command the parameter {\tt NODE}
is used twice.  The first use is for the definition of the
command and is an alphanumeric parameter of the
command.  The second {\tt NODE} is a numerical parameter of
the command.  The value for this {\tt NODE} parameter is taken from the one
specified at the time of execution.  The use of
the static procedure is specified by entering the command line:
\begin{verbatim}
       STATIC,,50
\end{verbatim}
and, at execution, the 50 will be the value of the {\tt NODE} parameter in the
procedure definition above (e.g., the first 50 nodal stresses will be
output).  All characters in the {\it name} (e.g., up to 8 characters) of a 
procedure must be specified.
It is not permitted to abbreviate the name of a procedure using
the first four characters of the procedure {\it name}.

The procedure {\tt STATIC} may be used in any subsequent analysis
by entering the valid procedure name and parameters (if
needed).  Currently it is not possible to preview a procedure
while a solution is in progress (they can be viewed from other windows in
a multi-window compute environment).
Thus in large analyses it is advised that a
review of the {\tt NAME.PCD} file be made to look at the contents.
Extreme care should be exercised to prevent long unwanted
calculations or outputs from an inappropriate use of a procedure.  For
example, a {\tt STREss,ALL} is a viable command for small problems but can
produce very large amounts of data for large problems.

\section{Output of Element Arrays}
\label{eouts}

When solving problems difficulties often occur for which additional information
is needed about an element.  {\sl FEAPpv} includes a capability to print the
arrays produced by the highest numbered element (i.e., the one whose number
is {\tt NUMEL}) by the last command.  The command is named {\tt EPRI}nt.
For example, after a {\tt TANG}ent command the use of {\tt EPRI}nt
would display the element tangent matrix (e.g., stiffness) and residual
vector for this element.  This option works for both symmetric and unsymmetric
tangents.  Similarly, the element mass matrix used for eigen computations
could be output using the command immediately after the {\tt MASS} command.
