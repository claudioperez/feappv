\section{Coordinate and Transformation Systems}
\label{transform}

The coordinates in {\sl FEAPpv} must all be given in a Cartesian system.
Input data, however, may be specified in {\it Cartesian}, {\it polar}
(which in three dimensions is interpreted as cylindrical coordinates),
or {\it spherical} coordinate systems.  If polar or spherical coordinates
are used to define the nodal data using the {\tt COOR}\-dinate command,
they may be transformed to the required
Cartesian form using the {\tt POLA}r or {\tt SPHE}rical commands, respectively.
Nodal coordinates generated with polar or spherical options in the {\tt BLOC}k
command do not require transformation. 
The data for a polar command is:
\begin{verbatim}
       POLAr
         NODE,n1,n2,inc
\end{verbatim}
\par\noindent
where {\tt n1} and {\tt n2} define a range of nodes and {\tt inc} is
the increment to be added to {\tt n1} for each step to {\tt n2}.  Alternatively,
all currently defined nodes may be transformed using the command
\begin{verbatim}
       POLAr
         ALL
\end{verbatim}
\par\noindent
The transformation is given by
$$ x_1 ~=~ x_0 \, + \, r \, \cos \, \theta$$
$$ x_2 ~=~ y_0 \, + \, r \, \sin \, \theta$$
and
$$ x_3 ~=~ z_0 \, + \, z$$
where $x_i$ are the Cartesian coordinates, $r$, $\theta$, $z$ are the
polar (cylindrical) inputs, and $x_0$, $y_0$, $z_0$ are shifts defined
by the {\tt SHIF}t command given as
\begin{verbatim}
       SHIFt
         X_0,Y_0,Z_0
\end{verbatim}
By default $x_0$, $y_0$, $z_0$ are zero.

The {\tt SPHE}rical command is similar to the {\tt POLA}r command.  The
input records are specified as:
\begin{verbatim}
       COORdinate
         N NG R THETA PHI
\end{verbatim}
Transformations use the relations
$$ x_1 ~=~ x_0 \, + \, r \, \cos \, \theta \, \sin \, \phi$$
$$ x_2 ~=~ y_0 \, + \, r \, \sin \, \theta \, \sin \, \phi$$
and
$$ x_3 ~=~ z_0 \, + \, r \cos \, \phi$$

\subsection{Coordinate Transformation}

Cartesian systems may be translated, stretched, reflected and/or rotated using
the {\tt TRAN}\-sform command.
Any coordinates input after this command are transformed using
$$\left[
\begin{matrix} x_1 \\ x_2 \\ x_3 \end{matrix} \right] ~=~
\left[
\begin{matrix}
T_{11} & T_{12} & T_{13} \\
T_{21} & T_{22} & T_{23} \\
T_{31} & T_{32} & T_{33} \\
\end{matrix} \right] \,
\left[
\begin{matrix} \hat x_1 \\ \hat x_2 \\ \hat x_3
\end{matrix} \right] ~+~
\left[
\begin{matrix} x_0 \\ y_0 \\ z_0
\end{matrix} \right]$$
where $\hat x_i$ are the input values and the transformation parameters are
defined by the command sequence
\begin{verbatim}
       TRANsform
         T_11  T_12  T_13
         T_21  T_22  T_23
         T_31  T_32  T_33
         X_0   Y_0   Z_0
\end{verbatim}
which must appear before any coordinates (i.e., the $\hat x_i$) are specified.

The {\tt TRAN}sform command may be used as many times as needed.  In particular,
it may be used with a portion of a mesh (substructure) in an include file
to replicate repeated parts of meshes.  When a reflection is performed,
{\sl FEAPpv} notes the coordinate transformation does not have a positive
determinant and resequences the node numbers on elements to maintain
a positive jacobian (provided the original data is correct in its local
Cartesian basis - $\hat{x}_i$).
