\chapter[Input Records]{INPUT RECORDS SPECIFICATION}
\label{record}

Data input specifications in {\sl FEAP} consist of records which may contain
from 1 to 255 characters of information in free format form.
Each record can contain up to 16 alphanumeric data items.  The maximum
field width for any single data item is 15 characters (14 characters of
data and 1 character for separating fields).  Specific types of data items
are discussed below.  Sets of records, called {\it data sets},
start with a text command which controls input of one or more 
data items.  Data sets may be grouped into a single file (called the input
data file) or may be separated into several files and joined together
using the {\it include} command described below.  Sets of records may also
be designated as a {\it save} set and later {\it read} again for reuse.

Each input record may be in the form of text and/or numerical
constants, parameters, or expressions.  Text fields all start with the 
letters {\tt a} through {\tt z} (either upper or lower case may be
used, however, internally {\sl FEAP} converts upper case letters to lower case).
The remaining characters may be either letters or numbers.
Constants are conventional
forms for specifying input data and may be integer or real
quantities as needed.  Parameters consist of one or two characters to
which values are assigned.  The first character of a parameter must be
a letter (a to z); the second may be a letter (a to z) or numeral (0 to 9).
Expressions are combinations of
constants, parameters, and/or functions which can be evaluated as the
required data input item.  Each of these forms is described below.

\section{Constants}

Constants may be represented as integers or floating point numbers.
Integers are specified without a decimal point as 1, -10, etc;
floating point numbers may only be expressed in the forms
\begin{verbatim}
       3.56,   -12.37,   1.34e+5,   -4.36d-05
\end{verbatim}
In particular, the forms
\begin{verbatim}
       1.0+3,  -3.456-03
\end{verbatim}
may not be used since they will be evaluated as an expression (see
below).  In particular, the above two examples would yield data values
of {\tt 4.0} and {\tt -6.456}, respectively.

The specification of each constant is restricted to 14
significant figures (including the exponent value) plus a separator (either
a comma or a blank).  If more
significant figures are needed in an exponent form, parameters and
an expression may be used.  For example,
\begin{verbatim}
       a1 =  1.234567890123*1.e-5
\end{verbatim}
produces a number with the full 14 digits but with an exponent larger than
could otherwise be obtained with this precision and stay within the
14 character limit.

\section{Parameters}

The use of parameters will simplify the data input
required to define problems for a {\sl FEAP} solution.
Data may be specified as a single character parameter
(e.g., $a$, $b$, through $z$), two character parameters
(e.g., $aa$, $ab$ through $zz$), or a character and a numeral
(e.g., $e0$ through $e9$).  All alphabetic input
characters are automatically converted to lower case, hence there are 962
unique parameters permitted at any one time.  Values are
assigned to parameters by the {\tt PARA}meter data command during mesh
generation or modification.  The general form to assign a constant
to a parameter is
\begin{verbatim}
       a  = 3.567
       e1 = 200.0e9
       nu = 0.3
\end{verbatim}
Blanks are permitted and are ignored in the processing of a
record (except in expressions).
Once a parameter is defined it may be used in place of
any constant in the data input.  For example the following would use
the value of the parameter $a$ defined above
\begin{verbatim}
       COORdinates
         1,,a,0.
\end{verbatim}
With this assignment the 1-coordinate of the 1-node would have a
value of 3.567.

Parameters may have their values redefined as many times as
needed by using the PARAmeter data command followed by other commands
and data using the values of assigned parameters.  A user may then
specify another
PARAmeter command to redefine parameters, followed by additional
data inputs, etc.

\section{Expressions}

The most powerful form of data input in {\sl FEAP} is through the use of
expressions in combination with parameters.  An expression may
include parameters and/or constants.  Expressions may include
operations of addition, subtraction, multiplication, division, and
exponentiation.  In addition, some functions may be used.  A
hierarchical evaluation is performed according to the rules
defined in Table \ref{tab51}.

\begin{table}[ht!]
\begin{center}
\begin{tabular}{ c l l }
Order & Operation & Notation \\
   &                            &            \\
1. & Parenthetical expressions  &  {\tt(  )}   \\
2. & Functions                  &            \\
3. & Exponentiation             & \char '136    \\
4. & Multiplication or Division & {\tt *}  or {\tt /}  \\
5. & Addition or Subtraction    & {\tt +}  or {\tt -}  \\
\end{tabular}
\end{center}
\caption {Hierarchy for expression evaluation}
\label{tab51}
\end{table}
Evaluations within this hierarchy proceed from left to right in
each expression.  At the present time only one level of parenthesis
may appear in any expression.  Accordingly, the expression
\begin{verbatim}
       1./4. + 4
\end{verbatim}
is evaluated as 4.25, whereas
\begin{verbatim}
       1./(4. + 4)
\end{verbatim}
is evaluated as 0.125.

All constants, parameters, and expressions are evaluated as double
precision real quantities, however, they are permitted in place of integer
data also.  Expressions may appear in any location
in place of a constant or an expression.  Accordingly, a force may
be assigned as
\begin{verbatim}
       FORCe
         1,,a/12. + 3.
\end{verbatim}
Additionally, node and element numbers may also
appear as expressions.  This permits the input of {\it substructure}
parts in a modular form.  For example,
\begin{verbatim}
       BLOCk
         CARTesian,4,4,n,e,m
         1,0.+x,0.+y
         2,5.+x,0.+y
         3,5.+x,5.+y
         4,0.+x,5.+y
                        ! end with blank record
       COOR
         n+25,0,5.5+x,2.5+y
                        ! end with blank record
       ELEM
         e+16,p,n+14,n+15
                        ! end with blank record
\end{verbatim}
could be used to input a block of nodes and elements (see Section \ref{block}).
By specifying
the values of the parameters {\tt n,e,m,x,} and {\tt y} a form of a
{\it substructure} is defined.  The part may be replicated using either
the {\tt INCLude} option or the
name associated with {\tt SAVE} and {\tt READ} in the mesh
data input statements.

\section{Functions}

The following functions may appear in an expression, a statement, or a
parameter definition:
\begin{verbatim}
       abs   exp,  int,  log,   sqrt,
       sin,  cos,  tan,  atan,  asin,  acos,
       sind, cosd, tand, atand, asind, acosd,
       cosh, sinh, tanh,
\end{verbatim}
The trigonometric and inverse trigonometric functions which end in {\it d}
involve values of angles in degrees; whereas, the ones without involve values
in radians.

Each function has one argument which is contained between parenthesis
(which counts as the one level of depth).  The argument may be an
expression but may not contain any parenthesis or additional functions.
Thus, the expression
\begin{verbatim}
       p = 4.*atan(1)
\end{verbatim}
will compute the value of $\pi$ and assign it to the parameter $p$.
Internal computations are all preformed in double precision arithmetic (e.g.,
as {\tt REAL*8} variables).
Again note that the function parenthesis count as one level, hence
\begin{verbatim}
       q = tan(1./(3.+a))
\end{verbatim}
is not a legal expression at the present time.
It should be replaced by the pair of statements
\begin{verbatim}
       q = 1./(3.+a)
       q = tan(q)
\end{verbatim}

\section{Include Commands in Mesh Input}

Any data input records may be placed in a separate file and read using
the {\tt INCLude} command.  The form for the include is a single record
\begin{verbatim}
       INCLude,filename
\end{verbatim}
where {\tt filename} is the name of the file containing the input data.
This command may be used at any time and include files may call other
include files (to a maximum level of 9).  Thus, if the nodal coordinates
are created by another program and written to a file named {\tt Blockxy}
{\footnote{Upper and lower case letters are treated as different on 
workstations but the same on PCs}},
they may be input as {\sl FEAP} data using:
\begin{verbatim}
       COORdinates
       INCLude,Blockxy
                          ! blank termination record
\end{verbatim}
The information in each file must always be in the format required by
{\sl FEAP}.
If another format is written, then it is necessary to either translate the
data to the correct form or to write and link a user routine which can
input the data.
The creation of user routines is discussed in the {\sl FEAP Programmers Manual}.

\section{Read and Save Commands in Mesh Input}

A group of mesh input statements also may be retained for future use by
placing them between the statements
\begin{verbatim}
       SAVE,filename
       .....
       .....
       SAVE,END
\end{verbatim}
{\tt filename} may be any 1 to 14 alphanumerical characters. Thus if
a {\tt SAVE MSH1} is used a new file named {\tt MSH1} will be
created to store the mesh commands to be saved.

For example, the following option may be used to generate nodal
forces with a variation in a load parameter.
\begin{verbatim}
       PARAmeter
         a= 5.
                       ! end with blank record
       SAVE,msh1       ! may also be SAVE,mes1
       PARAmete
         b= a/2
                       ! end with blank record
       FORCe
         31,0,b
         32,1,a
         34,0,a
         35,0,b
                       ! end with blank record
       SAVE,END
\end{verbatim}
A different loading state may then be specified by:
\begin{verbatim}
       PARAmeter
         a= -4.
                       ! terminator
       READ,msh1
\end{verbatim}
The value of $b$ will be recomputed using the new value of $a$
and the nodal forces will then be recomputed.  Many options are
possible using the features of parameters, expressions, INCLude, and SAVE
and READ commands.
