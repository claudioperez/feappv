\chapter[Element Library]{ELEMENT LIBRARY}
\label{elmlib}

{\sl FEAP} contains a library of standard elements and material models which may
be employed to solve a wide range of problems in solid and structural mechanics,
and heat transfer analysis.
In addition, users may program and add new elements to the program.
The type of element to
be employed in an analysis is specified as part of the \texttt{MATErial} data
sets.  The first record of each material data set also contains the material
property number.  Each material property number is an integer ranging from
one (1) to the maximum number of material sets specified on the control data
record (which immediately follows the \texttt{FEAP} start/title record); however,
as noted earlier,
the maximum number on material sets on the control data may be specified
as zero and {\sl FEAP} will
automatically compute the maximum number of material sets from the input data.
The second record of the material set data defines the type of element to be
used.  The library of standard elements includes the following types:

\begin{enumerate}
\item
SOLId - The solid element is used to solve continuum problems with
either small or large deformations.
Options exist to use finite elements based on displacement,
mixed, and enhanced strain formulations.
The element contains a library of
models for elastic, viscoelastic, or elastoplastic constitutive equations.
For two dimensional problems each element is a quadrilateral
with between 4 and 9-nodes (enhanced formulation permits only 4-node
quadrilaterals).  The two dimensional displacement formulation also permits
use of 3 or 6-node triangular elements.
The degrees of freedom on each node are displacements,
$u_i$, in the coordinate directions.
The degrees of freedom are ordered as:
2-D Plane problems, $u_x , u_y $, coordinates are $x , y$;
2-D Axisymmetric problems, $u_r , u_z $, coordinates are $r , z$;
For three dimensional problems each element is an 8-node hexahedron (brick).
with degree-of-freedoms $u_x , u_y , u_z$.
The displacement element may also be a 4-node tetrahedron.

\item
FRAMe - The frame element is used to model structural members which include
axial, bending, and shearing deformations only.  The model is formulated in
terms of force resultants which are computed by integration of stress components
over the cross-sectional area of the member.
Each element has 2-nodes and may be used in a two or three dimensional
problem.  The degrees of freedom on each node are: Displacements, $u_i$, in the
coordinate directions and; A rotation, $\theta_z$, about the z-axis
for two dimensions
and rotations, $\theta_i$, about all axes for three dimensions.
The degrees of freedom are ordered as:
2-D $u_x , u_y , \theta_z$;
3-D $u_x , u_y , u_z , \theta_x , \theta_y , \theta_z$;

\item
TRUSs - The truss element is used to model structural members which include
axial deformations and forces only.  The axial
force resultant is computed by integration of the axial stress component
over the cross-sectional area of the member.
Each element has 2-nodes and may be used in any one to three dimensional
problem.  The degrees of freedom on each node are displacements, $u_i$, in each
coordinate direction; thus, the number is the same as the spatial dimension
of the problem.
The degrees of freedom are ordered as: $u_x , u_y , u_z$

\item
PLATe - The plate element is used to model structural behavior of planar
bodies which have one dimension small compared to the two other dimensions.
The element may be used for small deformation analyses only and
includes bending and transverse shearing deformations.
Provisions are also included to permit modeling of plates for which the
transverse shearing deformations are ignored.  The model is formulated in
terms of force resultants which are computed by integration of stress components
over the thickness of the plate.
Each element may be a triangle with 3-nodes or a quadrilateral
with 4-nodes and is used in a two dimensional
problem.  The degrees of freedom on each node are: The transverse displacement,
$u_3 = w$, and rotations $\theta_x$ and $\theta_y$ about the coordinate axes.
The degrees of freedom are ordered as: $w , \theta_x , \theta_y$;

\item
SHELl - The shell element is used to model structural behavior of curved
bodies which have one dimension small (a thickness normal to
the remaining surface coordinates) compared to the other dimensions of
the surface.  The shell for small deformations includes bending and
in-plane deformations only (no transverse shearing strains).
The model is formulated in
terms of force resultants which are computed by integration of stress components
over the cross-sectional thickness of the shell.
Each element is a quadrilateral with 4-nodes
and may be used in a three dimensional
problem.  The degrees of freedom on each node are: Displacements,
$u_i$, and rotations, $\theta_i$, about the coordinate axes.
The degrees of freedom are ordered as:
$u_x , u_y , u_z , \theta_x , \theta_y , \theta_z$ (6-dof);
The large displacement shell element includes in-plane, bending, and shearing
deformations in a 5 degree-of-freedom form.  This element is formulated
in an energy-momentum conserving form and may not converge quadratically in
general applications.

\item
MEMBrane- The membrane element is used to model structural behavior
of curved bodies which are thin and carry loading by in-plane loading
only.  These elements are generally unstable unless attached to a
contiguous solid or otherwise restrained.  The model is formulated in
terms of the in-plane force resultants and a cross-sectional thickness.
Each element is a quadrilateral with 4-nodes
and may be used in a three dimensional
problem.  The degrees of freedom on each node are: Displacements,
$u_i$.  The degrees of freedom are ordered as:
$u_x , u_y , u_z $;

\item
THERmal -  The thermal element is used to compute temperatures in solid
bodies or truss elements.  The element solves the Fourier heat conduction
equation.  For the truss element the problem is one-dimensional.
For two dimensional problems each element is a quadrilateral
with between 4 and 9-nodes.  For three dimensional problems each element
is a brick with 8-nodes or a tetrahedron with 4-nodes.
The degree of freedom on each node is temperature, $T$, and, by default,
is placed in the first position in the unknowns (i.e., first degree of freedom).
If the element is combined with a solid element to perform thermo-mechanical
analyses it is necessary to relocate the temperature degree of freedom using
the option on the material set element type record (see the \texttt{MATE}erial
set command description in the {\sl FEAP} Mesh User Manual in Appendix A).
Alternatively, the location may be set using a \texttt{GLOB}al command option.

\item
POINt element - The point element may consist of a mass, linear dashpot, and/or
linear spring. The dashpot and spring are fixed at one end and added to the
degrees-of-freedom specified by the node of a 1-node element.
The dashpot and spring are oriented using a specified direction vector.
The element may be used in any two or three dimensional problem.
The degrees of freedom are ordered as: $u_x , u_y , u_z$ (ndm-dof);

\item
PRESsure loading - The pressure loading element is used to apply normal loading
to the surface of three dimensional objects.  The loading is associated with
elements defining the surface on which to apply the load.  Loads may be applied
with respect to the normals in the reference configuration (dead loads) or
with repect to the current configuration (follower loads).  For follower
loads a tangent matrix is computed (unsymmetric) and included during assembly.

\item
GAP - The gap element is used to model a restraint between nodes
which permits only compressive loads to be transmitted.  The restraint
must be in one of the coordinate directions.  This element may
be used in one, two, or three dimensional problems.

\item
USER -  Each user element must be developed and added to the program.
Provisions are included which permit the addition of up to 50 additional
element modules to the program.
The shape of the element, the number of degrees of freedom at each node,
and other parameters may be set by the user.  See the {\sl FEAP}
Programmers Manual for information on adding a user element.
\end{enumerate}

Each element requires additional input data to describe the
specific constitutive model, the finite element formulation to be used,
loading applied to elements, etc.
As an example consider an analysis of a two dimensional
continuum with a single material and constrained to a plane strain
deformation state.  The problem is to be modeled by an elastic
material with isotropic properties.
\begin{verbatim}
       MATErial,1
         SOLId
         ELAStic ISOTropic 30e+06 0.3
                                  ! blank termination record
\end{verbatim}

The property \texttt{ELAStic} is required for all types of \texttt{SOLId} elements.
The {\it solid elements} contain both small and finite deformation options for
two and three dimensional problems.  For small deformations there are three
element types: (1) A displacement model; (2) A mixed $\B{u}-p-\theta$ model;
and (3) An enhanced strain model.  In finite deformations only the displacement
and the mixed model exist at this time.  The material records for an
analysis which includes solid elements is shown above.

In two dimensional applications
the displacement and the mixed formulation may be described by elements
with between four (4) and nine (9) nodes.  The enhanced strain element
is limited to four (4) nodes only.  A three (node) triangular element
may be formed for the displacement model by repeating the number of any
node or by specifying only three nodes on an element.
In three dimensional applications the element is described by an eight (8)
node hexahedron.  The displacement model may also describe wedge
by coallescing appropriate nodes of a hexahedron and a 4-node tetrahedron.

The material models and other options
available for use with the solid elements are described in the next chapter.

The {\it frame elements} can treat small and large displacement problems.
The small displaecment element is restricted to elastic behavior
and includes effects of bending and axial deformations.  Cubic interpolations
are used.
The finite deformation frame elements are based on the exact kinematic
formulation of Ibrahimbegovic ~~\cite{adnan98a}.
The element includes elastic resultant and one dimensional
models and one dimensional inelastic behavior based on integration
over the cross
section.  Cross sectional shapes are included for thin tubes, solid circular
shape, rectangles, angles, channels, and wide flange shapes.
Interpolations are linear along the beam axis.
All elements have
two nodes.  To definie the orientation of the cross section for a three
dimensional analysis it is necessary to define a \texttt{REFE}r\-ence
\texttt{VECT}or, \texttt{DIRE}c\-tion, or \texttt{NODE}.
The frame element is included using the commands:
\begin{verbatim}
       MATErial,1
         FRAMe
		 ....
\end{verbatim}
The required data for frame elements is the material model, cross section
data, and for three dimensional frames geometric information to orient the
coordinate axes of the cross section.  Typically, \texttt{ELAStic} models are
required and can be supplemented by plastic or viscoelastic properties.  The
cross section data is given either as \texttt{CROSs} section properties or
\texttt{SECTion} types.  The geometric data for orienting cross section axes
is given by \texttt{REFErence VECTor} or \texttt{REFErence NODE} options.

The {\it truss elements} include small and large deformation formulations.
The elements have two nodes and include a number of one dimensional
constitutive models as indicated in the next chapter.
The truss element is included using the commands:
\begin{verbatim}
       MATErial,1
         TRUSs
		 ....
\end{verbatim}
Required data is material model (e.g., typically \texttt{ELAStic}) and cross
section \texttt{CROSs} giving the area of the section.

The {\it plate element} is restricted to small deformation applications
in which only the bending response of flat slabs is included.  The problem
is treated as a two-dimensional problem for the mesh (in the $x_1$-$x_2$
coordinate plane).  At present only
linear thermo-mechanical response is included for the material models.
Each element may be a three node triangle or a four node quadrilateral.
The plate element is included using the commands:
\begin{verbatim}
       MATErial,1
         PLATe
		 ....
\end{verbatim}
Required data is the material model (e.g., \texttt{ELAStic}) and the thickness
given by the \texttt{THICk} option.

The {\it shell elements} are capable of both small and finite deformation
analysis. 
Both resultant and through thickness integration forms are available for the
small displacement formulation.  For the through thickness formulation
all the constitutive forms available for the two-dimensional small
deformation analyses are also available for the shell.
The resultant formulation is restricted to elastic behavior.
The large displacement element is also currently
restricted to an elastic resultant formulation.
The small deformation model includes bending and membrane strains only -
no transverse shearing deformation is included - thus restricting application
to thin shell problems only.  The finite deformation shell is based on the
energy-momentum conserving development of Simo et. al. ~\cite{simosh92}
and includes exact
large displacement deformations for membrane, bending and shearing strains.
Since the formulation is based on the energy-momentum algorithm it is
necessary to specify a \texttt{TRAN}sient analysis with either the \texttt{STAT}ic
or \texttt{CONS}erving options (see chapter on transient solutions).  Also,
the tangent matrix is non-symmetric, thus to achieve quadratic rates of
convergence the \texttt{UTAN}gent solution command must also be employed.
The shell element is included using the commands:
\begin{verbatim}
       MATErial,1
         SHELl
		 ....
\end{verbatim}
Required data is the material model (e.g., \texttt{ELAStic}) and the thickness
given by the \texttt{THICk} option.

The {\it membrane elements} are derived from the shell elements by deleting
the bending and shearing deformations, thus leaving only the in-plane
strain deformation terms.  Elements for small and large displacements
are included but are restricted in the current release to elastic behavior.
The membrane element is included using the commands:
\begin{verbatim}
       MATErial,1
         MEMBrane
		 ....
\end{verbatim}
Required data is the material model (e.g., \texttt{ELAStic}) and the thickness
given by the \texttt{THICk} option.

The {\it point elements} are restricted to linear elastic behavior with
linear dashpot and point mass.
The point element material set is included using the commands:
\begin{verbatim}
       MATErial,1
         POINt
           MASS   m
           DAMPer c
           SPRIng k
           ORIEnt v_1,v_2,v_3 (ndm values)
\end{verbatim}
The \texttt{ORIEnt} vector is used to describe the direction cosines for the
orientation of the dashpot and spring.  The input order for \texttt{MASS, DAMPer,
SPRIng} and \texttt{ORIEnt} is arbitrary.  Unspecified terms are assumed zero.
The \texttt{ORIEnt} command is required if a damper or spring is specified.

The {\it thermal elements} are all based on a standard Galerkin (displacement)
formulation (i.e., no mixed model approximations are available).  The
element topologies available are identical to those for the displacement
form of the solid element.

At present the finite deformation material
models for the solid elements do not permit a coupled thermo-mechanical
analysis.  The small deformation models for elasticity do permit coupled
thermo-mechanical analyses to be performed using a partitionied solution
algorithm.  The material behavior for the thermal analysis is a linear
Fourier model.  Both isotropic and orthotropic models are available.
The thermal element is included using the commands:
\begin{verbatim}
       MATErial,1
         THERmal
		 ....
\end{verbatim}
Required data is the material model given by the \texttt{FOURier} option.

The {\it pressure load element} is specified by material set records:
\begin{verbatim}
       MATErial,1
         PRESsure
		 ....
\end{verbatim}
Loading is specified by options \texttt{LOAD} and, for large displacement
forms, \texttt{FINI}te.  Loading intensity may be associated with a proportional
loading function.

The {\it gap element} requires very little data to use.  The material
record is given as:
\begin{verbatim}
       MATErial,1
         GAP
           DIREction,x-dir
           DEGRee,n-dof
           PENAlty,pen-value
                                  ! blank termination record
\end{verbatim}
where \texttt{x-dir} is an integer ranging from 1 to ndm;
\texttt{n-dof} is the degree-of-freedom to which the gap condition is
applied and \texttt{pen-value} is a penalty parameter used to enforce
the constraint.
The gap element is used with a two node element where, if \texttt{x-dir}
is positive, the first to second node indicate
a positive direction to enforce the constraint and if \texttt{x-dir} is
negative the first to second node are taken in a negative coordinate
sense.  If \texttt{n-dof} has the same value as the absolute value of
\texttt{x-dir} the gap is treated in a phsical sense.  However, if
it is different, a 'gap' condition between the displacements of the
two nodes is used.  Thus, for the equal sense and a positive \texttt{x-dir}
a movement of the
second node in a positive \texttt{x-dir} relative to the first node
\texttt{opens} the gap without restraint or reduces the restraint force until
an opening takes place.  A negative motion of the second node relative
to the first closes the gap, and when the distance between the two is
negative or zero a penalty restraint is inserted.  If \texttt{x-dir} is
negative an opposite interpretation to the above is used.  If the penalty is
too small an overlap of the regions will exist and if it is considered
to be excessive either the penalty parameter value
should be increased or an augmented Lagrangian solution should be performed.

A fully Lagrange multiplier form of the gap element may also be used by
specifying a third node on the element.  One degree of freedom from
the third node (i.e., the dof \texttt{n-dof}) will be used to store the
Lagrange multiplier value.  Special care must be used when using any
Lagrange multiplier solution method as no diagonal results in the tangent
solution matrix for this equation.  To avoid solution difficulties it is
usually required to use a direct solution method in which the profile
(active column)solver is used -- this is the default solver or may be
specified using either of the commands:
\begin{verbatim}
       DIREct         ! In-core solver

       DIREct,BLOCk   ! Out-of-core blocked solver
\end{verbatim}
while in \texttt{BATCh} or \texttt{INTEractive} solution mode.

The specification of {\it user elements} must contain a number of an
element module which has been added to {\sl FEAP}.
Each user developed element module
is designated as \texttt{subroutine elmtnn(...)}, where \texttt{nn} ranges
from \texttt{01} to \texttt{50}.
Accordingly, a typical set of data for a user element \texttt{elmt12}
is given as:
\begin{verbatim}
       MATErial,1
         USER     12              ! Use elmt12(...) module
         xxxxxx                   ! Additional data records
                                  ! blank termination record
\end{verbatim}

The first two records of the \texttt{MATE}rial set always must be:

\begin{verbatim}
       MATErial ma
         type unum mset doflist
\end{verbatim}
where \texttt{ma} is the material set number, \texttt{type} is the element type
(e.g., solid, truss, etc.), \texttt{unum} is the user element number,
\texttt{mset} is the material set number given for each element (by default
it is the material number - this option permits two material types
to access the same element connection list), and \texttt{doflist} is the
list of global degree-of-freedoms to assign the internal element order (by
default this is the order 1,2,3,...,ndf).  For the standard elements contained
in {\sl FEAP} it is one needs only the \texttt{type} parameter unless
degrees-of-freedom are to be relocated (e.g., for thermal analysis).
