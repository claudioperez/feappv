\section{Coordinate and Transformation Systems}
\label{transform}

The coordinates in {\sl FEAP} must all be given in a cartesian system.
Input data, however, may be specified in {\it cartesian}, {\it polar}
(which in three dimensions is interpreted as cylindrical coordinates),
or {\it spherical} coordinate systems.  If polar or spherical coordinates
are used to define the nodal data using the {\tt COOR}\-dinate command,
they may be transformed to the required
cartesian form using the {\tt POLA}r or {\tt SPHE}rical commands, respectively.
Nodal coordinates generated with polar or spherical options in the {\tt BLOC}k
command do not require transformation. 
The data for a polar command is:
\begin{verbatim}
       POLAr
         NODE,n1,n2,inc
\end{verbatim}
\par\noindent
where {\tt n1} and {\tt n2} define a range of nodes and {\tt inc} is
the increment to be added to {\tt n1} for each step to {\tt n2}.  Alternatively,
all currently defined nodes may be transformed using the command
\begin{verbatim}
       POLAr
         ALL
\end{verbatim}
\par\noindent
The transformation is given by
$$ x_1 ~=~ x_0 \, + \, r \, \cos \, \theta$$
$$ x_2 ~=~ y_0 \, + \, r \, \sin \, \theta$$
and
$$ x_3 ~=~ z_0 \, + \, z$$
where $x_i$ are the cartesian coordinates, $r$, $\theta$, $z$ are the
polar (cylindrical) inputs, and $x_0$, $y_0$, $z_0$ are shifts defined
by the {\tt SHIF}t command given as
\begin{verbatim}
       SHIFt
         X_0,Y_0,Z_0
\end{verbatim}
By default $x_0$, $y_0$, $z_0$ are zero.

The {\tt SPHE}rical command is similar to the {\tt POLA}r command.  The
input records are specified as:
\begin{verbatim}
       COORdinate
         N NG R THETA PHI
\end{verbatim}
Transformations use the relations
$$ x_1 ~=~ x_0 \, + \, r \, \cos \, \theta \, \sin \, \phi$$
$$ x_2 ~=~ y_0 \, + \, r \, \sin \, \theta \, \sin \, \phi$$
and
$$ x_3 ~=~ z_0 \, + \, r \cos \, \phi$$

\subsection{Coordinate Transformation}

Cartesian systems may be translated, stretched, reflected and/or rotated using
the {\tt TRAN}\-sform command.
Any coordinates input after this command are transformed using
$$\left[
\begin{matrix} x_1 \\ x_2 \\ x_3 \end{matrix} \right] ~=~
\left[
\begin{matrix}
T_{11} & T_{12} & T_{13} \\
T_{21} & T_{22} & T_{23} \\
T_{31} & T_{32} & T_{33} \\
\end{matrix} \right] \,
\left[
\begin{matrix} \hat x_1 \\ \hat x_2 \\ \hat x_3
\end{matrix} \right] ~+~
\left[
\begin{matrix} x_0 \\ y_0 \\ z_0
\end{matrix} \right]$$
where $\hat x_i$ are the input values and the transformation parameters are
defined by the command sequence
\begin{verbatim}
       TRANsform
         T_11  T_12  T_13
         T_21  T_22  T_23
         T_31  T_32  T_33
         X_0   Y_0   Z_0
\end{verbatim}
which must appear before any coordinates (i.e., the $\hat x_i$) are specified.

The {\tt TRAN}sform command may be used as many times as needed.  In particular,
it may be used with a portion of a mesh (substructure) in an include file
to replicate repeated parts of meshes.  When a reflection is performed,
{\sl FEAP} notes the coordinate transformation does not have a positive
determinant and resequences the node numbers on elements to maintain
positive jacobians (provided the original data is correct in its local
cartesian basis - $\hat{x}_i$).

\section{Looping to Replicate Mesh Parts}
\label{mloop}

Many models for problems analyzed by finite element methods have mesh parts
which are similar except for stretching and rotation transformations.
\textsl{FEAP} provides input capabilities to generate the model using
\texttt{LOOP}-\texttt{NEXT} commands.  The basic input structure is given
by the command sequence
\begin{verbatim}
       LOOP,n
         ...
       NEXT
\end{verbatim}
where $n$ defines the number of times to repeat the commands contained
within the loop.  The value of $n$ may be a constant or a parameter.
Any standard \textsl{FEAP} mesh commands may be used between the \texttt{LOOP}
and \texttt{NEXT} statements, however, it is easiest to use commands which do
not require explicit definitions for node or element numbers.

\begin{figure}[hb!]
\epsfysize=2.2in
\centerline {\hfil \epsfbox{figs/lp2blk.eps} \hfil}
\caption{Two blocks using \texttt{LOOP-NEXT} commands}
\label{floop.0}
\end{figure}
A simple example is the repetition of two blocks of identical elements
in which the material number is different.  Assume first that a file
named \texttt{Imblock} is constructed which contains the commands
\begin{verbatim}
       BLOCk
         CART n1 n2 0 0 ma
           1   0  0
           2   a  0
           3   a  b
           4   0  b

       PARAMeter
          ma = ma + 1

\end{verbatim}
Then a second file is given which defines the initial values of parameters
and the looping control.  This file is given by the statements shown in
Table \ref{tloop.1}
where we note the use of the loop using the \texttt{TRANsform} command. 
\begin{table}[ht!]
\begin{center}
\begin{verbatim}
       FEAP * * Two block problem
         0  0  0  2  2  4
       PARAmeters
         a  = 5
         b  = 4
         n1 = 6
         n2 = 3
         ma = 1
 
       LOOP,2
         INCLude Imblock
         TRANsform
           1 0 0
           0 1 0
           0 0 1
           a 0 0
       NEXT
       MATE 1
         SOLID
           ELAStic ISOTropic 1000 0.25

       MATE 2
         SOLID
           ELAStic ISOTropic 2000 0.25

       END
\end{verbatim}
\caption{\texttt{LOOP-NEXT} mesh construction}
\label{tloop.1}
\end{center}
\end{table}
The above example produces the mesh shown in Fig. \ref{floop.0} and
is trivial (also not much is gained over a construction
using two block commands directly).

\begin{figure}[ht!]
\epsfysize=3.5in
\centerline {\hfil \epsfbox{figs/wheel5.eps} \hfil}
\caption{Disk with holes}
\label{floop.1}
\end{figure}
A more involved example is shown 
in Fig. \ref{floop.1} for a disk containing circular holes.  This
example was constructed using the commands shown in Table \ref{tloop.2}.
\begin{table}[ht!]
\begin{center}
\begin{verbatim}
       LOOP 5
         TRANSform
           cosd(th)  sind(th)    0
          -sind(th)  cosd(th)    0
               0         0       1
               0         0       0
         INCLude Iwseg
         TRANSform
           cosd(th)  sind(th)    0
           sind(th) -cosd(th)    0
               0         0       1
               0         0       0
         INCLude Iwseg
         PARAmeter
           th = th + 72

       NEXT
\end{verbatim}
\caption{\texttt{LOOP-NEXT} disk mesh construction}
\label{tloop.2}
\end{center}
\end{table}
The file \texttt{Iwseg} contains the mesh for one part of the repeating
mesh as shown in Fig. \ref{floop.2}.
\begin{figure}[ht!]
\epsfysize=2.0in
\centerline {\hfil \epsfbox{figs/wheelsg.eps} \hfil}
\caption{Mesh segment for disk with holes}
\label{floop.2}
\end{figure}

Many more involved meshe constructs may be considered using the
\texttt{LOOP}-\texttt{NEXT} commands.
