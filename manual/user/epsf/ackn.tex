\chapter[Acknowledgments]{ACKNOWLEDGMENTS}
\label{ackn}

The {\sl FEAP} system has been in continuous development since 1976.  The
program has been used in the training of a large number of graduate students at
the University of California, Berkeley, as well as, at many other 
institutions worldwide.  Numerous contributions have been made to {\sl FEAP}
by several individuals during the last twenty plus years.  Indeed, without these
contributions the program would not have many of the capabilities present today.  
I am sure that oversights will result in the following acknowledgments --
I apologize in advance for the missing ones.

Many improvements
related to element technology and solution strategies were contributed by
the late Professor Juan C. Simo, both while he was at Berkeley as well as
during his time at Stanford University.
Juan was in all respects a co-developer of the program.
The basic strategies for solving
non-linear problems resulted from contributions by Juan during many
years of interactions.  Element technology for finite deformation solid
elements for the mixed and enhanced strain are based on the insights of Juan,
especially his perceptions related to use of three field Hu-Washizu
type formulations.  The large motion beams and shells also resulted
from his research contributions and subsequent contributions by Professor
Adnan Ibrahimbegovic.  The coupled flexible-rigid
body formulation included in {\sl FEAP} was initiated with Juan and further
developed by Dr. Alecia Chen.
Juan also added the rotational update routines involving quaternions
to support the structural elements and the rigid body work.

Additional improvements to {\sl FEAP} resulted from 
contributions by present and former students, visiting scholars,
and users of earlier
versions of the program.  Listed in alphabetical order the contributors were:
Ferdinando Auricchio (University of Pavia, Italy),
Jerry Goudreau (Lawrence Livermore National Laboratory),
Anna Haraldsson (Institute for Mechanics at Darmstadt University of
Technology, Germany),
Tom Hughes (Stanford University),
Eric Kasper (California State Polytechnic University, San Luis Obispo),
Tod Larsen (Duke University),
Barham Nour-Omid,
Karl Schweitzerhof (Institute for Mechanics, University of Karlsruhe, Germany),
Jerome Solberg (UCB),
Tom Spelce,
Peter Wriggers (Hannover University of Technology, Germany),
Giorgio Zavarise (University of Torino, Italy),

Many additional contributions and suggestions for improvements
have been made by Berkeley colleagues
Francisco Armero, Jon Bray, Greg Fenves, Filip Filippou, Sanjay Govindjee, and
Panayiotis (Panos) Papadopoulos are gratefully acknowledged.

Finally, I acknowledge the inspiration and guidance of Olek Zienkiewicz during
the last thirty years.  His insights and contributions have greatly enhanced
finite element analysis methods and provided a motivation for the development
of a tool to investigate new areas and methodologies.

To all of the above contributors (and those I have inadvertently failed to
cite) I am deeply grateful.  Your contributions not only improved {\sl FEAP} but
usually led to my better understanding of the issues related to
developing software to solve problems in computational mechanics.

\vspace{0.2in}
\noindent
Robert L. Taylor \\
Berkeley, California \\
August 15, 2000

