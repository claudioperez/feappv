\chapter[Material Models]{MATERIAL MODELS}
\label{matmods}

The data input for each of the current material options is summarized
below.  Tables are included to indicate which elements types can use
each type of data option.  As much as possible a common format and
notation is used for all the element types.

\section{Orthotropic Linear Elastic Models}
\label{orthel}

The orthotropic linear elastic material model in {\sl FEAP} is
given by
\begin{equation}
 \hat{\Bf{\epsilon}}
~=~ \hat{\B{C}} \, \hat{\Bf{\sigma}} ~+~ \hat{\Bf{\epsilon}}^{th}
\end{equation}
where $\hat{\Bf{\epsilon}}$ and $\hat{\Bf{\sigma}}$ are the stress and strain
arrays in the principal material directions and the elastic compliance array
in principal material directions is:
\begin{equation}
\hat{\B{C}} ~=~ \left[
\begin{matrix}
~\frac{1}{E_1} & - \frac{\nu_{12}}{E_2} & - \frac{\nu_{31}}{E_1} & 0 & 0 & 0 \\
- \frac{\nu_{12}}{E_2} & ~\frac{1}{E_2} & - \frac{\nu_{23}}{E_3} & 0 & 0 & 0 \\
- \frac{\nu_{31}}{E_1} & - \frac{\nu_{23}}{E_3} & ~\frac{1}{E_3} & 0 & 0 & 0 \\
0 & 0 & 0 & \frac{1}{G_{12}} & 0 & 0 \\
0 & 0 & 0 & 0 & \frac{1}{G_{23}} & 0 \\
0 & 0 &0 & 0 & 0 & \frac{1}{G_{31}} \\
\end{matrix} \right]
\end{equation}
with $E_i$ elastic moduli in principal directions, $\nu_{ij}$ Poisson ratios
for strains measured in the principal directions.

The thermal strain is given by:
\begin{equation}
\hat{\Bf{\epsilon}}^{th} ~=~ \left[
\begin{matrix}
\alpha_1 \\ \alpha_2 \\ \alpha_3 \\ 0 \\ 0 \\ 0 \\
\end{matrix} \right]
\, \Delta T ~=~ \hat{\Bf{\alpha}} \, \Delta T
\end{equation}
where
\begin{equation}
\Delta T ~=~ T\, - \, T_0 \, ,
\end{equation}
$\alpha_i$ are coefficients of linear thermal expansion and
$T_0$ is a specified reference temperature.

The orthotropic material parameters are input as shown in Table \ref{tab131a}
using the commands {\tt ELAStic,ORTHotropic} and {\tt THERmal,ORTHotropic}.
For 2-dimensional analyses the values of $G_{23}$ and $G_{31}$ are not used
and may be omitted.
The angle the principal directions makes with the $x_1$ (or $x$)
axis for plane stress
and plane strain analyses or the $r$ axis for axisymmetric analysis may be
specified using the material {\tt ANGLe} command
as shown in Table \ref{tab134a}.
Using this angle {\sl FEAP} transforms the input material compliances to
\begin{equation}
\B{C} = \B{R}^T \, \hat{\B{C}} \, \B{R}
\end{equation}
and converts the constitutive equation to the form
\begin{equation}
\Bf{\sigma} ~=~ \B{D} \, \Bf{\epsilon} ~+~ {\Bf{\beta}}^{th}
\end{equation}
where
\begin{equation}
\B{C} = \B{C}^{-1}
\end{equation}
and
\begin{equation}
{\Bf{\beta}}^{th} = - \, \B{D} \, {\Bf{\epsilon}}^{th}
\end{equation}

Material data is given by the command set:
\begin{verbatim}
       MATErial,1
         SOLId
         ELAStic ORTHotropic e1 e2 e3 nu12 nu23 nu31 g12 g23 g31
         THERmal ORTHotropic a1 a2 a3 t0
                                  ! blank termination record
\end{verbatim}
Additional data options and parameters are defined in Table \ref{tab131a}.

\begin{table}[ht!]
\begin{center}
\begin{tabular}{| l | l | l |} \hline
Command & Type & Parameters \\ \hline
ELAStic & ORTHotropic & $E_1, E_2, E_3, \nu_{12},
\nu_{23}, \nu_{31}, G_{12}, G_{23}, G_{31}$ \\
ELAStic & ISOTropic & $E, \nu$ \\
ELAStic & TRANsverse & $E_1, E_2, \nu_{12}, \nu_{31}, G_{31}$ \\
DAMPing & RAYLeigh  & a0 , a1 \\
PLAStic & MISEs       & $Y_0, Y_{\infty}, \beta$ \\
PLAStic & HARDening   & $H_{iso}, H_{kin}$ \\
VISCoelastic & & $\mu_i, \tau_i$ \\
THERmal & ORTHotropic & $\alpha_1, \alpha_2, \alpha_3, T_0$ \\
THERmal & ISOTropic   & $\alpha, T_0$ \\ \hline
FOURier & ORTHotropic & $K_1, K_2, K_3, c$ \\
FOURier & ISOTropic   & $K, c$ \\ \hline
DENSity &       & $\rho$ \\ \hline
ANGLe   &       & $\psi$ \\ \hline
\end{tabular}
\end{center}
\caption{Material Model Data Inputs}
\label{tab131a}
\end{table}

\section{Isotropic Linear Elastic Models}
\label{isosel}

The isotropic models require less data since now only two independent
elastic parameters are needed to define $\hat{\B{C}}$. These are
taken as Young's modulus, $E$, and Poisson's ratio, $\nu$ and
for an isotropic material the elastic compliance array is
\begin{equation}
\hat{\B{C}} ~=~ \left[
\begin{matrix}
~\frac{1}{E} & - \frac{\nu}{E} & - \frac{\nu}{E} & 0 & 0 & 0 \\
- \frac{\nu}{E} & ~\frac{1}{E} & - \frac{\nu}{E} & 0 & 0 & 0 \\
- \frac{\nu}{E} & - \frac{\nu}{E} & ~\frac{1}{E} & 0 & 0 & 0 \\
0 & 0 & 0 & \frac{1}{G} & 0 & 0 \\
0 & 0 & 0 & 0 & \frac{1}{G} & 0 \\
0 & 0 &0 & 0 & 0 & \frac{1}{G} \\
\end{matrix} \right]
\end{equation}
with the shear modulus related through
\begin{equation}
G ~=~ \frac{E}{2 \, (1 \,+ \, \nu)}
\end{equation}
For thermally isotropic materials the expansion coefficient is constant in
all directions,thus
\begin{equation}
\hat{\Bf{\epsilon}}^{th} ~=~ \left[ \begin{matrix}
\alpha \\ \alpha \\ \alpha \\ 0 \\ 0 \\ 0 \\
\end{matrix} \right]
\, \Delta T
\end{equation}
The data input for the isotropic models is input using
the {ELASt\-ic,ISOTrop\-ic}
and {\tt THERm\-al,ISOTrop\-ic} commands as shown in Table \ref{tab131a}.
For an isotropic material it is not necessary to perform transformation of
the elastic arrays since $\B{C} ~=~ \hat{\B{C}}$.

The types of elements for which elastic material models may be specified
is indicated in Table \ref{tab131b}.

\section{Isotropic Finite Deformation Elastic Models}
\label{isofel}

Finite deformation hyperelastic models are provided in {\sl FEAP} for
several stored energy functions which are written in terms of deformation
measures.

Deformation measures may be defined in terms of positions in the reference
configuration, denoted by $\Bf{X}$, and positions in the current
configuration, denoted by $\Bf{x}$.  The motion of a point from the
reference to the current configuration at time $t$ is expressed as
\begin{equation}
\Bf{x} = \B{\varphi} ( \Bf{X}, t )
\end{equation}
The deformation gradient is then defined as
\begin{equation}
\Bf{F} = \frac{\partial \B{\varphi}}{\partial \Bf{X}}
\end{equation}

Additional measures of deformation are given by the right Cauchy-Green
deformation tensor
\begin{equation}
\Bf{C} = \Bf{F}^T \Bf{F}
\end{equation}
and the left Cauchy-Green deformation tensor
\begin{equation}
\Bf{b} = \Bf{F} \, \Bf{F}^T
\end{equation}
A measure of strain is provided by the Green strain
\begin{equation}
\Bf{E} = \frac{1}{2} \left( \Bf{C} - \Bf{1} \right)
\end{equation}

The hyperelastic model expressed in terms of the strain energy function
as a function of $\Bf{C}$ is given by
\begin{equation}
\Bf{S} = \frac{\partial W(\Bf{C})}{\partial \Bf{C}}
\end{equation}
where $W$ is a {\it stored energy} function.
Stress in the current configuration may be deduced by transformation (pushing)
the stress.  Accordingly
\begin{equation}
\B{\sigma} = \frac{1}{J} \, \Bf{F} \, \Bf{S} \, \Bf{F}^T
\end{equation}
Isotropic models may be
expressed in terms of the invariants of the deformation tensor.  Accordingly,
the three principal invariants given by
\begin{equation}
I_C = {\rm tr} \, \Bf{C}
\end{equation}
\begin{equation}
II_C = \frac{1}{2} \left( I_C^2 - {\rm tr} \, \Bf{C}^2 \right)
\end{equation}
and
\begin{equation}
III_C = \det \Bf{C} = J^2
\end{equation}
where $J$ is $\det \bf{F}$ may be used to write the stored energy function.

The deformation tensor may also be expressed in terms of principal
stretches, $\lambda_A$, and their associated eigenvectors, $\Bf{N}_A$.
Accordingly, one may write
\begin{equation}
\Bf{C} = \sum_{A=1}^3 \, \lambda_A^2 \, \Bf{N}_A \otimes \Bf{N}_A
\end{equation}
The invariants
are then given by
\begin{equation}
I_C = \lambda_1^2 + \lambda_2^2 + \lambda_3^2
\end{equation}
\begin{equation}
II_C = \lambda_1^2 \lambda_2^2 + \lambda_2^2 \lambda_3^2
+ \lambda_3^2 \lambda_3^2
\end{equation}
and
\begin{equation}
III_C = \lambda_1^2 \lambda_2^2 \lambda_3^2
\end{equation}
Alternatively, the three principal stretches may be used directly to
write the stored energy function.  Both forms are used in {\sl FEAP}.

In {\sl FEAP} the isotropic elastic moduli are defined to match results from
the small strain isotropic elastic models. Accordingly, they only
require specification of
the elastic modulus, $E$, and Poisson ratio, $\nu$ or, equivalently, the
bulk modulus, $K$, and shear modulus, $G$.

\subsection{St. Venant-Kirchhoff and Energy Conserving Model}

The simplest model is a St. Venant-Kirchhoff model given by:
\begin{equation}
\Bf{S} = \Bf{D} \, \Bf{E}
\end{equation}
where $\Bf{S}$ is the second Piola-Kirchhoff stress, $\Bf{E}$ is the Green
strain, and $\Bf{D}$ are the elastic moduli.
This model may be deduced from the stored energy function
\begin{equation}
W = \frac{1}{2} \, \Bf{E}^T \Bf{D} \, \Bf{E}
\end{equation}
For isotropy the model may be written in terms of the invariants of $\Bf{E}$;
however, the $\Bf{D}$ will have the same structure as in an isotropic linear
elastic material (see above).

The material data set for isotropy is given as

The same constitution is used to implement an energy-momentum
algorithm for finite deformation analyses.  The data to perform an
energy-momentum conserving form is given as
\begin{verbatim}
       MATErial,1
         SOLId
         FINIte
         ELAStic CONServing e nu
                                  ! blank termination record
\end{verbatim}
Note that the location of the {\tt FINI}te command is order insensitive.  Also
recall that the finite deformation designation may be given for all elements
as {\tt GLOB}al data.a

The St. Venant-Kirchhoff and energy conserving models should not be used
for problems
in which very large compressive deformations are expected.  The model gives
identical results to the small deformation isotropic model if deformations
are truly infinitesimal.  It is also a good model to use if the displacements
are large, but strains remain small.  For situations where large elastic
deformations are involved the {\tt NEOH}ookean, {\tt MNEO}hookean,
or {\tt OGDE}n models should be used.

\begin{table}[ht!]
\begin{center}
\begin{tabular}{| l | l | l |} \hline
Command & Type & Parameters \\ \hline
ELAStic & NEOHookean & $E, \nu$ \\
ELAStic & MNEOhookean & $E, \nu$ \\
ELAStic & OGDEn & $K, C_1, a_1, C_2, a_2, C_3, a_3$ \\
ELAStic & STVK & $E, \nu$ \\
ELAStic & CONServe & $E, \nu$ \\ \hline
\end{tabular}
\end{center}
\caption{Finite Deformation Elastic Material Data Inputs}
\label{tab131c}
\end{table}

\subsection{Neo-Hookean and Modified Neo-Hookean Models}

The stored energy functions for the finite deformation hyperelastic models
are split into two parts.  The first part defines the behavior associated
with volume changes and the second the behavior for other deformation states.
The volumetric deformation part is defined by a function $U(J)$ multiplied
by a material parameter.  The volumetric function is taken as
\begin{equation}
U(J) = ln^2 J
\end{equation}

The neo-Hookean hyperelastic model is deduced from the stored energy function
\begin{equation}
W = \left( K  - \frac{2}{3} G \right) \, U(J)
+ \frac{1}{2} \, G \, \left( I_C - 3 \right)
\end{equation}
The parameters $K$ and $G$ are equivalent to the small strain bulk and shear
moduli, respectively.  Input data for the model is specified in terms of
the equivalent small strain modulus ($E$) and Poisson ratio ($\nu$) such that
the $K$ and $G$ are given by
\begin{equation}
\label{enu}
K = \frac{E}{3 \, ( 1 - 2 \nu )} ~~~~;~~~~ G = \frac{E}{2 \, ( 1 + \nu )}
\end{equation}
The data set to use this form is given by
\begin{verbatim}
       MATErial,1
         SOLId
         FINIte
         ELAStic NEOHook E nu
                                  ! blank termination record
\end{verbatim}

A modified form to the neo-Hookean model is also available.  The modified
form defines a stored energy function which splits the two terms into
pure volumetric behavior and pure deviatoric behavior.  To accomplish
the construction the deformation gradient is split into the product of
a volumetric and deviatoric form as
\begin{equation}
\Bf{F} = \Bf{F}_{vol} \, \Bf{F}_{dev}
\end{equation}
where
\begin{equation}
\Bf{F}_{vol} = J^\frac{1}{3} \Bf{1}
\end{equation}
and
\begin{equation}
\Bf{F}_{dev} =  {J}^{- \, \frac{1}{3}} \Bf{F}
\end{equation}
The stored energy function is then given as
\begin{equation}
W = K  \, U(J)
+ \frac{1}{2} \, G \, \left( {J}^{- \, \frac{2}{3}} I_C - 3 \right)
\end{equation}

The parameters $K$ and $G$ are again specified by their small strain
equivalent $E$ and $\nu$ defined in Eq. \ref{enu}.

The data set to use the modified form is given by
\begin{verbatim}
       MATErial,1
         SOLId
         FINIte
         ELAStic MNEOHook E nu
                                  ! blank termination record
\end{verbatim}

\subsection{Ogden Model}

{\sl FEAP} also contains a model for hyperelastic behavior which is
expressed directly in terms of the principal stretches, $\lambda_A$.
This model has a stored energy function expressed in the form:
\begin{equation}
W = K  \, U(J) + \sum_{A=1}^3 \, w(\lambda_A,J)
\end{equation}
and is based on the {\it Valanis-Landel hypothesis} (\cite{valan,ogden}).
The {\it deviatoric} principal stretches are defined as
\begin{equation}
\tilde{\lambda}_A = J^{- \frac{1}{3}} \lambda_A
\end{equation}
and used to write the scalar stored energy functions as
\begin{equation}
w(\tilde{\lambda}_A) = \sum_j \frac{C_j}{a_j} \left(
\tilde{\lambda}_A^{a_j} - 1 \right)
\end{equation}
where $j$ ranges from 1 to 3 terms.
The data input for the Ogden model is given as
\begin{verbatim}
       MATErial,1
         SOLId
         FINIte
         ELAStic OGDEn K C_1 a_1 C_2 a_2 C_3 a_3
                                  ! blank termination record
\end{verbatim}

\subsection{Logarithmic Stretch Model}

An alternative principal stretch model is defined by strains expressed as
\begin{equation}
\epsilon_A = \log{ \lambda_A }
\end{equation}
The stored energy function for this form is identical to the small
strain isotropic model expressed in terms of the principal strains.
Accordingly,
\begin{equation}
W( \lambda_A ) = \frac{1}{2} \, 
\left( K - \frac{2}{3} G \right) \left( \sum_{A=1}^3 \epsilon_A \right)^2
+ G \, \sum_{A=1}^3 \epsilon_A^2
\end{equation}

This form of the finite strain implementations in {\sl FEAP} is the only
one which may be used in elastic-plastic analyses.
It is not recommended for situations involving hyperelastic behavior at
large strains.
The data input for the Logarithmic stretch  model is given as
\begin{verbatim}
       MATErial,1
         SOLId
         FINIte
         ELAStic log E nu
                                  ! blank termination record
\end{verbatim}
Note that the descriptor {\it log} is placed to fill the second field, it is
not used explicitly by {\sl FEAP}.

\section{Rayleigh Damping}
\label{raydamp}

The effects of damping may be included in transient solutions
assuming a damping matrix in the form
\begin{equation}
\B{C} ~=~ a_0 \, \B{M} ~+~ a_1 \, \B{K}
\end{equation}
This defines a form called {\it Rayleigh Damping}.
The input for this form of damping is given by:
\begin{verbatim}
       MATERIAL
         .....
           DAMPing RAYLeigh a0 a1
\end{verbatim}

This command is only included for small deformation elements using a
linear elastic material model and is used only for
time dependent solutions specified by a {\tt TRANsient} solution command.
Rayleigh damping may also be defined for modal solutions
(Section \ref{damping}).

\section{Viscoelastic Models}
\label{viscmod}

Materials which behave in a time dependent manner require extensions of the
elastic models cited above.  One model is given by viscoelasticity where
stress may be related to strain through either differential or integral
constitutive models (e.g., see {\sl FEAP} Theory Manual).  At present, the
implementation in {\sl FEAP} is restricted to isotropic viscoelasticity in
which time effects are included for the deviatoric stress components only.
If we split the stress as:
\begin{equation}
\B{\sigma} ~=~ \sigma_{vol} \, \Bf{1} + \B{\sigma}_{dev}
\end{equation}
where $\sigma_{vol}$ represents the spherical part given by
$\frac{1}{3} \sigma_{kk}$ and $\B{\sigma}_{dev}$ is the deviatoric stress part.
Similarly the strain may be split as
\begin{equation}
\B{\epsilon} ~=~ \frac{1}{3} \, \theta \, \Bf{1} + \B{\epsilon}_{dev}
\end{equation}
where $\theta$ is the trace of the strain ($\epsilon_{kk}$)
and $\B{\epsilon}_dev$ is the deviatoric part.

The constitutive equation may now be written as
\begin{equation}
\B{\sigma}_dev ~=~ 2 \, G \, \int_{-\infty}^t
\mu(t - \tau ) \, \frac{d \B{\epsilon}_{dev}}{d \tau} \, d \tau
\end{equation}
where $mu(t)$ is a relaxation function.  The term $G mu(t)$ is called the
relaxation modulus.   In {\sl FEAP} the relaxation function is represented
by a Prony series
\begin{equation}
\mu(t) ~=~  \mu_0 + \sum_i \, \mu_i \, \exp \frac{-t}{\tau_i}
\end{equation}
The $\tau_i$ are time parameters defining the relaxation times
for the material and the $\mu_i$ are constant terms.  Currently, {\sl FEAP}
limits the representation to three (3) exponential terms.
The value of $\mu_0$ is computed from
\begin{equation}
\mu_0 ~=~  1 - \sum_i \, \mu_i
\end{equation}
Thus, the elastic modulus $G$ represents the instantaneous elastic response
and $G \, \mu_0$ the equilibrium, or long time, elastic modulus.
Only positivef $\mu_i$ are permitted and care must
be taken in defining the $\mu_i$ to ensure that $\mu_0$ is positive or zero.
If $\mu_0$ is zero the response can have steady creep and never reach an
equilibrium configuration.

Input data for a one term model is given by the followin data set:
\begin{verbatim}
       MATErial,1
         SOLId
         ELAStic ISOTropic 30e+06 0.3
         VISCoelastic term1 0.7   10.0
                                  ! blank termination record
\end{verbatim}
Here $\mu_1$ is 0.7 giving a $\mu_0$ of 0.3.  The relaxation time
is 10 time units.

After defining the response by the above exponential representation, the
constitutive equations are integrated in time by assuming the
strain rate is constant over each time step.  The method for integration
uses exact integraion over each time step and leads to a simple recursion
for each exponential term (e.g., see \cite{taylorpg}).  Additional details
are also given in the {\sl FEAP} Theory manual.

For finite deformation problems the viscoelastic parameters are related
to the second Piola-Kirchhoff stress and Green strain.

\section{Plasticity Models}
\label{plasmod}

Classical elasto-plastic material models are included in {\sl FEAP} for
small and finite deformation problems.  The finite deformation model
is based on logarithmic principal stretches and product split of the
deformation gradient.  This leads to a form which is similar to that
for small strains.  Accordingly, here we limit our discussion to the
small strain problem.

The stress for an elasto-plastic material may be computed by assuming
an additive split of the strain as
\begin{equation}
\B{\epsilon} = \B{\epsilon}^{el} + \B{\epsilon}^{pl}
\end{equation}

An associative flow rule is assumed so that the plastic strain rate
may be computed from a {\it yield function}, $F$, as
\begin{equation}
\dot{\B{\epsilon}}^{pl} = \dot{\gamma} \,
\frac{\partial F}{\partial \B{\sigma}}
\end{equation}
The relation may be integrated in time using a backward Euler (implicit)
time integration to compute a discrete form of the problem.

Isotropic and kinematic hardening are also added to the model.  The kinematic
hardening is limited to a linear form where it is assumed that
\begin{equation}
\B{\alpha} = H_{kin} \, \B{\epsilon}^{pl}
\end{equation}
where $\B{\alpha}$ is the back stress and $H_{kin}$ is the kinematic hardening
modulus.  The isotropic hardening is taken in a linear and saturation form
as
\begin{equation}
Y(e^{pl}) = Y_{\infty} + (Y_0 - Y_{\infty}) \exp (- \beta \, e^{pl})
+ H_{iso} \, e^{pl}
\end{equation}
where $Y_0$ is the initial uniaxial yield stress, $Y_{\infty}$ a stress
at large values of strain, $\beta$ a delay constant, and $H_{iso}$ is a linear
isotropic hardening modulus.  The accumulated plastic strain is computed
from
\begin{equation}
e^{pl} = \int_0^t \, \dot{\gamma} \, d \tau
\end{equation}

In {\sl FEAP} the discrete problem is solved using a closest point return
map algorithm (e.g., see \cite{simort85,simort86,simo:hughes}).

Input properties for a simple material with no saturation hardening and
linear isotropic hardening is given by:
\begin{verbatim}
       MATErial,1
         SOLId
         ELAStic ISOTropic 30e+06 0.3
         PLAStic MISEs     30e+03
         PLAStic HARDening 3000   0
                                  ! blank termination record
\end{verbatim}

\section{Heat Conduction Material Models}
\label{heat}

For thermal analysis a linear heat conduction capability is included in
{\sl FEAP}.  The constitutive equation is given by a linear Fourier
model in which the heat flux $\B{q}$ is related to the thermal gradient $\B{h}$
by the relation
\begin{equation}
\B{q} ~=~ - \, \B{K} \, \B{h}
\end{equation}
where, in the principal directions,
\begin{equation}
\B{K} ~=~ \left[ \begin{matrix}
K_1 & 0 & 0 \\
0 & K_2 & 0 \\
0 & 0 & K_3 \\
\end{matrix} \right]
\end{equation}
The values for $K_i$ and, for transient problems, the specific heat, $c$,
are specified using the command {\tt FOUR}\-ier,{\tt ORTH}o\-trop\-ic
or for the case where all are equal using {\tt FOUR}\-ier,{\tt ISOT}rop\-ic
as indicated in Table \ref{tab131a}

\begin{table}[ht!]
\begin{center}
\begin{tabular}{| l | c | c | c | c | c | c | c |} \hline
Command & Solid & Truss & Frame & Plate & Shell & Membrane & Thermal \\ \hline
ELAStic       & X & X & X & X & X & X & - \\
PLAStic       & X & X & F & - & S & - & - \\
VISCoelastic  & X & X & F & - & - & - & - \\
THERmal       & X & X & X & X & - & X & - \\
FOURier       & X & X & - & - & - & - & X \\
DENSity       & X & X & X & X & X & X & X \\
ANGLe         & X & - & - & X & X & X & X \\ \hline
\end{tabular}
\end{center}
\caption{Material Commands vs. Element Types. X=all, F=finite, S=small.}
\label{tab131b}
\end{table}

\section{Mass Matrix Type Specification}
\label{masstype}

The mass matrix for continuum problems and the specific heat matrix
for thermal problems may be either a {\it consistent}, {\it lumped},
or {\it interpolated} form.  By default {\it FEAP} uses a lumped matrix.
If $\B{M}_{cons}$ is the consistent matrix and $\B{M}_{lump}$ is the diagonal
lumped matrix, the interpolated matrix is defined as:
\begin{equation}
\B{M}_{interp} ~=~ (1 \,-\, a) \, \B{M}_{cons} ~+~ a \, \B{M}_{lump}
\end{equation}
The type of mass and, where required,
the parameter $a$ are input using the {\tt MASS} command as shown in
Table \ref{tab132a} and the elements which are affected by the command
are indicated in Table \ref{tab132b}.

\begin{table}[ht!]
\begin{center}
\begin{tabular}{| l | l | l |} \hline
Command & Type & Parameters \\ \hline
MASS & LUMPed       & {}  \\
MASS & CONSistent   & {}  \\
MASS & {}           & $a$ \\ \hline
\end{tabular}
\end{center}
\caption{Material Model Mass Related Inputs}
\label{tab132a}
\end{table}

\begin{table}[ht!]
\begin{center}
\begin{tabular}{| l | c | c | c | c | c | c | c |} \hline
Command & Solid & Truss & Frame & Plate & Shell & Membrane & Thermal \\ \hline
MASS          & X & X & X & X & - & - & X \\ \hline
\end{tabular}
\end{center}
\caption{Mass Command vs. Element Types}
\label{tab132b}
\end{table}

\section{Element Cross Section and Load Specification}
\label{xsect}

\subsection{Resultant formulations}

The plane stress and structural elements require specification of
cross-section information.
For the plane stress, plate, and shell
elements this is a thickness which is specified using the {\bf
THICkness} command as shown in Table \ref{tab133a}.
The plate element also permits the effects
of transverse shear deformation to be included and, if this is
different than the $5/6$ default value it is also given using the thickness
command.
For the truss and frame elements it is necessary to provide cross-sectional
property for area, and for the frame elements, flexural effects as indicated
in Table \ref{tab133a}.

Element loads for surface pressure and body force are input using the
{\tt LOAD} and {\tt BODY} force commands as shown in Table \ref{tab133a}.

The types of elements affected by the {\tt THICkness}, {\tt LOAD} and
{\tt BODY} commands is indicated in Table \ref{tab133b}.

\begin{table}[ht!]
\begin{center}
\begin{tabular}{| l | l | l |} \hline
Command & Type & Parameters \\ \hline
THICkness & {}      & $h, \kappa$ \\
CROSS     & section & $A, I_{xx}, I_{yy}, I_{xy},
J_{zz}, \kappa_x, \kappa_y$ \\ \hline
BODY & forces   & $b_1, b_2, b_3$ \\
LOAD & normal   & $q$ \\ \hline
\end{tabular}
\end{center}
\caption{Cross Section and Body Force Inputs}
\label{tab133a}
\end{table}

\begin{table}[ht!]
\begin{center}
\begin{tabular}{| l | c | c | c | c | c | c | c |} \hline
Command & Solid & Truss & Frame & Plate & Shell & Membrane & Thermal \\ \hline
THICkness     & X & - & - & X & X & X & X \\
CROSs         & - & X & X & - & - & - & - \\ \hline
BODY          & X & X & X & - & X & X & - \\
LOAD          & - & - & - & X & X & X & - \\ \hline
\end{tabular}
\end{center}
\caption{Geometry and Loads vs. Element Types}
\label{tab133b}
\end{table}

\subsection{Section integration formulations}

Structural element behavior may also be defined by numerical integration
over the cross section using the {\tt SECT}ion command.
For the three-dimensional truss and frame elements
the cross section may be defined by alternate forms which include: {\tt TUBE},
a thin circular tube; {\tt RECT}angle, a rectangular solid section; {\tt WIDE}
flange, a wide flange composite section; {\tt CHAN}nel, a channel composite
section; {\tt ANGL}e, an angle composite section; and {\tt CIRC}le, a solid
circular section.  The basic form of a section command is:

\begin{verbatim}
       SECTion TYPE (EV(i),i=1,6)
\end{verbatim}
The data parameters {\tt EV} for each type are summarized in
Table \ref{tab133c}.

\begin{table}[ht!]
\begin{center}
\begin{tabular}{| l | c | c | c | c | c | c | c |} \hline
TYPE & EV(1) & EV(1) & EV(2) & EV(3) & EV(4) & EV(5) & EV(6) \\ \hline
TUBE & $r$   & $t$   & $n$   & $q_n$ &   &   &  \\
RECTangle & $y_b$ & $z_b$ & $y_t$ & $z_t$ & $q_y$ & $q_z$ & \\
WIDE flange & $h$ & $f_t$ & $f_b$ & $t_t$ & $t_b$ & $t_w$ & \\
CHANnel     & $h$ & $f_t$ & $f_b$ & $t_t$ & $t_b$ & $t_w$ & \\
ANGLe       & $h$ & $f$ & $t_h$ & $t_f$ &   &   &   \\
CIRCle      & $r$ & $q$ &   &   &  &  & \\ \hline
\end{tabular}
\end{center}
\caption{Types and data for integrated cross-sections.}
\label{tab133c}
\end{table}

In Table \ref{tab133c} $r$ denotes radius, $t$ thickness, $h$ height, $f$
flange width, $t$ top, $b$ bottom, $q$ quadrature order, and $n$ number
of segments.  The cross section is assumed to lie in a $y$-$z$ plane.

\section{Miscellaneous Material Set Parameter \\ Specifications}
\label{miscmat}

In addition to the above material, geometric and loading parameters the
values for some other variables may also be set.

It is possible to replace global parameters for the type of two dimensional
analysis using the {\tt PLANe STREss}, {\tt PLANe STRAin}, or {\tt AXISymmetric}
commands.  Similarly the global value for the temperature degree of
freedom to use in coupled thermo-mechanical problems may be changed
for the current material set using the {\tt TEMPerature} command.  The
formats are indicated in Table \ref{tab134a} and the affected element types in
Table \ref{tab134b}.
The values for the number of quadrature points (in elements, not cross sections)
to be used for computing arrays and element
outputs may be set using the {\tt QUADrature} command.
Generally, {\sl FEAP} will select an appropriate order of quadrature to
be used in computing the arrays and for output of element quantities.  Thus,
care should be used in changing the default values.

{\sl FEAP} includes capabilities to solve finite deformation problems
using the {\tt SOLId}, {\tt FRAMe}, {\tt TRUSs}, {\tt SHELl}, {\tt MEMBrane}
and {\tt GAP} elements.  To select the
finite deformation element it is necessary to use the {\tt FINIte} deformation
option instead of the default {\tt SMALl} deformation option.
This may be done for all materials using the {\tt GLOBal} command.
There are three different element technologies
which may be selected {\tt DISPlacement} (which is the default), {\tt MIXEd},
or {\tt ENHAnced strain} types. The data options for these are indicated
in Table \ref{tab134a} and the affected element types in Table \ref{tab134b}.

\begin{table}[ht!]
\begin{center}
\begin{tabular}{| l | l | l |} \hline
Command & Type & Parameters \\ \hline
QUADrature  & {}       & $n_{array}, n_{output}$ \\
PENAlty     &          & $k_{pen}$ \\
ADAPtive    & ERROr    & $\eta$ \\
TEMPerature & {}       & $T_{dof}$ \\ \hline
SMALl       & deformation & {} \\
FINIte      & deformation & {} \\
NONLinear   & {}       & {} \\ \hline
DISPlacment & {}       & {} \\
MIXEd       & {}       & {} \\
ENHAnced    & strain   & {} \\ \hline
PLANe       & STREss   & {} \\
PLANe       & STRAin   & {} \\
AXISymmetric & {}      & {} \\ \hline
\end{tabular}
\end{center}
\caption{Miscellaneous Material Model Inputs}
\label{tab134a}
\end{table}

\begin{table}[ht!]
\begin{center}
\begin{tabular}{| l | c | c | c | c | c | c | c |} \hline
Command & Solid & Truss & Frame & Plate & Shell & Membrane & Thermal \\ \hline
QUADrature      & X & - & -  & - & X & X & X \\
PENAlty         & - & - & -  & - & - & - & - \\
ADAPtive ERRor  & X & - & -  & - & - & - & - \\
TEMPerature     & X & X & X  & X & X & - & - \\ \hline
SMALl           & X & X & X  & - & X & X & - \\
FINIte          & X & X & X  & - & X & X & - \\
NONLinear       & - & X & X  & - & - & - & - \\ \hline
DISPlacement    & X & - & -  & - & - & - & - \\
MIXEd           & X & - & -  & - & - & - & - \\
ENHAnced        & X & - & -  & - & - & - & - \\ \hline
PLANe STREss    & X & - & -  & - & - & - & X \\
PLANe STRAin    & X & - & -  & - & - & - & X \\
AXISymmetric    & X & - & -  & - & - & - & X \\ \hline
\end{tabular}
\end{center}
\caption{Miscellaneous Material Commands vs. Element Types}
\label{tab134b}
\end{table}

\chapter[Discrete Elements]{DISCRETE ELEMENTS}
\label{diselm}

{\sl FEAP} has options to add discrete mass, damping, and stiffness
terms to a problem.  The mass terms are added as {\it lumped} terms
for each degree of freedom.  The data for discrete masses are
included as input in the form
\begin{verbatim}
       MASS
         m,mg,M1_m,M2_m,M3_m ... Mndf_m
         n,ng,M1_n,M2_n,M3_n ... Mndf_n
                                  ! blank termination record
\end{verbatim}
where {\tt m, n} are node numbers, {\tt mg, ng} are generation increments
to nodes, and {\tt Mi\_m, Mi\_n} are discrete mass values.
Generation of missing nodes will take place if the {\tt mg} value is
non-zero.  Mass values will be interpolated linearly for the {\it i-th}
degree of freedom.

Damping values also may be given for any node.  Each linear damper is
fixed at one end and attached to a degree of freedom at the other.
Damping values are input as
\begin{verbatim}
       DAMPer
         m,mg,C1_m,C2_m,C3_m ... Cndf_m
         n,ng,C1_n,C2_n,C3_n ... Cndf_n
                                  ! blank termination record
\end{verbatim}
where{\tt Ci\_m, Ci\_n} are discrete damper values for the {\it i-th}
degree of freedom.

Finally, linear stiffness (springs)
may be attached to any node.  Each linear spring is
fixed at one end and attached to a degree of freedom at the other.
Damping values are input as
\begin{verbatim}
       STIFness
         m,mg,K1_m,K2_m,K3_m ... Kndf_m
         n,ng,K1_n,K2_n,K3_n ... Kndf_n
                                  ! blank termination record
\end{verbatim}
where{\tt Ki\_m, Ki\_n} are discrete stiffness values for the {\it i-th}
degree of freedom.
