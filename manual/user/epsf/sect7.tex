\section{Global Data}
\label{global}

{\sl FEAP} uses the {\tt GLOB}al command to specify data which is common
to all elements.  For
example, in two-dimensional applications it is possible to specify that
all elements should select a plane stress, a plane strain, or an axisymmetric
representation.  If the example problem is to be solved as a plane
strain problem, the global data is specified as:
\begin{verbatim}
       GLOBal
         PLANe STRAin
                            ! blank termination record
\end{verbatim}
\par\noindent
Thus, by changing the record describing the type of two dimensional
analysis the system elements will all use the same type of behavior.  If
it is desired, for some modeling reason, to have one type of element
use a different formulation the global data can be ignored by specifying
the particular type of analysis needed as part of the {\tt MATErial} property
data.

A problem in solid mechanics may be designated as
{\tt SMALl} or {\tt FINIte} deformation using global commands.
In addition, the variable used for temperature in a coupled thermo-mechanical
analysis and the {\tt REFErence} vector or node for three dimensional
problems using structural frame elements may be defined globally.
Options also exist for users to add their own global options.

Problems for which {\it ground accelerations} are specified as proportional
load tables may be solved using a specified pattern of amplification factors,
$f_i$, for each degree of freedom.  These factors are applied to a discrete
masss input using the {\tt MASS} command using the command
\begin{verbatim}
       GLOBal
         GROUnd factors f_1 f_2 ... f_ndf
                            ! blank termination record
\end{verbatim}

For small deformation transient analysis damping effects may be introduced
for use by the solid and structural elements as Rayleigh damping.  Each
material may have different Rayleigh damping parameters (see \ref{raydamp}).
Alternatively, the Rayleigh parameters may be assigned as global values 
using the commands
\begin{verbatim}
       GLOBal
         RAYLeigh damping a0 a1
                            ! blank termination record
\end{verbatim}
\par\noindent
where the parameters {\tt a0} and {\tt a1} are defined in Section \ref{raydamp}.
The global damping value may also be used for modal solutions as described
in Section \ref{damping}.
