\chapter[Mesh Manipulation Commands]{MESH MANIPULATION COMMANDS}
\label{manip}

Once an initial mesh is completely defined it may be further processed
to merge nodes with the same coordinates using the {\tt TIE} command, or
force a sharing of degrees-of-freedom using the {\tt LINK} and/or {\tt ELINk}
commands.  These commands may be given in any order immediately following
the mesh {\tt END} command.  While they may be in any order the data is
first saved in temporary files and {\sl FEAP} later executes the commands in
a definite order.  Thus if data printing is on information may appear in
a different order than given in the input file.

\section{The TIE Command}
\label{tie}

The ability to merge nodes which have the same coordinates permits
the generation of a mesh in separate parts without having to consider
a common node numbering system between the individual parts.
The {\tt TIE} command permits merging
based on material set numbers, region numbers, a range of node numbers,
or on all the defined node numbers.
The latter is achieved by entering the command as:
\begin{verbatim}
       TIE
\end{verbatim}
without any parameters.  A range of node numbers to search may be
specified also.  For example, if the merge is to be done only for
nodes numbered between 34 and 65 the command is issued as:
\begin{verbatim}
       TIE,, 34, 65
\end{verbatim}
It is however, not possible to merge nodes from two different ranges of
numbers.

It is also possible to merge parts based on material numbers.  For example,
if a problem with two bodies is generated using material set 1
for body one and material set 2 for body two,
a merge may be achieved for the parts of each body
without any possibility of merging nodes in body one to those in
body two.  This is achieved using the commands:
\begin{verbatim}
       TIE MATErial 1  1
       TIE MATErial 2  2
\end{verbatim}
If it is desired to tie nodes for materials 1 and 2 together, the command
\begin{verbatim}
       TIE MATErial 1  2
\end{verbatim}
may be used.

Alternatively, the nodes to be merged may be associated with a region.  In
this option it is necessary to include {\tt REGIon} commands as part of the
element generation process (i.e., using either {\tt ELEMent} or {\tt BLOCk}).
An example of this option is explained as part of Example 4 in
the {\sl Example Manual}.
The basic command to merge parts in {\it Region m} to those in {\it Region n}
is
\begin{verbatim}
       TIE REGIon m  n
\end{verbatim}
The parameters {\tt m} and {\tt n} may have the same or different values.

When the tie option is used one node from a merged pair is deleted from
the mesh and its number on the element connections replaced by the retained
number.  It is not possible to display or output values for the
deleted node.  If printing is in effect at the end of the mesh
generation process, the nodes deleted are listed in the {\sl FEAP} data output
file.  For plots, the projections will also be performed assuming the deleted
node does not exist.
This is different than creating common solution
values for degree of freedoms associated with two nodes
using the {\tt LINK} and {\tt ELINk} options described below.  In a link
option the node is not deleted, and thus projections may create a
different solution at the two nodes.

\section{The LINK and ELINk Commands}
\label{link}

The link options may be used to make the solution of one or more of
the degrees-of-freedom associated with two nodes have the same value.
This option is useful in creating repeating type solutions, that is, those
in which the solution on a surface is repeated on an identical surface
with a different location.
The link may be performed based on node numbers using the {\tt LINK}
command, or for all nodes on an edge using the {\tt ELINk} command.

\section{The PARTition Command}
\label{part}

The solution of coupled problems may be performed by {\sl FEAP} either
as a total problem or by partitioning the problem into separate smaller
problems.  For example, the solution of a coupled thermo-mechanical problem
may be performed by solving the thermal and the mechanical parts of the
problem separately for each solution time.

By default all the 
degree-of-freedoms in a problem are assigned to the first partition.
To assign individual degree-of-freedoms to different partitions
the command
\begin{verbatim}
       PARTition
         P1,P2,P3,...
\end{verbatim}
is inserted after the mesh {\tt END} command and before the first solution
command.  Values for {\tt P-i} must be between 1 and 4
and must be specified for all active
degree-of-freedoms (i.e., the total number specified on the control 
record described in Section \ref{start}).

The use of partitions can significantly reduce the cost of solving some
coupled problems since the size of the coefficient matrix for each of the
parts is much smaller than that of the total problem.  Furthermore, in
{\sl FEAP} the type of algorithm to solve each part can be set individually.
Thus, it is possible to set a static option for the mechanical part and a
transient algorithm for the thermal part.  The individual parts may also
be symmetric whereas the fully coupled problem is often unsymmetric.
Such is not the case for the fully coupled solution algorithm where some
care must be given to prevent numerical problems.  One is the different
order of the equations which may be treated as described next.

\section{The ORDEr Command}
\label{order}

In the solution of coupled problems the individual parts often involve
solution of transient problems with different orders.
For example, in the solution of coupled thermo-mechanical problems a
transient heat problem is of first order while a transient mechanical
problem is of second order.  Solution of these problems in a fully
coupled mode requires use of a single transient algorithm.  Thus, for
example to solve the fully coupled transient thermo-mechanical problem
can be performed using any of the algorithms defined in Section \ref{trans2}.
There can be numerical problems in solving the thermal problem
if large numbers of time steps are used.  The problems originate from the
missing second order rate term in the thermal problem which may cause the
numerical acceleration to generate an overflow and thus terminate an
analysis.  To avoid this difficulty the {\tt ORDE}r command may be inserted
as
\begin{verbatim}
       ORDEr
         O1,O2,O3,....
\end{verbatim}
where the {\tt Oi} define the order of the transient term for each
degree-of-freedom and for {\sl FEAP} must be defined between 0 (static)
and 2 (second order).  Thus, for a coupled thermo-mechanical problem
the temperature degree-of-freedom should be set to one (1) and the
displacement degree-of-freedoms to two (2).
