\chapter[Command Language Programs]{COMMAND LANGUAGE PROGRAMS}
\label{command}

{\sl FEAP} performs solution steps based upon user specified {\it command
language statements}.  The program provides commands which can be used to solve
problems using standard algorithms, such as linear static and transient
methods and Newton's method to solve non-linear problems.
Appendix B of the Users Manual describes all the programming
commands which are included in the current system.  These commands are
combined to define the solution algorithm desired.

To enter the solution command language part of {\sl FEAP}
the user issues the command BATCh or for an
interactive execution mode the command INTEractive.  A solution
is terminated by the command END (QUIT or just Q
also may be used in interactive mode).

Thus, the input file must contain at least one set of
\begin{verbatim}
       BATCh
        ....  ! Solution specification steps
       END
\end{verbatim}
or
\begin{verbatim}
       INTEractive
\end{verbatim}
for any solution process to be possible.

More than one BATCh-END and/or
INTE\-rac\-tive-END sequence may be used during the solution process.

The set of {\it basic} solution commands is:
\begin{verbatim}
       ACCE   CAPT   CHEC   DEBU   DISP   DT     EIGE   EPRI
       FORM   INIT   LIST   LOOP   MASS   MESH   NEXT   NOPR
       PARA   PLOT   PRIN   PROP   REAC   SHOW   SOLV   STRE
       SUBS   TANG   TIME   TOL    TPLO   TRAN   UTAN   VELO
\end{verbatim}
Descriptions to use the above commands are contained in Appendix B.
All commands available in an installed program may be displayed during
an interactive mode of solution by issuing
the command {\tt MANUal,,3} followed by a {\tt HELP} command.
However, with the basic
set of commands given above quite sophisticated solution algorithms may
be constructed.  Each of the commands may be issued in a
lower or upper case mode.  For example, a command which
always should be issued when first solving a problem
is the {\tt CHECk} command.  In either a batch or interactive mode,
the command is issued as:
\begin{verbatim}
       CHECk       !perform check of mesh correctness
\end{verbatim}
This command instructs {\sl FEAP} to make
basic checks for correctness of the mesh data prepared by the user{\footnote
{The check part of user developed elements must be implemented for the
check command to work properly}}.
One of the basic checks is an assessment of the element volume (or area)
at each node based on the specified sequence of
element nodes.  If the volume Jacobian of an element is negative or zero
at a node a diagnostic will be written to the output file.  If
all the volumes (or areas) are negative most of the system element routines
will perform a resequencing of the nodes and repeat the check.
If the resequencing gives no negative results the mesh will be
accepted as correct.

A check
also may reveal and report element nodes which have {\it zero} volume.
This may be an error or may result from merging nodes on quadrilaterals
to form triangles.  This is an acceptable way to make 3-node triangular
elements from 4-node quadrilateral elements, but in other cases may not
produce elements preserving the order of interpolation of the quadrilateral.
{\it It is the responsibility of the analyst to check correctness of finite
element solution software.} One good procedure is the patch test in which
basic polynomial solutions, for which the user can compute exactly the correct
solution (by hand), can be checked (see Chapter 11 in Volume 1 of
Zienkiewicz and Taylor for a description of the patch test).

The {\tt CHECk} command should always be used in situations where
either a new mesh
has been constructed or modifications to the element connection lists have
been made. {\it No analysis should be attempted for a mesh with}
{\it negative volumes as incorrect results will result}.  Note,
however, that if a correct mesh is produced after
the {\tt CHECk} command resequences nodes,
the data in the input file is {\it not corrected}, consequently, it
will be necessary to always use a {\tt CHECk} command when solving a
problem with this data input file.
Since the amount of output from a {\tt CHECk} can be quite large, it
is recommended that the user correct the mesh for subsequent
solutions.  Alternatively, it is possible to produce a new
input data file, which is correct, using the {\tt OUTMesh} command.
The command is given as:
\begin{verbatim}
       OUTMesh     !Output current mesh to "Ifile".opt
\end{verbatim}
The output is written to a file with the same name as the input file
but with a {\tt .opt} extender added. The file
only includes the mesh coordinates, element connections, boundary
restraint codes, and nodal force and displacement
values.  It is necessary to append the
material set data and any solution steps.  It is not necessary to specify
any {\tt TIE} commands as the results from merges are incorporated
as part of the mesh produced by the {\tt OUTMesh} command.

\section{Problem Solving}

Each problem is solved by using a set of the command language statements
which together form the {\it algorithm} defining the particular solution
method employed.  The commands to solve a linear static problem are:
\begin{verbatim}
       BATCh               !initiate batch execution
         TANG              !form tangent matrix
         FORM              !form residual
         SOLVe             !solve equations
         DISPlacement,ALL  !output all displacements
         STREss,ALL        !output all element stresses
         REACtion,ALL      !output all nodal reactions.
       END                 !end of batch program
\end{verbatim}
The command sequence
\begin{verbatim}
       TANG
       FORM
       SOLVe
\end{verbatim}
is the basic solution step in {\sl FEAP} and for simplicity (and efficiency)
may be replaced by the single command
\begin{verbatim}
       TANG,,1
\end{verbatim}
This single statement is more efficient in numerical operations
since it involves only a single
process to compute all the finite element arrays, whereas the three
statement form requires one for {\tt TANG} and a second for {\tt FORM}.
Thus,
\begin{verbatim}
       BATCh               !initiate batch execution
         TANG,,1           !form and solve
         DISPlacement,ALL  !output all displacements
         STREss,ALL        !output all element stresses
         REACtion,ALL      !output all nodal reactions.
       END                 !end of batch program
\end{verbatim}
is the preferred solution form.
Some problems have tangent matrices which are unsymmetric.  For these situations
the {\tt TANG}ent command should be replaced by the {\tt UTAN}gent command.
The statements {\tt DISP}lacement, {\tt STRE}ss, and {\tt REAC}tion control
information which is written to the output file and to the screen.
The commands {\tt PRIN}t and {\tt NOPR}int may be used to control or prevent
information appearing on the screen - information always goes to the
output file.  Printing to the screen is the default mode.  See Appendix B
for the options to control the displacement, stress, and reaction outputs.

Additional commands may be added to the program given above.  For example,
inserting the following command after the solution step (i.e., the
{\tt TANG,,1} command) will produce a screen plot of the mesh:
\begin{verbatim}
       PLOT,MESH           !plot mesh
\end{verbatim}
Further discussion for plotting is given in Chapter \ref{plot}.

\subsection{Solution of Non-linear Problems}
\label{nonlin}

The solution of non-linear problems is often performed using Newton's
method which solves the problem
\begin{equation}
\B{R}(\B{u}) = \B{0}
\end{equation}
using the iterative algorithm

\begin{enumerate}
\item Set initial solution
\begin{equation}
\B{u}^0 = \B{0}
\end{equation}
\item Solve the set of equations
\begin{equation}
\B{K} \, \Delta\B{u}^i = \B{R}(\B{u}^i)
\end{equation}
where
\begin{equation}
\B{K} = - \, \frac{\partial \B{R}}{\partial \B{u}}
\end{equation}
\item
Update the solution iterate
\begin{equation}
\B{u}^{i+1} = \B{u}^i + \Delta\B{u}^i
\end{equation}
\end{enumerate}
The steps are repeated until a norm of the solution is less than some tolerance.

{\sl FEAP} implements the Newton algorithm using the
following commands:
\begin{verbatim}
       LOOP,iter,10     !perform up to 10 Newton iterations
         TANG,,1        !form tangent, residual and solve
       NEXT,iter        !proceed to next iteration
\end{verbatim}
The tolerance used for controlling the solution is
\begin{equation}
E^i = \Delta{\B{u}^i} \cdot \B{R}^i
\end{equation}
with convergence assumed when
\begin{equation}
E^i < tol \, E^0
\end{equation}
The value of the tolerance is set using the {\tt TOL} command (default
is $10^{-12}$).

While the sample above specifies 10 iterations, fewer will be used if
convergence is achieved.  Convergence is tested during the {\tt TANG,,1}
command.  If convergence is achieved, {\sl FEAP} transfers to the
statement following the {\tt NEXT} command.
If convergence is not achieved in 10 iterations, {\sl FEAP}
exits the loop, prints a {\tt NO CONVERGENCE} warning,
and continues with the next statement.
For the algorithm given above, the only difference between a converged
and non-converged exit from the loop is the number of iterations
used.  However, if there are commands inserted between the {\tt TANG} and
{\tt NEXT} statements they are not processed for the iteration in which
convergence is achieved.
Obviously, solutions which do not converge during a time step may produce
inaccurate results in the later solution steps.
Consequently, users should check the output log of non-linear solutions
for any {\tt NO CONVERGENCE} records.

Remarks:
\begin{enumerate}
\item
Blank characters before the first character in a command are ignored
by {\sl FEAP}, thus,
the indenting of statements shown is optional but provides for clarification
of key parts in the algorithm.

\item
In the above loop command the {\tt ITER} in the
second field is given to provide clarity.  This is
optional; the field may be left blank.
\end{enumerate}

By replacing the Newton steps
\begin{verbatim}
       LOOP,iter,10
         TANG,,1
       NEXT,iter
\end{verbatim}
with
\begin{verbatim}
       TANG             !form tangent only
       LOOP,iter,10     !perform 10 modified Newton iterations
         FORM           !form residual
         SOLVe          !solve linearized equations
       NEXT,iter        !proceed to next iteration
\end{verbatim}
a modified Newton algorithm results.  The modified Newton method
forms only one tangent
and each iteration is performed by computing and solving the residual equation
with the same tangent.  When {\sl FEAP} forms the tangent while in a
direct solution
of equations mode the triangular factors are also computed
so that the {\tt SOLVe} only performs re-solutions during
each iteration.  While a modified Newton method involves fewer computations
during each iteration it often
requires substantially more iterations to achieve a
converged solution.  Indeed, if the tangent matrix is an accurate linearization
for the non-linear equations, the asymptotic rate of convergence for a
Newton method is quadratic, whereas a modified Newton method is often
only linear (if the residual equation set
is linear the tangent matrix is constant
and both the Newton and modified Newton methods should converge after one
iteration, that is, iteration two should produce a residual which is
zero to within the computer precision).

The {\sl FEAP} command language is
capable of defining a large number of standard algorithms.  Each user is urged
to carefully study the complete set of available commands and the options
available for each command.
In order to experiment with the capabilities of the language, it is suggested
that small problems be set up to test any proposed command language program
and to ensure that the desired result is obtained.

\subsection{Solution of linear equations}
\label{eqsoln}

The use of Newton's method results in a set of linear algebraic equations
which are solved to give the incremental displacements.  {\sl FEAP} includes
several options for solving linear equations.  The default solution scheme
is the variable band, profile scheme discussed in
Chapter 15 of {\it The Finite Element Method, Vol 1}, 4th edition.
This solution scheme may be used to solve problems where the incremental
displacements are either in real arithmetic or in complex arithmetic.  The
coefficient matrix of the linear equations results in large storage requirements
within the computer memory.  A profile optimization scheme is available to
renumber the equations in an attempt to minimize this storage.  The solution
command {\tt OPTI}\-mize may be used to perform the profile minimization.
A summary of the results is given and may be compared to that without
optimization.  If necessary, the optimization may be omitted using
the command {\tt OPTI,OFF}.  The default solution is without optimization.

For problems in which the memory requirements exceed that which is provided
in the program (i.e., the dimensioned size of the blank common), there are
alternatives which require reduced amounts of storage.  The alternatives
are available for problems in real arithmetic only.  For problems with
symmetric coefficient arrays (i.e., those for which the command {\tt TANG}ent
is used to form the array), a sparse solver may be used.  The sparse solver
is activated by issuing the solution command {\tt DIRE}ct,{\tt SPAR}se before
the first use of the {\tt TANG}ent command.  {\it WARNING: If the sparse
solver requires more space than dimensioned in blank common the current
version of the program can crash with no error message printed in a file
or to the screen}.  Alternatively, the profile solution scheme may be
employed with a blocking scheme used to retain unneeded parts of the
coefficient array during the solution process.  This option is may be
requested using the command {\tt DIRE}ct,{\tt BLOC}k.  There must be
sufficient free disk capacity on the computer to store the total coefficient
array.  The speed of solution is reduced using this option by the need
to write and read data from the hard disk drive.  The blocked solution scheme
may be used for either symmetric or unsymmetric coefficient arrays.

The final option available is an iterative, preconditioned conjugate
gradient scheme (PCG method).  The PCG method is applicable to symmetric,
positive definite coefficient arrays only.  Thus, only the {\tt TANG}ent
command may be used.  The PCG with diagonal preconditioner is requested
by the command {\tt ITER}\-a\-tion before the first {\tt TANG}\-ent
command.  A block nodal preconditioner may be requested using the
command {\tt ITER,BPCG}.  Experience to date suggests the iteration
method is effective and efficient only for three dimensional linear
elastic solids problems.  Success has been achieved when the solids
are directly connected to shells and beam; however, use with thin
shells has resulted in very slow convergence - rendering the method
ineffective.  Use with non-linear material models (e.g., plasticity)
has not been successful in static problem applications.  Use of the PCG
method in dynamics improves the situation if a mass term is available for
each degree of freedom (i.e., lumped mass on frames with no rotational
mass will probably not be efficient).

\section{Transient Solutions}
\label{trans}

{\sl FEAP} provides several alternatives to construct transient solutions.
To solve a non-linear time dependent problem
using Newton's method with a time integration method
the following commands may be issued:
\begin{verbatim}
       DT,,0.01           !set time increment to 0.01
       TRANsient,method   !specify "method" for time stepping
       LOOP,time,12       !perform 12 time steps
         TIME             !advance time by 'dt' (i.e., 0.01)
         LOOP,iter,10     !perform up to 10 Newton iterations
           TANG,,1        !form tangent, residual and solve
         NEXT,iter        !proceed to next iteration
         DISP,,1,12       !report displacements at nodes 1-12
         STRE,NODE,1,12   !report stresses at nodes 1-12
       NEXT,time          !proceed to next time step
\end{verbatim}
In addition to output for {\tt DISP}lacement, transient algorithms permit
the output of {\tt VELO}\-city and {\tt ACCE}l\-er\-a\-tion (see Appendix B).

{\sl FEAP} provides several alternatives to construct transient solutions.
A transient solution is performed by giving the solution command language
statement
\begin{verbatim}
       TRANsient,method
\end{verbatim}
The type of transient solution to be performed
depends on the {\it method} option specified.
{\sl FEAP} solves three basic types of transient formulations:

\subsection{Quasi-static solutions}
\label{static}

The governing equation to be solved by the quasi-static option is expressed
as:
\begin{equation}
\B{R} (t) ~=~ \B{F}(t) ~-~ \B{P}( \B{u}(t) ) ~=~ {\bf 0}
\end{equation}
where, for example, the $\B{P}$ vector is given by the stress divergence
term of a solid mechanics problem as:
\begin{equation}
\B{P}( \B{u}(t) ) ~~=~~ \B{P}_{\sigma} ~=~
\int_{\Omega} \B{B}^T \, \Bf{\sigma} \, dV
\end{equation}
The solution options for this form are:

\begin{enumerate}
\item
The default algorithm which solves
\begin{equation}
\B{R} (t_{n+1}) ~=~ \B{F}(t_{n+1}) ~-~ \B{P}( \B{u}(t_{n+1}) ) ~=~ {\bf 0}
\end{equation}
using the commands
\begin{verbatim}
       LOOP,time,nstep
         TIME
         LOOP,Newton,niters
           TANG,,1
         NEXT
         ... Outputs
       NEXT
\end{verbatim}
The default option does not require a {\tt TRAN}sient command; however it
may also be specified using the command
\begin{verbatim}
       TRANsient,OFF
\end{verbatim}

\item
Quasi-static solutions may also be solved using a generalized
mid\-point con\-fig\-uration for the residual equation.  This option
is specified by the command
\begin{verbatim}
       TRANsient,STATic,alpha
\end{verbatim}
and solves the equation
\begin{equation}
\B{R} (t_{n+\alpha}) ~=~ \B{F}(t_{n+\alpha}) ~-~ \B{P}( \B{u}(t_{n+\alpha}) )
~=~ {\bf 0}
\end{equation}
where
\begin{equation}
\B{u}(t_{n+\alpha}) ~=~
\B{u}_{n+\alpha} ~=~ (1 - \alpha) \, \B{u}_n \,+\, \alpha \, \B{u}_{n+1}
\end{equation}
and
\begin{equation}
\B{F}(t_{n+\alpha}) ~=~
\B{F}_{n+\alpha} ~=~ (1 - \alpha) \, \B{F}_n \,+\, \alpha \, \B{F}_{n+1}
\end{equation}

The parameter $\alpha$ must be greater than zero; the default value is 0.5.
Setting $\alpha$ to 1 should produce answers identical to those from
option 1.
The transient option to be used must be given prior to specifying the time loop
and solution commands shown above.
\end{enumerate}

\subsection{First order transient solutions}
\label{trans1}

The governing equation to be solved for first order transient solutions
is expressed as:
\begin{equation}
\B{R} (t) ~=~ \B{F}(t) ~-~ \B{P}( \B{u}(t) , \dot{\B{u}} (t) ) ~=~ {\bf 0}
\end{equation}
where, for example, $\B{u}$ are the nodal temperatures $\B{T}$ and the $\B{P}$
vector is given by:
\begin{equation}
\B{P}~~=~~ \int_{\Omega} (\nabla N)^T \, \B{q} \, dV
~+~ \B{C} \, \dot{\B{T}}
\end{equation}
with $\B{C}$ the heat capacity matrix.

The solution options for this form are:

\begin{enumerate}
\item
{A backward Euler method which solves the problem
\begin{equation}
\B{R} (t_{n+1})
~=~ \B{F}(t_{n+1}) ~-~ \B{P}( \B{u}(t_{n+1}) , \dot{\B{u}}_{n+1} (t) )
~=~ {\bf 0}
\end{equation}
where
\begin{equation}
\dot{\B{u}}_{n+1} ~=~ \frac{1}{\Delta t} [ \B{u}_{n+1} \,-\, \B{u}_n ]
\end{equation}
The command:
\begin{verbatim}
       TRANsient,BACK
\end{verbatim}
is used to specify this solution option.}

\item
{A generalized midpoint method which solves the problem
\begin{equation}
\B{R} (t_{n+\alpha}) ~=~ \B{F}(t_{n+\alpha})
~-~ \B{P}( \B{u}(t_{n+\alpha}) , \dot{\B{u}}_{n+\alpha} (t) ) ~=~ {\bf 0}
\end{equation}
where
\begin{equation}
\dot{\B{u}}_{n+\alpha} ~=~ \frac{1}{\Delta t} [ \B{u}_{n+1} \,-\, \B{u}_n ]
\end{equation}
This solution option is selected using the command
\begin{verbatim}
       TRANsient,GEN1,alpha
\end{verbatim}
where $0 < \alpha \le 1$ (default is 0.5); $\alpha = 1$ should produce answers
identical to those from the backward Euler option.}
\end{enumerate}

\subsection{Second order transient solutions}
\label{trans2}

The governing equation to be solved for second order transient solutions
is expressed as:
\begin{equation}
\B{R} (t) ~=~
\B{F}(t) ~-~ \B{P}( \B{u}(t) , \dot{\B{u}} (t) , \ddot{\B{u}}(t) )
~=~ {\bf 0}
\end{equation}
where, for example, the $\B{P}$ vector is given by:
\begin{equation}
\B{P}~~=~~ \B{P}_{\sigma} ~+~ \B{C} \, \dot{\B{u}} ~+~ \B{M} \, \ddot{\B{u}}
\end{equation}
with $\B{C}$ the damping and $\B{M}$ the mass matrix.

The solution options for second order problems are:

\begin{enumerate}
\item
{A Newmark method \cite{newmark} which solves the problem
\begin{equation}
\B{R} (t_{n+1}) ~=~ \B{F}(t_{n+1})
~-~ \B{P}( \B{u}_{n+1} , \B{v}_{n+1} , \B{a}_{n+1}) ~=~ {\bf 0}
\end{equation}
where
\begin{equation}
\B{v}_n ~=~ \dot{\B{u}}_n ~~~~~;~~~~~ \B{a}_n ~=~ \ddot{\B{u}}_n
\end{equation}
with updates computed as:
\begin{equation}
\B{u}_{n+1} ~=~ \B{u}_n \,+\, {\Delta t} \, \B{v}_n \,+\,
{\Delta t}^2 \, [ \, (0.5 - \beta ) \, \B{a}_n
\,+\, \beta \, \B{a}_{n+1} ,\ ]
\end{equation}
and
\begin{equation}
\B{v}_{n+1} ~=~ \B{v}_n \,+\, {\Delta t} \,[\, (1 - \gamma) \, \B{a}_n
\,+\, \gamma \, \B{a}_{n+1} \, ]
\end{equation}
in which $\beta$ and $\gamma$ are parameters controlling stability and
numerical dissipation.  The command
\begin{verbatim}
       TRANsient,NEWMark
\end{verbatim}
is used to select this integration scheme.  Optionally, the command
\begin{verbatim}
       TRANsient
\end{verbatim}
also selects the Newmark algorithm.
Default values are $\beta = 0.25$ and $\gamma = 0.5$.

The second order problem using the Newmark method may require special
care in computing the initial state if non-zero initial conditions
or loading terms exist.  To compute the initial state it is necessary to
first compute a mass matrix and then the initial accelerations.  The
commands are
\begin{verbatim}
       TRANsient,NEWMark
       INITial (DISPlacements and/or RATEs)
       FORM,ACCEleration
       LOOP,time,nstep
         TIME
         LOOP,Newton,niters
           TANG,,1
         NEXT
         ... Outputs
       NEXT
\end{verbatim}
It is also necessary to use this sequence for the following method.
If $\B{F} (0)$, $\B{u} (0)$, and $\B{v} (0)$ are zero, the
{\tt FORM}, {\tt ACCE}leration command should be omitted to conserve
memory resources.}

In the above the setting of any non-zero initial displacements or rates
may be specified using the {\tt INIT}ial command.  The initial command
requires additional data which in a {\tt BATC}h solution option appears
immediately after the {\tt END} command.  In an interactive mode a user
receives a prompt to specify the data.

\item
{A Hilber-Hughes-Taylor (HHT) method \cite{hht77} which solves the problem
\begin{equation}
\B{R} (t_{n+\alpha}) ~=~ \B{F}(t_{n+\alpha})
~-~ \B{P}( \B{u}_{n+\alpha} , \B{v}_{n+\alpha} , \B{a}_{n+\alpha}) ~=~ {\bf 0}
\end{equation}
where
\begin{equation}
\B{u}_{n+\alpha} ~=~ (1 - \alpha) \, \B{u}_n \,+\, \alpha \, \B{u}_{n+1}
\end{equation}
\begin{equation}
\B{v}_{n+\alpha} ~=~ (1 - \alpha) \, \B{v}_n \,+\, \alpha \, \B{v}_{n+1}
\end{equation}
\begin{equation}
\B{a}_{n+\alpha} ~=~ \B{a}_{n+1}
\end{equation}
The displacement and velocity quantities at $t_{n+1}$ are updated using
the Newmark formulas given above.  This solution option is selected
using the command
\begin{verbatim}
       TRANsient,ALPHa,beta,gamma,alpha
\end{verbatim}
The alpha parameter should be specified between zero and 1.
Default values are $\beta = 0.5$, $\gamma = 1$, and $\alpha = 0.5$.}

\item
{An energy conserving form of the alpha
method \cite{simoem92a,simoem92b,gonz96} (i.e., similar to the
HHT method) with the acceleration given as:
\begin{equation}
{\B{a}_{n+\alpha}} ~=~
\frac{1}{\Delta t} \,[\, \B{v}_{n+1} \,-\, \B{v}_n \,]
\end{equation}
This solution option is selected using the command
\begin{verbatim}
       TRANsient,CONServe,beta,gamma,alpha
\end{verbatim}
The alpha parameter should be specified between zero and 1.
Default values are $\beta = 0.5$, $\gamma = 1$, and $\alpha = 0.5$.
Note that the conserving form does not involve the accelerations
in the equations of motion (only displacement and
velocity); consequently, it is not necessary to compute initial accelerations
as in the Newmark and HHT methods.  For linear problems the conserving method
gives identical results (except for accelerations) as the Newmark method;
however, the parameters to achieve the equality are different.  Default
parameters should achieve equality provided Newmark is started by accounting
for any non-zero accelerations at time zero.}

\item
{An explicit solution to the equations
\begin{equation}
\B{R} (t_{n+1}) ~=~ \B{F}(t_{n+1})
~-~ \B{P}_{\sigma}( \B{u}_{n+1} , \B{v}_{n+1} )
~-~ \B{M} \, \B{a}_{n+1} ~=~ {\bf 0}
\end{equation}
which uses the Newmark formulas with $\beta = 0$ and specifies $gamma$ by
the command
\begin{verbatim}
       TRANsient,EXPLicit,gamma
\end{verbatim}
The gamma parameter should
be greater or equal to 0.5, the default is $\gamma = 0.5$.
A solution using the explicit option uses the command sequence:
\begin{verbatim}
       TRANsient,EXPLicit
       INITial (DISPlacements and/or RATEs)
       FORM,ACCElerations (initial acceleration)
       LOOP,time,nstep
         TIME
         FORM
         EXPLicit
         ... Outputs
       NEXT
\end{verbatim}
}
\end{enumerate}

{\sl FEAP} permits the type of transient problem
to be changed during the solution phase.  Thus, it is possible to
compute a configuration using a quasi-static option and then change
to a solution mode which includes the effects of rate terms (e.g.,
inertial effects).

\section{Transient Solution of Linear Problems}
\label{tranlin}

The solution of second order linear equations by the finite element method
leads to the set of equations
\begin{equation}
\label{eqm1}
\B{M} \, \ddot{\B{u}}(t) ~+~ \B{C} \,\dot{\B{u}}(t) ~+~ \B{K} \,\B{u}(t)
~=~ \B{F}(t)
\end{equation}
In structural dynamics,
the matrices $\B{M}$, $\B{C}$, and $\B{K}$
denote {\it mass}, {\it damping}, and {\it stiffness}, respectively.  The
vector $\B{F}$ is a force vector.
For the case where $\B{M}$, $\B{C}$, and $\B{K}$ are constant {\it symmetric}
matrices a
solution to Eq. \ref{eqm1} may be constructed
by partitioning the solution into the parts
\begin{equation}
\label{eqm2}
\B{u} ~=~ \left[
\begin{matrix} \B{u}_u \\ \B{u}_s \\ \end{matrix}
\right]
\end{equation}
where $( \cdot )_u$ denotes an unknown part and $( \cdot )_s$ a specified
part.
With this division, the equations are then written in the
form:
\begin{equation}
\label{eqm3}
\left[
\begin{matrix}
\B{M}_{uu} & \B{M}_{us} \\
\B{M}_{su} & \B{M}_{ss} \\
\end{matrix}
\right] \, \left[
\begin{matrix}
\ddot{\B{u}}_u \\ \ddot{\B{u}}_s \\
\end{matrix}
\right] + \left[
\begin{matrix}
\B{C}_{uu} & \B{C}_{us} \\
\B{C}_{su} & \B{C}_{ss} \\
\end{matrix}
\right] \, \left[
\begin{matrix}
\dot{\B{u}}_u \\ \dot{\B{u}}_s \\
\end{matrix}
\right] + \left[
\begin{matrix}
\B{K}_{uu} & \B{K}_{us} \\
\B{K}_{su} & \B{K}_{ss} \\
\end{matrix}
\right] \, \left[
\begin{matrix}
\B{u}_u \\ \B{u}_s \\
\end{matrix}
\right] ~=~ \left[
\begin{matrix}
\B{F}_u \\ \B{F}_s \\
\end{matrix}
\right]
\end{equation}

A solution is then constructed by first solving the first row of these
equations. The value of the reactions (i.e., $\B{F}_s$) associated
with the known part of the solution $\B{u}_s$ may be computed later if it is
needed.  The solution of the first row of these equations may be
constructed by several approaches.  The equations may be integrated in
time directly using a numerical step-by-step procedure (e.g., the Newmark
method); the solution may be constructed using normal modes and if necessary
specified multiple support conditions added; the equations may be solved in
the frequency domain.  In the next sections we discuss a solution using
modal methods and in a subsequent section a solution for the a response
due to periodic excitations is presented.

\subsection{Normal mode solution}
\label{normal}

The normal modes are obtained
by assuming the vector $\B{u}_s$ is zero and setting
\begin{equation}
\label{eqm4}
\B{u}_u = \Bf{\phi}_j \,
\exp ( i \, \omega_j \, t )
\end{equation}
where $\Bf{\Phi}_j$ is a {\it mode shape} and $\omega_j$ is its associated
{\it natural frequency}.
Differentiating with respect to time leads to the problem
\begin{equation}
\label{eqm6}
\left[
- \, \omega_j^2 \, \B{M}_{uu} ~+~ i \, \omega_j \, \B{C}_{uu} ~+~ \B{K}_{uu}
\right] \,
\Bf{\phi}_j \, \exp ( i \, \omega_j \, t ) ~=~ \ \B{F}_u
\end{equation}
in which $i = \sqrt{-1}$.
The normal modes of free vibration then may be obtained by setting the force
vector $\B{F}_u$ to zero and, for the present, ignoring the damping matrix
$\B{C}_{uu}$.  For this case the problem reduces to:
\begin{equation}
\label{eqm7}
\left[
- \, \omega_j^2 \, \B{M}_{uu} ~+~ \B{K}_{uu} \right] \,
\Bf{\phi}_j ~=~ \ \B{0}
\end{equation}
which may be solved as the
general linear eigen problem
\begin{equation}
\label{eqm8}
\left[
\Bf{\Phi} \right]^T \, \left[ \B{K}_{uu} \right] \, \left[ \Bf{\Phi}
\right] ~=~ \left[
\Bf{\Phi} \right]^T \, \left[ \B{M}_{uu} \right] \, \left[ \Bf{\Phi}
\right] \, \left[ \Bf{\Lambda} \right]
\end{equation}
where
\begin{equation}
\label{eqm9}
\left[
\Bf{\Phi} \right]  ~=~
\left[ \begin{matrix}
\Bf{\phi}_1 & \Bf{\phi}_2 & \cdots & \Bf{\phi}_n
\end{matrix} \right]
\end{equation}
is the set of {\it normal modes} and
\begin{equation}
\label{eqm10}
\left[
\Bf{\Lambda} \right]  ~=~
\left[ \begin{matrix}
\omega_1^2 & 0 & \cdots & 0 \\
0 & \omega_2^2 & \cdots & 0 \\
0 & 0 & \ddots & 0 \\
0 & \cdots & 0 & \omega_n^2 \\
\end{matrix} \right]
\end{equation}
is a diagonal matrix of the {\it natural frequencies squared}.

The solution for the normal modes are normalized so that
\begin{equation}
\label{eqm11}
\Bf{\Phi}^T \B{M}_{uu} \Bf{\Phi} ~=~ \B{I}
\end{equation}
and
\begin{equation}
\label{eqm12}
\Bf{\Phi}^T \B{K}_{uu} \Bf{\Phi} ~=~ \Bf{\Lambda}
\end{equation}

In {\sl FEAP} the solution for part (and for small problems all)
of the normal modes may be obtained using
a {\it subspace iteration} method and the solution commands:
\begin{verbatim}
       MASS
       TANGent
       SUBSpace,,nf
\end{verbatim}
where {\tt nf} is the number of modes to compute.  Additional parameters
may be given to use a lumped (diagonal) mass, to specify a shift
on the tangent matrix, and/or to improve the
convergence properties of the subspace method (See Appendix B for specifying
additional options).

For example, if the modes for an unsupported structure are desired, the
tangent matrix is singular and the subspace method will fail to converge
or an error may result during the construction of the factors of $\B{K}$ matrix.
In this case a {\it shift} may be used where the frequencies squared are
given as
\begin{equation}
\label{eqm13}
\omega_j^2 ~=~ \tilde{\omega}_j^2 \, + \, \chi
\end{equation}
Now the general linear eigen problem is given by:
\begin{equation}
\label{eqm14}
\left[
\Bf{\Phi} \right]^T \, \left( \left[ \B{K}_{uu} \right] \, \, - \, \chi
\, \left[ \B{M}_{uu} \right) \right] \, \left[ \Bf{\Phi}
\right] ~=~
\left[ \Bf{\Phi} \right]^T \, \left[ \B{M}_{uu} \right] \,
\left[ \Bf{\Phi} \right] \, \left[ \tilde{\Bf{\Lambda}} \right]
\end{equation}
which may be solved using the command language algorithm
\begin{verbatim}
       MASS
       TANGent,,,chi
       SUBSpace,,nf
\end{verbatim}
in which {\tt chi} denotes the value of $\chi$ in Eq. \ref{eqm13}.

\subsection{Damping effects}
\label{damping}

The effects of damping may be included in the modal formulation
and still retain real normal modes by assuming a damping matrix in the form
\begin{equation}
\label{eqm15}
\B{C} ~=~ a_0 \, \B{M} ~+~ a_1 \, \B{K}
\end{equation}
This defines a form called {\it Rayleigh Damping}.
With this form the Damping matrix has the property:
\begin{eqnarray}
\label{eqm16}
\Bf{\Phi}^T \B{C}_{uu} \Bf{\Phi} &=& a_0 \, \B{I} ~+~ a_1 \, \Bf{\Lambda}
\nonumber \\
&=& 2 \, \Bf{\zeta} \, \Bf{\Lambda}^\frac{1}{2}
\end{eqnarray}
where $\Bf{\zeta}$ is a diagonal matrix of damping ratios.
The damping ratio may be related to the parameters $a_0$ and $a_1$ as
\begin{equation}
\label{eqm16a}
2 \, \Bf{\zeta} ~=~ a_0 \, \Bf{\Lambda}^{\frac{1}{2}}
~+~ a_1 \, \Bf{\Lambda}^{- \, \frac{1}{2}}
\end{equation}
Values for the parameters $a_0$ and $a_1$ may be computed from two values
of frequencies where a specified damping ratio $\zeta$ is desired.  If
the two frequencies are denoted by $\omega_i$ and $\omega_j$ the parameters
are given by
\begin{equation}
\label{eqm16b}
a_0 ~=~ 2 \, \zeta \frac{\omega_i \, \omega_j}{\omega_i + \omega_j}
\end{equation}
and
\begin{equation}
\label{eqm16c}
a_1 ~=~ \frac{2 \, \zeta }{\omega_i + \omega_j}
\end{equation}
Other values may also be selected.
For further information consult "Dynamics of Structures", by A.K. Chopra,
\cite{chopra}.
The data input to {\sl FEAP} is given by the command
\begin{verbatim}
       RAYLeigh,,a-0,a-1
\end{verbatim}

Rayleigh damping may also be included in transient problems solved 
by time integration methods.  In this case the damping matrix may be specified
independently for each material as global or
material parameters (see Section \ref{raydamp}).

\subsection{Solution of transient problems}
\label{modsol}

Using normal modes the solution of the transient problem is constructed
by substituting the solution
\begin{equation}
\label{eqt1}
\B{u}_u(t) ~=~ \Bf{\Phi} \, \B{v}(t)
\end{equation}
into the first row of Eq. \ref{eqm3} and premultiplying by $\Bf{\Phi}^T$.
Using the orthogonality properties from Eqs. \ref{eqm11}, \ref{eqm12},
and \ref{eqm16} the result is given by:
\begin{equation}
\label{eqt2}
\ddot{\B{v}} ~+~ 2 \, \Bf{\zeta} \, \Bf{\Lambda}^\frac{1}{2} \, \dot{\B{v}}
~+~ \Bf{\Lambda} \, \B{v} ~=~ \Bf{\Phi}^T \B{F}_u(t) ~=~ \B{G}(t)
\end{equation}
which is a set of uncoupled second order differential equations.  An individual
equation is given by:
\begin{equation}
\label{eqt3}
\ddot{v}_j ~+~ 2 \, \zeta_j \, \omega_j \, \dot{v}_j ~+~  \omega_j^2 \, v_j
 ~=~ g_j(t)
\end{equation}
Each of the equations may be integrated numerically or, if the loading
is assumed in some functional form, exactly.  For example, assuming piecewise
linear variation in a time step {\sl FEAP} performs an exact integral provided
support solutions all have zero values.  The
numerical solution using modal methods is given by the command language
algorithm
\begin{verbatim}
       DT,,dt-value
       LOOP,time,n-steps
         TIME
         MODAl
          .....  ! outputs/plots
       NEXT,time
\end{verbatim}

\subsection{Specified multiple support excitation}
\label{multis}

In the previous sections the modal response was constructed by assuming
all specified support locations had zero values.
A solution to Eq. \ref{eqm3} which includes the effects
of non-zero support excitations may be constructed by expressing the solution
in the form:
\begin{equation}
\label{eqs1}
\left[ \begin{matrix} \B{u}_u \\ \B{u}_s \end{matrix} \right] ~=~
\left[ \begin{matrix} \Bf{\Phi} \\ \B{0} \end{matrix} \right] \,
\left[ \B{v} \right] ~+~
\left[ \begin{matrix} \Bf{\Psi} \\ \B{I} \end{matrix} \right] \,
\left[ \B{w} \right]
\end{equation}
where the arrays $\Bf{\Phi}$ and $\Bf{\Psi}$ represent the normal modes of
vibration and {\it static} modes to satisfy non-zero specified boundary
conditions, respectively.  For the static modes we solve the problem
\begin{equation}
\label{eqs2}
\B{K}_{uu} \, \Bf{\Psi} ~+~ \B{K}_{us} \, \B{I} ~=~ \B{0}
\end{equation}
The solution for the normal modes is obtained from Eq. \ref{eqm8}.

Once these modes are known, the first row of Eq. \ref{eqm3} may be premultiplied
by $\Bf{\Phi}^T$ to give
\begin{eqnarray}
\label{eqs3}
&& \Bf{\Phi}^T \B{M}_{uu} \Bf{\Phi} \, \ddot{\B{v}} ~+~
\Bf{\Phi}^T \B{M}_{uu} \Bf{\Phi} \, \dot{\B{v}} ~+~
\Bf{\Phi}^T \B{K}_{uu} \Bf{\Phi} \, \B{v} \nonumber \\
&& \qquad= \B{G} ~-~
\Bf{\Phi}^T \left[ \B{M}_{uu} \Bf{\Psi}
+ \B{M}_{us} \right] \, \ddot{\B{w}} ~-~
\Bf{\Phi}^T \left[ \B{C}_{uu} \Bf{\Psi}
+ \B{C}_{us} \right] \, \ddot{\B{w}}
\end{eqnarray}
Invoking the orthogonality conditions Eqs. \ref{eqm11}, \ref{eqm12},
and \ref{eqm16} leads to the set of decoupled
equations
\begin{equation}
\label{eqs4}
\ddot{\B{v}} ~+~ 2 \, \Bf{\zeta} \, \Bf{\Lambda}^\frac{1}{2} \, \dot{\B{v}}
~+~ \Bf{\Lambda} \, \B{v} ~=~ \Bf{\Phi}^T \B{F}_u(t) ~=~ \B{G}(t) ~-~
\B{A}_1 \, \ddot{\B{w}} ~-~ \B{A}_2 \, \dot{\B{w}}
\end{equation}
where
\begin{equation}
\label{eqs5}
\B{A}_1  ~=~
\Bf{\Phi}^T \left[ \B{M}_{uu} \Bf{\Psi} \, + \, \B{M}_{us} \right]
\end{equation}
and
\begin{equation}
\label{eqs6}
\B{A}_2  ~=~
\Bf{\Phi}^T \left[ \B{C}_{uu} \Bf{\Psi} \, + \, \B{C}_{us} \right]
\end{equation}
For Rayleigh damping only one matrix is required since
\begin{equation}
\label{eqs6a}
\B{A}_2  ~=~ a_0 \, \B{A}_2
\end{equation}

In {\sl FEAP} these equations are integrated using the energy momentum
method in which the discrete time values are given as
\begin{equation}
\label{eqs7}
\B{v}_n  ~ \approx ~  \B{v}(t_n)
\end{equation}
and the solution advanced using the equations:
\begin{equation}
\label{eqs8}
\B{v}_{n+1} ~=~ \B{v}_n + \Delta t \, \dot{\B{v}}_n + \frac{1}{2} \Delta t^2
\, \ddot{\B{v}}_{n+\frac{1}{2}}
\end{equation}
and
\begin{equation}
\label{eqs9}
\dot{\B{v}}_{n+1} ~=~ \dot{\B{v}}_n + \Delta t \, \ddot{\B{v}}_{n+\frac{1}{2}}
\end{equation}
Values at the mid time step $t_{n+\frac{1}{2}}$ are computed as:
\begin{equation}
\label{eqs10}
\B{v}_{n+\frac{1}{2}} ~=~
\frac{1}{2} \left( \B{v}_n + \B{v}_{n+1} \right)
\end{equation}
\begin{equation}
\label{eqs11}
\dot{\B{v}}_{n+\frac{1}{2}} ~=~
\frac{1}{2} \left( \dot{\B{v}}_n + \dot{\B{v}}_{n+1} \right)
\end{equation}
and
\begin{equation}
\label{eqs12}
\ddot{\B{v}}_{n+\frac{1}{2}} ~=~ \frac{1}{\Delta t} \, \left(
\dot{\B{v}}_{n+1} ~-~ \dot{\B{v}}_n \right)
\end{equation}
Finally, the equations of motion are written at the mid step
giving:
\begin{equation}
\label{eqs13}
\ddot{\B{v}}_{n+\frac{1}{2}} ~+~ 2 \, \Bf{\zeta} \,
\Bf{\Lambda}^{\frac{1}{2}} \, \dot{\B{v}}_{n+\frac{1}{2}}
~+~ \Bf{\Lambda} \, \B{v}_{n+\frac{1}{2}} ~=~ \B{G}_{n+\frac{1}{2}} ~-~
\B{A}_1 \, \ddot{\B{w}}_{n+\frac{1}{2}}~-~
\B{A}_2 \, \dot{\B{w}}_{n+\frac{1}{2}}
\end{equation}

The values of the time derivatives for $\B{w}_{n+\frac{1}{2}}$ are determined
from the inputs of $\B{w}_n$ using Eqs. \ref{eqs8} to \ref{eqs12}.

The specification of the data for a problem which is to be subjected to
multiple support excitations requires the following data and solution steps:

\begin{enumerate}
\item
During mesh input, specify the base patterns and their associated
proportional loading factors.  Base patterns are given by the mesh {\tt BASE}
command with data for each node given as follows
\begin{verbatim}
       BASE
        node1,gen1,(base-set1(i),i=1,ndf)
        node2,gen2,(base-set2(i),i=1,ndf)
          etc. for additional nodes
                                        ! Blank terminator record
\end{verbatim}
In the above non-zero {\tt base-setj(i)} values define the individual base
set numbers.  A zero value indicates the degree-of-freedom
is assigned to the unknown part of a solution vector.  Base sets
should be numbered from one (1) to a maximum number.

Recall that it
is also necessary to assign each node with a non-zero base set to a
specific proportional load set using the {\tt FPRo}portional mesh command.
For example, this may be done using the data set:
\begin{verbatim}
       FPROportional
        node1,gen1,(prop-set1(i),i=1,ndf)
        node2,gen2,(prop-set2(i),i=1,ndf)
          etc. for additional nodes
                                        ! Blank terminator record
\end{verbatim}
Warning: Degree-of-freedoms with the same base set number {\it must} have
the same proportional load set number.
\item
During the solution process it is necessary to compute the normal modes and
their associated natural frequencies using the command statements:
\begin{verbatim}
       MASS
       TANGent
       SUBSpace,,nf
\end{verbatim}
Subsequently it is necessary to issue the commands:
\begin{verbatim}
       BASE
       TRANSient,CONServing
\end{verbatim}
followed by the modal solution commands:
\begin{verbatim}
       DT,,delta-t
       LOOP,time,n-steps
         TIME
         MODAl
          .... output statements
       NEXT,time
       TRANSient,CONServing
\end{verbatim}
\end{enumerate}

The solution steps indicated above are order dependent.
Modes must exist in order to
perform the {\tt BASE} step.  The base step computes the base modes $\Bf{\Psi}$
and constructs the array $\B{A}_1$ needed to set up the multiple support
excitation steps given above.  It also requires a factored stiffness matrix
constructed by the {\tt TANG}ent command.  Since base supports are provided,
no {\it shift} should be included on the tangent command.

\section{Periodic inputs on linear equations}
\label{periodic}

The solution of second order linear equations by the finite element method
leads to the set of equations given by Eq. \ref{eqm1}.
If the applied loading is periodic the force may be expressed in the form
\begin{equation}
\label{eq2}
\B{F}(t) ~=~ \hat{\B{F}}(\omega) \exp (i \, \omega \, t)
\end{equation}
where $i = \sqrt{-1}$ and $\omega$ is a specified periodic input frequency.
The notation $\hat{(\cdot)}$ denotes a complex quantity.  Thus,
the intensity of the force is assumed to be a complex vector.  At present,
the implementation in {\sl FEAP} restricts the force specified during
input to be real. Accordingly,
\begin{eqnarray}
\label{eq3}
\B{F}_r &=& \Re ( \hat{\B{F}} ) \\
\B{F}_i &=& \Im ( \hat{\B{F}} ) = \B{0}
\end{eqnarray}
The real part of the force may be input using the mesh commands {\tt FORC}e,
{\tt CFOR}ce,  {\tt EFOR}ce, and/or {\tt CSUR}face.
For the case where $\B{M}$, $\B{C}$, and $\B{K}$ are constant matrices a
solution to Eq. \ref{eqm1} may be constructed by assuming the solution
in the form:
\begin{equation}
\label{eq4}
\B{u}(t) ~=~ \hat{\B{u}}(\omega) \exp (i \, \omega \, t)
\end{equation}
which may be differentiated to define the time derivatives of $\B{u}$.
This leads to the equation:
\begin{equation}
\label{eq5}
\left[ - \, \omega^2 \, \B{M} ~+~ i \, \omega \, \B{C} ~+~ \B{K} \right] \,
\hat{\B{u}}(\omega) ~=~ \ \hat{\B{F}} (\omega)
\end{equation}
which may be solved for each specified frequency and load to give a
solution for the $\hat{\B{u}}(\omega)$.

There are some cases where part of the displacement vector $\hat{\B{u}}$ is
known and non-zero.  For this case we can partition Eq. \ref{eq5} into
parts.  Let the coefficient matrix be given by
\begin{equation}
\label{eq6}
\hat{\B{A}} ~=~ - \, \omega^2 \, \B{M} ~+~ i \, \omega \, \B{C} ~+~ \B{K}
\end{equation}
and partition the solution into
\begin{equation}
\label{eq7}
\hat{\B{u}} ~=~ \left[
\begin{matrix} \hat{\B{u}}_u \\ \hat{\B{u}}_s \\ \end{matrix}
\right]
\end{equation}
where $( \cdot )_u$ denotes an unknown part and $( \cdot )_s$ a specified
part.
With this division, the equations to be solved may be written in the
form:
\begin{equation}
\label{eq8}
\left[
\begin{matrix}
\hat{\B{A}}_{uu} & \hat{\B{A}}_{us} \\
\hat{\B{A}}_{su} & \hat{\B{A}}_{ss} \\
\end{matrix}
\right] \, \left[
\begin{matrix}
\hat{\B{u}}_u \\ \hat{\B{u}}_s \\
\end{matrix}
\right] ~=~ \left[
\begin{matrix}
\hat{\B{F}}_u \\ \hat{\B{F}}_s \\
\end{matrix}
\right]
\end{equation}
A solution may be achieved by first solving the equation set
\begin{equation}
\label{eq9}
\hat{\B{A}}_{uu} \,  \hat{\B{u}}_u ~=~
\hat{\B{F}}_u \, - \,  \hat{\B{A}}_{us} \, \hat{\B{u}}_s
\end{equation}
for the unknown part of the solution vector.  Again, during mesh input,
only a real part may currently be specified for $\hat{\B{u}}_s$.  This
may be done using the mesh command options {\tt DISP}lace\-ment,
{\tt CDIS}pl, {\tt EDIS}pl, and/or {\tt CSUR}\-face.
Once the unknown part is
computed the reaction forces may be determined from the remaining part as:
\begin{equation}
\label{eq10}
\hat{\B{F}}_u ~=~
\left[
\begin{matrix}
\hat{\B{A}}_{uu} & \hat{\B{A}}_{us} \\
\end{matrix}
\right] \, \left[
\begin{matrix}
\hat{\B{u}}_u \\ \hat{\B{u}}_s \\
\end{matrix}
\right]
\end{equation}

In {\sl FEAP} the above is implemented by first declaring the problem to
be complex.  This is accomplished by starting a problem as
\begin{verbatim}
       *COMPLEX
       FEAP * * Title information
           etc.
\end{verbatim}
The constant real arrays ${\B{M}}$, ${\B{C}}$, ${\B{K}}$ are then
formed and stored
in a sparse matrix format in which only the non-zero terms are retained.
Then for each specified frequency $\omega$ the array $\hat{\B{A}}_{uu}$
is formed and stored in an in-core profile form.  This matrix is complex and at present only an in-core
profile solution scheme is available in {\sl FEAP}.  The solution
is then performed and the unknown part combined with the known part to
form the total solution vector $\hat{\B{u}}$.  The command:
\begin{verbatim}
       CXSOlve,,omega
\end{verbatim}
is used to perform this step.  The first issue of the command will form
the arrays $\B{M}$, $\B{C}$, and $\B{K}$ which are then used in all subsequent
specifications of the frequency $\omega$.

After a solution is available the usual {\sl FEAP} commands may be used
to output or plot the solution.  For example, the command
\begin{verbatim}
       DISPlacement,,k1,k2,k3
\end{verbatim}
outputs the real part of the displacement for nodes {\tt k1} to {\tt k2} at
increments of {\tt k3}.  Similarly,
\begin{verbatim}
       DISPlacement,IMAGinary,k1,k2,k3
\end{verbatim}
outputs the imaginary part of the displacement for nodes {\tt k1} to {\tt k2} at
increments of {\tt k3}.  Finally,
\begin{verbatim}
       DISPlacement,CMPL,k1,k2,k3
\end{verbatim}
outputs the real and imaginary parts of the displacement for the nodes.

The plot commands {\tt PLOT,REAL} and {\tt PLOT,IMAG} set the display
contours to the real and imaginary parts, respectively.  The usual plot
commands (e.g., {\tt PLOT,CONT,i}) then give the desired solution part. 
