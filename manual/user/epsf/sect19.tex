\chapter[Contact]{CONTACT PROBLEMS}
\label{contact}

The solution of problems in which the boundaries of one part of the
system may interact with the boundaries of another part are called
{\it contact} problems.  {\sl FEAP} provides options to solve problems
in two or three dimensions in which conditions are imposed to prevent
the penetration of one body into another.  In order to specify a
contact analysis it is necessary to define the surfaces of the bodies
which are to be considered during a contact analysis.  In addition a
user must describe which of these surfaces are to be considered as 
possible contacting pairs.  Finally, the modeling of the behavior of
one surface interacting and/or sliding against another must be specified.

In the current release of {\sl FEAP} the control of the penetration
between bodies is implemented as a penalty or an augmented Lagrangian
method.  Provisions are included to permit adding a Lagrange multiplier
option at a later date.  For general problems the penetration is monitored
from a {\it slave} point and a {\it master} point.  The slave point is
a node and the master point is interpolated from a simple facet form
associated with the boundary of an element.  This is often called a point
to surface strategy.

A contact problem is described by inserting the above information
into the input data file after the mesh {\tt END} record and before
the first solution commands.  Contact data may be given either before
or after mesh manipulation commands.  The command sequence:
\begin{verbatim}
       CONTact options
         ......
       END contact
\end{verbatim}
defines the extent of contact input records.

\section{Surface Definitions}

After the {\tt CONT}act command it is necessary to define at least two
surfaces which will be considered during the contact.  A surface command
is defined by the form
\begin{verbatim}
       SURFace number
         surface_type
           surface data sets
                          ! termination record
\end{verbatim} 
where {\tt number} is a numerical identifier for the surface which will
be used as part of the {\tt PAIR} data defined below.  The
{\tt surface-type} defines the shape of a contact facet and must be
selected from: {\tt POIN}t, {\tt LINE}, {\tt TRIA}ngle, or
{\tt QUAD}rilateral.  The {\tt POIN}t and {\tt LINE} options are used
for two dimensional problems.  The {\tt POIN}t, {\tt TRIA}ngle, and
{\tt QUAD}rilaterall options are used for three dimensional problems.
The surface data sets may be defined using a {\tt FACE}t, {\tt BLOC}k,
{\tt BLEN}d, or {\tt REGI}on option.  Using the {\tt FACE}t option
requires specification of the node numbers for each element boundary segment.
Typical data for a two dimensional problem with 2-node element boundary
segments consists of:
\begin{verbatim}
       SURFace number
         LINE
           FACEts
             M MG Mnode_1 Mnode_2
             N NG Nnode_1 Nnode_2
                ......
                          ! termination record
\end{verbatim} 
where generation occurs from facets {\tt M} to {\tt N} using increments
of {\tt MG} to each {\tt Mnode-i}.  This is performed in a manner similar
to the element generations using the {\tt ELEM}ent mesh command.
The facet inputs must describe a single surface entity, that is,`there can be
no gaps between any facets.  The facets do not need to be in order but
must be complete for a single surface.

A surface may be {\it open}, with two distinct end points, or {\it closed}
as for a wheel.

The {\tt BLOCk} option is analogous to the way surface loads are generated
using the {\tt CSUR}face mesh command.  The data for 2-node boundary segments
is given as
\begin{verbatim}
       SURFace number
         LINE
           BLOCk SEGMent
             1 x_1 y_1
             2 x_2 y_2
             3 x_3 y_3
                          ! termination record
\end{verbatim} 
If only two master nodes are used to describe a block the segment is
a straight line, whereas three points describe a parabola in the natural
coordinate space.  The {\tt BLOCk SEGMent} command may be preceded by
a {\tt BLOCk GAP value} to increase or decrease a search tolerance and/or by
a {\tt BLOCk POLAr} command to perform the search in polar (or cylindrical)
coordinates.

The {\tt BLENd} option is analogous to the way sides are generated for
blending function mesh generation.  At present only two dimensional
surfaces may be defined by the contact blend option.  The data input is
\begin{verbatim}
       SURFace number
         LINE
           BLENd SEGMent
             type sn_1 sn_2 sn_3 ....
                          ! termination record
\end{verbatim} 
where {\tt type} is selected from {\tt CART}esian,
{\tt POLA}r, or {\tt SEGM}ent.

\section{Contact Material Models}

The behavior of one surface interacting with another may be modeled in
different ways.
The current release includes very simple model in which the surface is
considered as regular (no roughness or micro-mechanical details are to
be specified) but may have frictional resistance.  For frictionless
contact no material definition is required - {\sl FEAP} will assign default
conditions.  If friction is present it is necessary to define a Coulomb
frictional behavior.  This is included as the data set
\begin{verbatim}
       MATErial number
         STANDard
           FRICtion COULomb value
                          ! termination record
\end{verbatim} 
where {\tt value} is a constant coefficient of friction.

\section{Pair Definition}

The interaction between two surfaces is controlled by the {\tt PAIR} command.
This command describes which two surfaces are to be considered, the type of
contact solution, the solution method, and solution tolerances.  A typical
data set for solution of problems is given by
\begin{verbatim}
       PAIR number
         NTOS slave master
         SOLM PENAlty k_n k_t
         TOLE values  t_1 t_2 t_3
                          ! termination record
\end{verbatim} 
The parameter {\tt number} is an identifier numeral for the pair.  The
basic solution strategy in two dimensions is node-to-segment ({\tt NTOS}) and
requires the specification of a {\tt slave} surface identifier numeral and
a {\tt master} surface identifier numeral.  The solution method may
be given as {\tt SOLM PENAlty} with {\tt k-n} and {\tt k-t} the {\it penalty}
parameters used for normal penetration control and tangential stick
control, respectively.  alternatively, the command may be given as
\texttt{SOLM LAGM} to impose normal gap constraints using a Lagrange multiplier
method.  The parameter \texttt{k-n} may also be used to provide some
\textit{stiffness} on the surface.  This stiffness is effective only during
iteration process -- final gap is imposed exactly using the Lagrange
multiplier approach.  The {\tt TOLE}rance option defines the values
for solution tolerances: {\tt t-1} is a tolerance for defining initial
penetration; {\tt t-2} is a tolerance for considering a contact open; and
{\tt t-3} is an out of segment tolerance.
Generally, some value for the out of segment tolerance is required to maintain
contact when a slave node moves from one master segment to the next.  Other
options exist to define augmentation forms and material models.
