\section{Nodal Boundary Condition Inputs}
\label{bc}

The basic {\sl FEAP} boundary condition quantities
are values for non-zero nodal {\it forces} and nodal {\it displacements}.
For problems in solid mechanics these terms have physical meaning; however,
for general classes of problems forces and displacements are interpretted
in a {\it generalized} sense - e.g., as flux and dependent variable pairs.
Non-zero values for forces and displacements may both be input
at each node.  It is not necessary to input conditions for
any node where all the components are zero.  The actual condition
to be imposed (i.e., force or displacement)
is determined by the active values of the {\it boundary
restraint conditions}.  A non-zero value of a boundary restraint
condition for a degree-of-freedom implies that the value of the
specified nodal displacement is to be imposed; whereas, a zero
value implies that the value of the specified nodal force is to be
applied.  Generally,
these quantities are specified by components associated with the
directions in the global cartesian coordinates describing a mesh.
It is possible, however, to specify components
which are associated with directions different than the global coordinate ones.
At present, the only option is a set of coordinates which are described
by a rotation angle about the $x_3$ axis with respect to the $x_1$ axis.
The input of boundary condition quantities associated with nodes may
be specified based on: Node numbers; Nodal coordinate values; or
Edge coordinate values.

\subsection{Basic input form.}

The basic options to input the nodal quantities associated with boundary
conditions is shown in Table 11.1.  The use of a basic form (i.e., {\tt
BOUNdary, FORCe, DISPlacement, ANGLe}) implies a specification using
a node number.  The opther options do not require node numbers and are
preferred when possible.

\begin{table}[ht!]
\begin{center}
\begin{tabular}{| l | l | l | l | l |} \hline
Type &  Boundary & Forces & Displacements & Angle \\
\hline Nodal & {\tt BOUNdary} & {\tt FORCe} & {\tt DISPlacement} & {\tt ANGLe}\\
Edge    & {\tt EBOUndary} & {\tt EFORce} & {\tt EDISplacement} & {\tt EANGle} \\
Coordinate & {\tt CBOUndary} & {\tt CFORce} & {\tt CDISplacement} & {\tt CANGle} \\
\hline
\end{tabular}
\end{center}
\caption{Nodal Boundary Condition Quantity Inputs}
\label{tab111}
\end{table}
An example of the use of the nodal option for input of a force in the
2-direction on node 19 is given by:
\begin{verbatim}
       FORCe
         19  0    0.0     10.0
                             ! Termination record
\end{verbatim}
The input records for basic {\tt FORC}e, {\tt DISP}\-lacement, {\tt BOUN}d\-ary
condition and {\tt ANGL}e
commands are similar to those for {\tt COOR}\-inates
with the node and generation increment in the first two fields and the list
of values for each degree-of-freedom in the remaining field.  The values of
all arrays are set to zero at the start of each problem, hence only non-zero
values need be specified for forces, displacements, boundary conditions and
angles.

Similarly, the specification of a non-zero displacement at a node may be given
using the command
\begin{verbatim}
       DISPlacement
	     19  0  0.0  -0.1
\end{verbatim}
The value of a force or displacement will be selected based on the {\it
boundary restraint code} value.  Non-zero boundary restraint codes imply
a specified displacement and zero values a specified load.  The boundary
restraint codes may be set using the command
\begin{verbatim}
       BOUNdary codes
	     19  0  0  1
\end{verbatim}
which states the first degree-of-freedom is a specified force (zero by
default) and the second a specified displacement (again zero by default).
Thus, if both of the above force and displacement commands are included
only the non-zero displacement will be used.  During execution it is possible
to change the boundary restraint codes to then use the non-zero force.

To use the basic input option
it is a users responsibility to determine the
correct number for each node - often the graphics capability of {\sl FEAP}
can assist in determining the correct node numbers; however, for a very large
number of forces this is a tedious method. Accordingly, there are two other
options available to input the nodal values.

\subsection{Edge input form.}

The second option available to specify the
nodal quantities is based on coordinates and is
used to apply a common value to all nodes located
at some constant coordinate location called the {\it edge} value.
The options {\tt EBOUndary, EFORce, EDISplacement, EANGle}
are used for this purpose.  For example, if it is
required to impose a zero displacement for the first degree of freedom
of all nodes located at $y = 0.5$.
The edge boundary conditions may set using
\begin{verbatim}
       EBOUndary
         2    0.5  1   0
                             ! Termination record
\end{verbatim}
In the above the 2 indicates the second coordinate direction (i.e., $x_2$ or $y$
for cartesian coordinates) and $0.5$ is the value of the $x_2$ or $y$ coordinate
to be used to find the nodes.  The last two fields are the boundary condition
pattern to apply to all the nodes located.  That is, above we are indicating
the first degree-of-freedom is to have specified displacements and the
second is to have specified forces.
{\sl FEAP} locates all nodes which are within a small tolerance of
the specified coordinate {\it after the mesh input is completed}.

By default the edge options will be appended to
any previously defined data at a
node by the pattern specified.  If it is desired to {\it replace} the
conditions edge options are specified as:
\begin{verbatim}
       EBOUndary,SET
         1    0.5  1   0
         2    0.5  0   1
                             ! Termination record
\end{verbatim}
By the default where no option is set or
with the inclusion of the {\tt ADD} parameter the boundary  restraint code
at a node located at (0.5, 0.5) will be fully restrained (i.e., have both
directions with a unit (1) restrained value).  With the {\tt SET} option
as shown above
the node would have only its second degree-of-freedom restrained.

\subsection{Coordinate input form.}

Using the options {\tt
CBOUndary, CFORce, CDISplacement, CANGle} indicates that the quantities
are to be input based on the coordinates of a node.
An example to specify a 10 unit force in the $y$-direction for a two-dimensional
problem node located at $x = 4.0$ and $y = 5.0$ is given by:
\begin{verbatim}
       CFORCe
         NODE 4.0  5.0    0.0     10.0
                             ! Termination record
\end{verbatim}
This method will place the force on the node nearest the specified point.
If two nodes have the same or equally close
coordinate only one will have the force applied.
While much easier, this method is still somewhat tedious if a large number of
forces need to be applied.  Options exist to generate the forces automatically
for some distributed loading types (e.g., see Section \ref{surf}).

Once again, coordinate generated data will replace previously generated values
unless the {\tt ADD} parameter is added.  Thus the final outcome of the above
{\tt CFOR}ce command would be to have a force value
for the first degree-of-freedom of 10.0.

\subsection{Hierarchy of input forms.}

The input of the nodal boundary data is performed by {\sl FEAP} in a specific
order.  Data input in the basic form is interpretted immediately after the
data records are read.  Values assigned by the basic input replace any
previously specified values - they are not accumulated.

Data input by the edge option is interpretted before any coordinate specified
data.  By default the data is added to any previously specified information;
however, if the data is specified in a {\tt Exxx,SET} option the information is
replaced.  Multiple edge sets may be input and are interpretted later in
the order they were encountered in the input file.  Thus, use of the sequence
of commands
\begin{verbatim}
       EBOUndary,SET
         1 10.0  1 0
                             ! Termination record
       EBOUndary,ADD (or blank)
         1  0.0  1 0
         2  0.0  0 1
                             ! Termination record
\end{verbatim}
defines two data sets.  The first will replace the boundary code definition
for any node which has $x_1$ equal to 10.0 by a restrained first dof and
an unrestrained second dof.  Subsequently, the second set will restrain
all the first dof at any node with $x_1$ equal to zero
and also restrain the second dof at any node with $x_2$ equal to zero. 
Thus, if there is a node with ($x_1$, $x_2$) of (0.0, 0.0) the node will
be fully restrained

After all edge data sets are processed the data defined by the coordinate
option is processed.  By default it is also interpretted in a {\tt SET}
mode unless the data set is defined by a {\tt Cxxx,ADD} command.

When using the coordinate or edge options it is recommended that the graphics
options in {\sl FEAP} be used to check that all desired quantities are located.
For the coordinate method other options are available to specify
forces, displacements, and boundary conditions.  These are described further
in Appendix A.

\subsection{Time dependent load functions}

Each nodal force or displacement may be multiplied by a time dependent,
{\it proportional} loading function.  By default the sum of all proportional
loads is used as the multiplying factor.  Each load function is defined
by the {\tt PROP}ortional command during a solution phase.  Each proportional
loading record is defined by a number.  Thus, the number for a proportional
load varies from one ({\tt 1}) to a maximum ({\tt NPLD}). Specific
proportional loading functions may be assigned to a nodal force or
displacement using the {\tt FPRO}p, {\tt EPRO}p, and/or {\tt CPRO}p commands. 
These commands are processed in a set mode in the same basic, edge, and
coordinate sequence defined above for the other nodal boundary data.
For example,
\begin{verbatim}
       FPROportional
         m mg pm_1 pm_2 ... pm_ndf
         n  0 pn_1 pn_2 ... pn_ndf
                             ! Termination record
\end{verbatim}
would generate a pattern of proportional loads between nodes {\tt m} and {\tt n}
at increments of {\tt mg}.  The patterns {\tt pm\_i pn\_i} should be identical
to produce predictable results. 
Each {\tt pm\_i} refers to a specific proportional loading function (see
section in command language chapter).  If a {\tt pm\_i} is zero the forced
quantity will be multiplied by the {\it sum} of all proportional loadings
active at a particular time instant.


As a second example, the command sequence
\begin{verbatim}
       EPROportional
         1 10.0  1 0 3
                             ! Termination record
\end{verbatim}
would assign the non-zero force or displacement quantities of
all nodes where $x_1$ is 10.0 to have their first dof
multiplied by proportional loading number 1 and the third dof by proportional
loading number 3.  Any second dof would be multiplied by the {\it sum}
of all defined proportional loading functions.  For this to work properly
it is necessary to have at least three proportional loading functions
defined during the solution phase.


Proportional loading functions may also be used to specify acceleration
effects on lumped masses.  The {\tt MPRO}p command is used to specify
the mass loading function numbers on nodes which have discrete masses
specified by the {\tt MASS} mesh command.  The {\tt MPRO}p command is
input as:
\begin{verbatim}
       MPROportional
         m mg  mp_1 mp_2 ... mp_ndf
         n ng  np_1 np_2 ... np_ndf
                             ! Termination record
\end{verbatim}
and generation can be performed in a manner similar to the {\tt FPRO}p
command.

In each momentum equation a discrete mass term associated with
an {\tt MPRO} command will be computed as:
\begin{equation}
\Bf{M}_{nn} \, \left( \ddot{\Bf{x}}_n - \Bf{g}(\Bf{x}_n) \right)
\end{equation} 
where $n$ is the node number and the components of $\Bf{g}$ are defined as
\begin{equation}
g_i(\Bf{x}_n) = f_i \, prop_k(t) ~~{\rm where} ~ k = np_i(n)  
\end{equation}
The factors $f_i$ are specified using the {\tt GROUnd} global command.
