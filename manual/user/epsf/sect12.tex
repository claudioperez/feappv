\section{Surface Loading}
\label{surf}

{\sl FEAP} uses the {\tt CSUR}face command to specify distributed tractions
and displacements on portions of two or three dimensional
surfaces defined by interpolation patches.
For two dimensional problems the command has the structure
\begin{verbatim}
       CSURface
         type, data
         LINEar
         1,X_1,Y_1,P_1
         2,X_2,Y_2,P_2
                             ! blank termination record
\end{verbatim}
or
\begin{verbatim}
       CSURface
         type, data
         QUADratic
         1,X_1,Y_1,P_1
         2,X_2,Y_2,P_2
         3,X_3,Y_3,P_3
                             ! blank termination record
\end{verbatim}
where {\tt type} is an optional data type selected from:
{\tt CART}esian; {\tt POLA}r; {\tt GAP};
{\tt NORM}al traction;
{\tt TANG}ential traction; or {\tt DISP}lacement pattern (default is normal
traction).
If the data type is {\tt DISP}lacement the parameter {\tt data} specifies
the coordinate direction for the specified values.
Multiple records of {\tt type} may exist before input of interpolation
patches and patterns.

The parameters {\tt LINE}ar or {\tt QUAD}ratic define the order of the
interpolation patch.  The values of $x_1$, $y_1$ and $x_2$,$y_2$ define
coordinate end points on the patch and,
for quadratic surfaces, $x_3$, $y_3$ define
the middle point coordinates for the patch.
The parameters $p_1$, $p_2$, and $p_3$ define the values of the traction
or the displacement at the corresponding coordinates on the patch.

{\sl FEAP} will search for all nodes which are closer to the interpolation
patch than {\tt GAP} (default is $10^{-3}$).  Using the element boundary
segments which have outward normals to the patch (by right hand coordinate
rule as shown for a two-dimensional problem in Figure \ref{figsurf})
will be located and the values interpolated to nodes.  For tractions
the equivalent nodal loads will be computed.  In two dimensions it is
not necessary for the interpolation patch to exactly match the element
boundary segments.

\begin{figure}
\unitlength=1mm
\special{em:linewidth 0.4pt}
\linethickness{0.4pt}
\begin{picture}(123.67,90.33)
\put(121.33,42.00){\line(-2,5){6.67}}
\put(114.67,58.33){\line(-5,6){9.67}}
\put(105.00,70.00){\line(-2,1){15.33}}
\put(89.67,77.67){\line(-6,1){19.67}}
\bezier{316}(122.00,42.00)(109.00,79.67)(70.33,82.33)
\put(70.67,80.67){\line(0,-1){16.00}}
\put(88.33,77.33){\line(-1,-5){3.00}}
\put(103.00,70.33){\line(-1,-2){5.33}}
\put(114.33,58.33){\line(-5,-4){8.67}}
\put(121.00,42.00){\line(-4,-1){11.00}}
\put(110.00,39.00){\line(-1,3){4.00}}
\put(106.00,51.00){\line(-1,1){8.33}}
\put(97.67,59.33){\line(-4,1){12.67}}
\put(85.00,62.33){\line(-6,1){14.00}}
\put(122.33,41.67){\circle*{2.67}}
\put(104.33,69.67){\circle*{2.67}}
\put(70.67,82.67){\circle*{2.67}}
\put(83.00,80.33){\vector(1,4){2.67}}
\put(87.67,88.67){\makebox(0,0)[cc]{ n}}
\end{picture}
\caption{Two-Dimensional Surface Loading}
\label{figsurf}
\end{figure}

Use of the {\tt POLA}r option permits the coordinates $x_1$ and $x_2$ to
be given as a radius and angle (in degrees) and internally converted to
cartesian form.

A common error is to have an incorrect sequence for the boundary segments
so that the outward normal points in the wrong direction.  When no
loads are computed it is necessary to carefully check the normal direction
to a patch.
Also check that the value of the proportional loading factor is non-zero.
If none of these errors are identified then the value of the search gap
can be increased by inserting the command
\begin{verbatim}
       GAP,value
\end{verbatim}
before the interpolation patch data.

For three dimensional problems the command has the structure
\begin{verbatim}
       CSURface
         type, data
         SURFace
         1,X_1,Y_1,P_1
         2,X_2,Y_2,P_2
         3,X_3,Y_3,P_3
         4,X_4,Y_4,P_4
                             ! blank termination record
\end{verbatim}
where {\tt type} is the data type selected from: {\tt GAP};
{\tt NORM}al traction or
{\tt DISP}lacement pattern.  No tangential option currently exists.
Also, only those element surface facets which lie on or within the
interpolation patch are selected.  No partial facets are permitted.

The surface option may be used only for elements whose surface facets
are either 3-node triangles or 4-node quadrilaterals.
