% FEAP Manual: Version 7.2a - Revised: August 2000

\documentclass[12pt]{report}

\topmargin 0.4in
\advance \topmargin by -\headheight
\advance \topmargin by -\headsep

\setlength{\textwidth}{15.5cm}
\setlength{\oddsidemargin}{+0.4cm}
\setlength{\evensidemargin}{+0.4cm}
\setlength{\textheight}{21.7cm}
\setlength{\parindent}{0pt}
\setlength{\parskip}{3mm}

\usepackage{amsmath}
%\usepackage{psfig}
\usepackage{epsf}
%\usepackage{fullpage}
\usepackage{graphicx}

\newcommand{\Bf}[1]{\boldsymbol{#1}}
\newcommand{\B}[1]{\mathbf{#1}}
\newcommand{\ul}{\underline}

\pagestyle{headings}

\title{\sl{FEAP} - - A Finite Element Analysis Program \\
\rule{4.5in}{1pt} \\ {\large{Version 7.2 User Manual}} \\}



\author{Robert L. Taylor \\
Department of Civil and Environmental Engineering \\
University of California at Berkeley \\
Berkeley, California 94720-1710\\
E-Mail: rlt@ce.berkeley.edu \\}




\date{August 2000}



\begin{document}
\bibliographystyle{plain}
\maketitle

\pagenumbering{roman}
\tableofcontents

\pagebreak
\pagenumbering{arabic}

\input sect1.tex  % Introduction

\input sect2.tex  % Problem definition

\input sect4.tex  % Manual organization

\input sect5.tex  % Input records

\input sect6.tex  % Mesh input data

\input sect7.tex  % Global Data

\input sect10.tex % Coord and Element Connect

\input sect9.tex  % Coordinate Transformations

\input sect8.tex  % Regions and Groups; Flexible/Rigid Bodies

\input sect11.tex % Nodal BC

\input sect12.tex % Surface load

\input sect13a.tex % Elements

\input sect13b.tex % Material models

\input sect14.tex % End and miscellaneous commands

\input sect15.tex % Mesh manipulation

\input sect19.tex % Contact

\input sect16.tex % Rigid body activation

\input sect17a.tex % Command language, time solutions
\input sect17b.tex % Prop,augment,show,hist,eigv etc.

\input sect18.tex % Plot

\input ackn.tex

% Bibliography

\bibliography{/home/rlt/Manual/book/biblio/book}

% Appendices

\appendix

\input ../lmesh/appenda.tex
\input ../lmani/appendb.tex
\input ../lcont/appendc.tex
\input ../lmacr/appendd.tex
\input ../lplot/appende.tex

\end{document}
