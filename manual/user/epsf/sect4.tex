\chapter[Manual Organization]{MANUAL ORGANIZATION}
\label{organ}

The user manual for {\sl FEAP} is separated into several distinct
parts.  Each part describes the specific function and
the input data required for the commands currently available
in the system.  The manual consists of the following general sections:
\begin{enumerate}
\item
Methods to describe input data records and files (Chapter \ref{record}).
\item
Description of the start of a problem,
control information, and mesh input data (Chapter \ref{meshin}).
\item
Description of the element library and material models
(Chapters \ref{elmlib} and \ref{matmods}).
\item
Terminating mesh description (Chapter \ref{end}).
\item
Manipulating mesh parts to tie and link degrees of freedom (Chapter 
\ref{manip}).
\item
Description of the solution command language (Chapter \ref{command}).
This section of the manual includes basic solution algorithms to solve
problems.
\item
Plot features contained within the program (Chapter \ref{plot}).
\end{enumerate}
More information about each of the user manuals is contained in the
following sections.
The various options and parameters for each command to describe
mesh input, problem solution, and plotting are included
in the appendices to this manual.
A separate {\it Programmer Manual} describing the procedures to
add features and elements is also 
available for users who wish to modify or extend the capabilities of {\sl FEAP}.
