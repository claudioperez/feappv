\chapter[Rigid Body Analysis]{RIGID BODY ANALYSIS}
\label{rigid}

The rigid body capabilities are split into two classes.  One for small
displacement problems where translation and rotation parameters are
linear and one for the large displacement case where the rotational parameters
appear in a non-linear form.

\section{Small Displacement Analyses}

For the small displacement case the treatment of rigidity may be performed
using a {\it master-slave} concept for prescribed degrees-of-freedom.  A simple
implementation is included in the current version which permits
degrees-of-freedom for a slave node or a constant coordinate value to be
represented
in terms of degrees-of-freedom at a master point.  It is possible to have
some degrees-of-freedom rigid while others remain flexible.  For example,
a floor slab of a building may be constrained to be rigid for in-plane
deformations but flexible in transverse (plate bending) motions.  The commands
for specifying the master-slave set are inserted after the mesh {\tt END}
command and before the first solution data set.  The basic structure is:
\begin{verbatim}
       MASTer
         TYPE (EV(i),i=1,n)

\end{verbatim}
The {\tt TYPE} options are : {\tt NODE}, {\tt SURF}ace, and {\tt GAP}.  For
{\tt NODE} the input record is:
\begin{verbatim}
         NODE (Xm(i),i=1,ndm) (Xs(i),i=1,ndm) (RLINK(i),i=1,ndf)
\end{verbatim}
The nodes closest to the specified coordinates will be selected as the
master ({\tt Xm}) and slave ({\tt Xs}) nodes.  Zero values in
the {\tt RLINK} pattern define the degrees-of-freedom to be considered during
the slave phase.  The pattern must be consistent for proper behavior.  Thus,
if the $x_1$ and $x_2$ displacements are slaved so must the $\theta_3$
rotation parameter.  Similarly, for other patterns.

The record for {\tt SURF}ace is input as:
\begin{verbatim}
         SURFace (Xm(i),i=1,ndm) dir (RLINK(i),i=1,ndf)
\end{verbatim}
Here in addition to the master node coordinates, the direction of a normal
to the plane passing through the master node must be given.  Thus if
{\tt Xm} is given as (0 0 5) and {\tt dir} as 3 then all other nodes
within the gap value with coordinates ($x_1$ $x_2$ and 5) will be treated
as slave nodes.  The value of the gap may be reset from its default value
of $10^{-8}$ using the {\tt GAP EV(1)} command.

\section{Large Displacement Analyses}

As noted in Section \ref{flexrigid}
{\sl FEAP} permits groups of finite elements to be declared as rigid
or flexible.  The commands {\tt RIGI}d and, optionally, {\tt FLEX}ible
are required in the mesh data (i.e., between the {\sl FEAP} problem
initiation record and the {\tt END} of mesh record); however,
in order to activate the rigid option, it is necessary to also define
the type of integrations to perform for the rigid bodies and to define
any interconnections ({\it joints}) that exist between different rigid
bodies or a rigid body and a flexible body node.  The activation is
achieved by inserting a {\tt RIGI}d command {\it after} the {\tt END}
of mesh record and {\it before} the first solution {\tt BATC}h
or {\tt INTE}ractive command.  Similarly, to define joint interconnections
the {\tt JOIN}t command is placed in the same location.

{\sl FEAP} will automatically constrain groups of rigid elements 
which are contiguous to flexible elements 
to perform a combined flexible-rigid body analysis.
At present the rigid body options are limited to solid (continuum)
elements only.
Both explicit and implicit transient solutions are possible; however, for
the explicit option only the Spherical (Ball and Socket) Joint described
below is permitted.
The implicit formulation is available for the energy-momentum formulation
only and permits the use of several types of joints and
the constraints are formulated using a Lagrange multiplier method.
It is not possible to consider closed loops
consisting of only rigid bodies since redundant Lagrange multiplier
constraints will exist.

\section{Activation of Rigid Bodies}

To activate the rigid body options and to
define the integration method the single record

\begin{verbatim}
       RIGId,Nrbdof,Npart,Ntype
\end{verbatim}
is inserted between the {\tt END} mesh command and the {\it first}
solution command ({\tt BATC}h or {\tt INTE}ractive).
In this statement {\tt Nrbdof} is the number of rigid body degree-of-freedoms,
{\tt Npart} is the partition number of the rigid body, and {\tt Ntype} is
the integration type.  For most analyses the parameters may be omitted and
{\sl FEAP} will insert correct default values.  The default values for
{\tt Nrbdof} are:
\begin{center}
\begin{tabular}{c | c}
Mesh Dimension & Value \\ \hline
  1 & 1 \\
  2 & 3 \\
  3 & 6 \\
\end{tabular}
\end{center}
By default {\tt Npart} is assigned to partition 1 and {\tt Ntype} is set
to the energy-conserving algorithm which is number 5. (N.B. Other options
have not been tested and, thus may not be operational).

\section{Joints}
\label{joint}

Rigid bodies may be interconnected using {\it joints}.
The specification of the joints is initiated using a {\tt JOIN}t command
which also is located after the {\tt END} mesh command and before the first
{\tt BATC}h or {\tt INTE}ractive solution command.  Two of the selections from
the library of joints are:
\begin{enumerate}
\item
Ball and Socket:  Two rigid bodies may rotate freely about a specified
point.  A ball and socket joint is specified by a record
\begin{verbatim}
       BALL,RB_1,RB_2,X,Y,Z
\end{verbatim}
where {\tt RB-1} and {\tt RB-2} are the rigid body numbers associated
with the ball and socket, and {\tt X}, {\tt Y}, and {\tt Z} are the reference
system coordinates for the location of the ball and socket.

\item
Revolute:  Two rigid bodies may be constrained to rotate relative to
a specified direction in the reference coordinate system.  A classical
revolute is formed by combining the {\sl FEAP} {\tt REVO}lute with a
{\tt BALL} joint. The revolute is specified as:
\begin{verbatim}
       REVOlute,RB_1,RB_2,X_1,Y_1,Z_1, X_2,Y_2,Z_2
\end{verbatim}
where now the two coordinate points identify the direction of the
rotational axis in the reference state.  This axis is free to rotate
in space unless constrained by other restraints.
\end{enumerate}

Other types of joints are described in Appendix A.

N.B. The rigid body options are in a development mode and are not
operational for all types of solution methods.
