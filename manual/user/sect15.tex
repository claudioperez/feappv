\chapter[Mesh Manipulation Commands]{MESH MANIPULATION COMMANDS}
\label{manip}

Once an initial mesh is completely defined it may be further processed
to: (1) merge nodes with the same coordinates using the \texttt{TIE} command,
or (2) force a sharing of degrees-of-freedom using the \texttt{LINK}
command.  These commands may be given in any order immediately following
the mesh \texttt{END} command.  While they may be in any order the data is
first saved in temporary files and \textsl{FEAPpv} later executes the
commands in a definite order.

\section{The TIE Command}
\label{tie}

The ability to merge nodes which have the same coordinates permits
the generation of a mesh in separate parts without having to consider
a common node numbering system between the individual parts.
The \texttt{TIE} command permits merging
based on material set numbers, region numbers, a range of node numbers,
or on all the defined node numbers.
The latter is achieved by entering the command as:
\begin{verbatim}
       TIE
\end{verbatim}
without any parameters.  A range of node numbers to search may be
specified also.  For example, if the merge is to be done only for
nodes numbered between 34 and 65 the command is issued as:
\begin{verbatim}
       TIE,, 34, 65
\end{verbatim}
It is however, not possible to merge nodes from two different ranges of
numbers.

It is also possible to merge parts based on material numbers.  For example,
if a problem with two bodies is generated using material set 1
for body one and material set 2 for body two,
a merge may be achieved for the parts of each body
without any possibility of merging nodes in body one to those in
body two.  This is achieved using the commands:
\begin{verbatim}
       TIE MATErial 1  1
       TIE MATErial 2  2
\end{verbatim}
If it is desired to tie nodes for materials 1 and 2 together, the command
\begin{verbatim}
       TIE MATErial 1  2
\end{verbatim}
may be used.

Alternatively, the nodes to be merged may be associated with a region.  In
this option it is necessary to include {\tt REGIon} commands as part of the
element generation process (i.e., using either {\tt ELEMent} or {\tt BLOCk}).
The basic command to merge parts in \textit{Region m} to those in
\textit{Region n} is
\begin{verbatim}
       TIE REGIon m  n
\end{verbatim}
The parameters \texttt{m} and \texttt{n} may have the same or different values.

When the tie option is used one node from a merged pair is deleted from
the mesh and its number on the element connections replaced by the retained
number.  It is not possible to plot or output values for the
deleted node.  If printing is in effect at the end of the mesh
generation process, the nodes deleted are listed in the output
file.  This is different than creating common solution
values for degree of freedoms associated with two nodes
using the \texttt{LINK} option described below.  In a link
option the node is not deleted, and thus projections may create a
different solution at the two nodes.

\section{The LINK Command}
\label{link}

The link option may be used to make the solution of one or more of
the degrees-of-freedom associated with two nodes have the same value.
This option is useful in creating repeating type solutions, that is, those
in which the solution on a surface is repeated on an identical surface
with a different location.\scite{zt1n}
The link is performed based on node numbers using the \texttt{LINK} command.
