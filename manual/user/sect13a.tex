\chapter[Element Library]{ELEMENT LIBRARY}
\label{elmlib}

{\sl FEAPpv} contains a library of standard elements and material models which may
be employed to solve a wide range of problems in solid and structural mechanics,
and heat transfer analysis.
In addition, users may program and add new elements to the program.
The type of element to
be employed in an analysis is specified as part of the \texttt{MATErial} data
sets.  The first record of each material data set also contains the material
property number.  Each material property number is an integer ranging from
one (1) to the maximum number of material sets specified on the control data
record (which immediately follows the \texttt{FEAPpv} start/title record); however,
as noted earlier,
the maximum number on material sets on the control data may be specified
as zero and {\sl FEAPpv} will
automatically compute the maximum number of material sets from the input data.
The second record of the material set data defines the type of element to be
used.  The library of standard elements includes the following types:

\begin{enumerate}
\item
SOLId - A solid element is used to solve continuum problems with
either small or large deformations.  Options exist to use finite
elements based on displacement or mixed formulations.
Small deformation elements contains a library of
models for elastic, viscoelastic, or elastoplastic constitutive equations.
Finite deformation elements contain only elastic material models.
For two dimensional problems each element is a quadrilateral
with between 4 and 9-nodes.  The two dimensional displacement formulation
also permits use of 3 or 6-node triangular elements.
The degrees of freedom on each node are displacements,
$u_i$, in the coordinate directions.
The degrees of freedom are ordered as:
2-D Plane problems, $u_x , u_y $, coordinates are $x , y$;
2-D Axisymmetric problems, $u_r , u_z $, coordinates are $r , z$.
For three dimensional problems only a displacement form is included and each
element is an 8-node hexahedron (brick) or a 4-node tetrahedron
with degree-of-freedoms $u_x , u_y , u_z$.
No finite deformation element is included for three dimensional solids.

\item
FRAMe - The frame element is used to model structural members which include
axial, bending, and shearing deformations only.  The model is formulated in
terms of force resultants which are computed by integration of stress components
over the cross-sectional area of the member.
Each element has 2-nodes and may be used in a two or three dimensional
problem.  The degrees of freedom on each node are: Displacements, $u_i$, in the
coordinate directions and a rotation, $\theta_z$, about the z-axis
for two dimensions
and rotations, $\theta_i$, about all axes for three dimensions.
The degrees of freedom are ordered as:
2-D $u_x , u_y , \theta_z$;
3-D $u_x , u_y , u_z , \theta_x , \theta_y , \theta_z$;

\item
TRUSs - The truss element is used to model structural members which include
axial deformations and axial forces only.  The axial
force resultant is computed by integration of the axial stress component
over the cross-sectional area of the member.
Each element has 2-nodes and may be used in one, two and three dimensional
problems.  The degrees of freedom on each node are displacements, $u_i$, in each
coordinate direction; thus, the number is the same as the spatial dimension
of the problem.
The degrees of freedom are ordered as: $u_x , u_y , u_z$

\item
PLATe - The plate element is used to model structural behavior of planar (flat)
bodies which have one dimension small compared to the two other dimensions.
The element may be used for small deformation analyses only and
includes bending and transverse shearing deformations.
Provisions are also included to permit modeling of plates for which the
transverse shearing deformations are ignored.  The model is formulated in
terms of force resultants which are computed by integration of stress components
over the thickness of the plate.
Each element may be a triangle with 3-nodes or a quadrilateral
with 4-nodes and is used in a two dimensional
problem.  The degrees of freedom on each node are: The transverse displacement,
$u_3 = w$, and rotations $\theta_x$ and $\theta_y$ about the coordinate axes.
The degrees of freedom are ordered as: $w , \theta_x , \theta_y$;

\item
SHELl - The shell element is used to model structural behavior of curved
bodies which have one dimension small (a thickness normal to
the remaining surface coordinates) compared to the other dimensions of
the surface.  The shell is for small deformations only and includes bending and
in-plane deformations only (no transverse shearing strains).
The model is formulated in
terms of force resultants which are computed by integration of stress components
over the cross-sectional thickness of the shell.
In two-dimensions the element is a 2-node ring sector whereas in
three-dimensions each element is a quadrilateral with 4-nodes.
The degrees of freedom on each node are: Displacements,
$u_i$, and rotations, $\theta_i$, about the coordinate axes.
The degrees of freedom are ordered as:
$u_r , u_z , \theta$ in two dimensions and
$u_x , u_y , u_z , \theta_x , \theta_y , \theta_z$ (6-dof) in three dimensions.

\item
MEMBrane- The membrane element is used to model structural behavior
of curved bodies which are thin and carry in-plane loading
only.  The element is generally unstable unless attached to a
contiguous solid or otherwise restrained.  The model is formulated in
terms of the in-plane force resultants and a cross-sectional thickness.
Each element is a quadrilateral with 4-nodes
and may be used only in a three dimensional
problem.  The degrees of freedom on each node are: Displacements,
$u_i$.  The degrees of freedom are ordered as:
$u_x , u_y , u_z $;

\item
THERmal -  The thermal element is used to compute temperatures in solid
bodies or truss elements.  The element solves the Fourier heat conduction
equation.
For two dimensional problems each element is a quadrilateral
with between 4 and 9-nodes or a triangle with 3 or 6-nodes.
For three dimensional problems each element
is a brick with 8-nodes or a tetrahedron with 4-nodes.
The degree of freedom on each node is temperature, $T$, and, by default,
is placed in the first position in the unknowns (i.e., first degree of freedom).
If the element is combined with a solid element to perform thermo-mechanical
analyses it is necessary to relocate the temperature degree of freedom using
the option on the material set element type record (see the \texttt{MATE}erial
set command description in the {\sl FEAPpv} Mesh User Manual in Appendix A).

\item
USER -  Each user element must be developed and added to the program.
Provisions are included which permit the addition of up to 5 additional
element modules to the program.
The shape of the element, the number of degrees of freedom at each node,
and other parameters may be set by the user.  See Reference \cite{zt1n}
or the {\sl FEAP} Programmers Manual\scite{feapp} for information on adding
a user element.
\end{enumerate}

Each element requires additional input data to describe the
specific constitutive model, the finite element formulation to be used,
loading applied to elements, etc.

As an example consider an analysis of a two dimensional
continuum (solid) with a single material and constrained to a plane strain
deformation state.  The problem is to be modeled in small strain
by a linear elastic material with isotropic properties.
\begin{verbatim}
       MATErial,1
         SOLId
         ELAStic ISOTropic 30e+06 0.3
                                  ! blank termination record
\end{verbatim}
The property \texttt{ELAStic} is required for all types of \texttt{SOLId}
elements.  The material records for an
analysis which includes linear elastic solid elements is shown above.

By default all problems are in small strain, however, the input can include
the option \texttt{SMALl} if desired as
\begin{verbatim}
       MATErial,1
         SOLId
         SMALl
         ELAStic ISOTropic 30e+06 0.3
                                  ! blank termination record
\end{verbatim}
Statements following the \texttt{SOLId} may be in any order.  If a finite
deformation problem is desired it is necessary to include the \texttt{FINIte}
statement as 
\begin{verbatim}
       MATErial,1
         SOLId
         ELAStic ISOTropic 30e+06 0.3
         FINIte
                                  ! blank termination record
\end{verbatim}

For two dimensional problems in small deformations there are three
element types: (1) A displacement model; (2) A mixed $\B{u}-p-\theta$ model
and (3) an enhanced strain model.  By default the displacement model is used.
The element type may be specified by including an added statement in
the material specification: The choices are \texttt{DISPlacement},
\texttt{MIXEd} or \texttt{ENHAnced}.
In finite deformations only the two dimensional displacement and mixed forms
exist.  For three dimensional models only the small strain displacement model
is included with the current version.

In two dimensional applications
the displacement and the mixed formulation may be described by elements
with between four (4) and nine (9) nodes.
A three (node) triangular element
may be formed for the displacement model by repeating the number of any
node or by specifying only three nodes on an element.
In three dimensional applications the element is described by an eight (8)
node hexahedron.  The displacement model may also describe wedge
(by coalescing appropriate nodes of the hexahedron) and a 4-node tetrahedron.

The material models and other options
available for use with the solid elements are described in the next chapter.

In two-dimensions, the {\it frame elements} can treat small and large
displacement problems.  This is controlled by including a \texttt{SMALl} (default) or \texttt{FINIte} statement in the material model specification.  It is also
possible to include the effects of shear deformation (Timoshenko beam theory)
or to have no shear deformation in the bending behavior (Euler-Bernoulli
theory).  This is controlled by an appropriate record in the material
data as 
\begin{verbatim}
       SHEAR ON
\end{verbatim}
or
\begin{verbatim}
       SHEAR OFF
\end{verbatim}
In three-dimensions only the small strain form without shear deformation is
available.

The frame elements are restricted to elastic behavior.
Frame elements without shear deformation
include effects of bending and axial deformations only.  Cubic interpolations
are used for the bending behavior and a linear displacement for the
axial behavior.
For forms which include shear deformation, linear interpolation along
the beam axis is used
for all displacements and the rotation, thus, it is necessary to use more
elements to obtain good accuracy.  All elements have
two nodes.  To define the orientation of the cross section for a three
dimensional analysis it is necessary to define a \texttt{REFE}r\-ence
\texttt{VECT}or, \texttt{DIRE}c\-tion, or \texttt{NODE}.
A frame element is included using the commands:
\begin{verbatim}
       MATErial,1
         FRAMe
             ....
\end{verbatim}
The required data for frame elements is the material model, cross section
data, and for three dimensional frames geometric information to orient the
coordinate axes of the cross section.  Typically, \texttt{ELAStic} models are
required and the
cross section data given by \texttt{CROSs} section properties.
The geometric data for orienting cross section axes
is given by \texttt{REFErence VECTor} or \texttt{REFErence NODE} options.

The {\it truss elements} include small and large deformation formulations.
The elements have two nodes and include a number of one dimensional
constitutive models as indicated in the next chapter.
The truss element is included using the commands:
\begin{verbatim}
       MATErial,1
         TRUSs
             ....
\end{verbatim}
Required data is material model (e.g., typically \texttt{ELAStic}) and cross
section \texttt{CROSs} giving the area of the section.
Optional data is the type of inelastic constitutive model.

The {\it plate element} is restricted to small deformation applications
in which only the bending response of flat slabs is included.  The problem
is treated as a two-dimensional problem for the mesh (in the $x_1$-$x_2$
coordinate plane).  Only
linear thermo-mechanical response is included for the material models.
Each element may be a three node triangle or a four node quadrilateral.
The plate element is included using the commands:
\begin{verbatim}
       MATErial,1
         PLATe
             ....
\end{verbatim}
Required data is the material model (e.g., \texttt{ELAStic}) and the thickness
given by the \texttt{THICk} option.

The {\it shell element} is capable of only small deformation analysis. 
The model includes bending and membrane strains only -
no transverse shearing deformation is included - thus restricting application
to thin shell problems only.
A resultant elastic formulation is provided.
A shell element is included using the commands:
\begin{verbatim}
       MATErial,1
         SHELl
             ....
\end{verbatim}
Required data is the elastic material model (e.g., \texttt{ELAStic})
and the thickness given by the \texttt{THICk} option.
Additional data for loading is also available.

A {\it membrane element} is derived from the shell element by deleting
the bending deformations, thus leaving only the in-plane
strain deformation terms.  Elements for small displacements
are included and are restricted to elastic behavior.
The membrane element is included using the commands:
\begin{verbatim}
       MATErial,1
         MEMBrane
             ....
\end{verbatim}
Required data is the material model (e.g., \texttt{ELAStic}) and the thickness
given by the \texttt{THICk} option.

The {\it thermal elements} are all based on a standard Galerkin (irreducible,
displacement) formulation.\scite{zt1n}  The
element shapes available are identical to those for the displacement
form of a solid element.

At present the finite deformation material
models for the solid elements do not permit a coupled thermo-mechanical
analysis.  The small deformation models for elasticity do permit coupled
thermo-mechanical analyses to be performed using an iterative solution
algorithm.\footnote{While the model is linear, there is no tangent terms for
the coupling between temperature and displacement.  Thus, solution requires
added iterations} The material behavior for the thermal analysis is a linear
Fourier model.  Both isotropic and orthotropic models are available.
The thermal element is included using the commands:
\begin{verbatim}
       MATErial,1
         THERmal
           FOURier ...
\end{verbatim}
Required data is the material model given by the \texttt{FOURier} option.
Note also that it is necessary to specify the glogal degree of freedom
for the temperature when displacements are present (see below).

The specification of {\it user elements} must contain a number of an
element module which has been added to {\sl FEAPpv}.
Each user developed element module
is designated as \texttt{subroutine elmtnn(...)}, where \texttt{nn} ranges
from \texttt{01} to \texttt{05}.
Accordingly, a typical set of data for a user element \texttt{elmt02}
is given as:
\begin{verbatim}
       MATErial,1
         USER      2              ! Use elmt02(...) module
         xxxxxx                   ! Additional data records
                                  ! blank termination record
\end{verbatim}

The first two records of the \texttt{MATE}rial set always must be:

\begin{verbatim}
       MATErial ma
         type unum mset doflist
\end{verbatim}
where \texttt{ma} is the material set number, \texttt{type} is the element type
(e.g., solid, truss, etc.), \texttt{unum} is the user element number,
\texttt{mset} is the material set number given for each element (by default
it is the material number - this option permits two material types
to access the same element connection list), and \texttt{doflist} is the
list of global degree-of-freedoms to assign the internal element order (by
default this is the order 1,2,3,...,ndf).  For the standard elements contained
in {\sl FEAPpv} it is one needs only the \texttt{type} parameter unless
degrees of freedom are to be relocated (e.g., for thermal analysis).
