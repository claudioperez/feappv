\heada{SOLUTION}{TRANsient}
\hspace{1.2cm}{{ tran,name,<v1,v2,v3> \hfill}}\\{\smallskip}
\hspace{1.0cm}{{ tran,off \hfill}}
\headb

The use of the command {\tt TRAN}sient indicates that a
transient solution is to be computed.  Several options are
implemented:
\begin{enumerate}
\item{
The Newmark-beta step-by-step integration of the equations of motion.
}
\item{
A Generalized Newmark method \texttt{GN11} for
first order ordinary differential equations such as heat transfer, etc.
}
\item{
A Generalized Newmark method \texttt{GN22} for
second order ordinary differential equations such as vibration, etc.
}
\item{
An explicit implementation of Newmark.
}
\item{
An \texttt{SS11} method for
first order ordinary differential equations such as heat transfer, etc.
}
\item{
An \texttt{SS22} method for
}
\end{enumerate}

The {\tt OFF} option turns off any active time integration algorithm returning
{\sl FEAPpv} to its default quasi-static solution mode.

The method used depends on the specified {\tt NAME} in the command.
\begin{enumerate}
\item{
Newmark Method ({\tt name} is {\tt newm} or blank)

The values of the Newmark parameters are  specified  as
follows:
\begin{center}
\begin{tabular}{l l l}
 v1 &=  beta  - &the Newmark parameter which primarily \\
    &           &controls stability (default is 0.25). \\
 v2 &=  gamma - &the Newmark parameter which primarily \\
    &           &controls numerical damping ( default \\
    &           &is 0.50)  Note: gamma must be greater \\
    &           &than or equal to 0.50. \\
\end{tabular}
\end{center}
This option does  not  permit  an  {\it explicit}  solution
using  beta  =  0.0, only implicit solutions are considered.
Accordingly, it is recommended that values of beta be set to
0.25 (the default value) unless there is a compelling reason
not to use this value.  With gamma set to 0.50 and beta  set
to  0.25  the  method  becomes the "average" acceleration or
trapezoidal method.}
\item{
Generalized Newmark (GN11) where ({\tt name} is {\tt gn11})

The GN11 method requires  one  parameter  for
{\tt v1}, be set.  Permissible values are between 0 (explicit) and
1 (backward Euler). (Default is 1). 
}
\item{
Generalized Newmark (GN22) where ({\tt name} is {\tt gn22})

The GN22 method requires  two  parameters  for
\texttt{v1} and \texttt{v2}, be set.
Permissible values are between 0 (explicit) and
1 (backward Euler). (Default for both is 0.5 - which is equivalent
to Newmark with $\beta = 0.25$ and $\gamma = 0.5$). 
}
\item{
Explicit Newmark Method ({\tt name} is {\tt expl})

This option permits the explicit form  of  the  Newmark
method to be implemented.  The input parameter is only

\begin{center}
\begin{tabular}{l l}
 v2  =& gamma  (default = 0.5)
\end{tabular}
\end{center}
}
\item{
Single Step (SS11) where ({\tt name} is {\tt ss11})

The SS11 method requires  one  parameter  for
{\tt v1}, be set.  Permissible values are between 0 (explicit) and
1 (backward Euler). (Default is 1). 
}
\item{
Single Step (SS22) where ({\tt name} is {\tt ss22})

The SS22 method requires  two  parameters  for
\texttt{v1} and \texttt{v2}, be set.
Permissible values are between 0 (explicit) and
1 (backward Euler). (Default for both is 0.5 - which is equivalent
to Newmark with $\beta = 0.25$ and $\gamma = 0.5$). 
}
\end{enumerate}

It is possible to specify nonzero values for  the  initial
velocity  in second order system integrators using the
command {\tt INIT}ial ( for initial values).  If the  initial
state  is  not in equilibrium an initial acceleration may be
obtained by using a {\tt FORM,ACCE} command before initiating
any transient state.  It is necessary for the parameters
to first be set using a {\tt TRAN}sient command.  It is also
possible to compute self equilibrating static states with
non-zero displacements and then switch to a dynamic solution.
Alternatively, a restart mode ({\tt REST}art) may be used to
start from a previously computed non-zero state.
\vfill\eject
