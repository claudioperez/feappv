\heada{SOLUTION}{SUBSpace}
\hspace{1.0cm}{{ subs,<prin,n1,n2,stol> \hfill}}
\headb

The {\tt SUBS}pace command requests the solution for {\tt n1}
eigenpairs of a problem about the current state.  An additional
{\tt n2} vectors are used to expand the subspace and
improve convergence (by default, {\tt n2} is set to the minimum of
{\tt n1} plus 8 or 2 times {\tt n1} or the maximum
number of eigenvalues in the problem).  The {\tt SUBS}pace command
must be preceded by the specification of the tangent stiffness
array using a {\tt TANG}ent command, and a mass array (either
a lumped mass by {\tt MASS,LUMP} or a consistent mass by {\tt MASS}).
Note that the smallest {\tt n1} eigenvalues and eigenvectors are
computed with reference to the current {\tt shift} specified on
the {\tt TANG}ent command.  If {\tt n2} is larger than the number of
non-zero mass diagonals it is truncated to the actual number
that exist.  Whenever {\tt n1} is close to the number of non-zero
mass diagonals one should compute the entire set since
convergence will be attained in one iteration (this applies
primarily to small problems).

Use of the {\tt PRIN}t option produces an output of all subspace
matrices in addition to the estimates on the reciprocals
of the {\it shifted} eigenvalues.  For large problems considerable
output results from a use of this option, and thus it
is recommended for small problems only.

All eigenvalues are computed until two subsequent
iterations produce values which are accurate to {\tt stol},
(default {\tt stol} = max( {\tt tol}, 1.d-12)).
\vfill\eject
