\heada{SOLUTION}{PROPortional load}
\hspace{1.2cm}{{prop,,$<$n1$>$ \hfill}} \\{\smallskip}
\hspace{1.0cm}{{prop,,$<$n1,n2$>$ \hfill}}
\headb

In the solution of transient or quasi-static problems
in which the {\tt TIME} command is used to describe each new
time state the loading may be varied proportionally. At each
time the applied loading will be computed from:
\begin{verbatim}
       F(i,t) = f0(i) + f(i)*prop(t)
\end{verbatim}
where {\tt f0(i)} is a fixed pattern which is initially zero but
may be reset using {\tt NEWF}orce; {\tt f(i)} are
the {\it force} and {\it displacement}
nodal conditions defined during mesh input or
revised during a {\tt MESH} command; and prop(t) is the
value of the proportional loading at time {\tt t}.  Up to ten
different proportional loading factors may be set.  Individual
proportional factors may be assigned to degree of
freedoms using the mesh command {\tt FPRO}portional.  If the
assigned proportional loading number defined by {\tt FPRO} is
zero, the sum of all active sets is taken as the proportional
factor.  If the proportional loading number defined
by 'fpro' is 'n1' then the value defined by set 'n1' only is
used.  This permits individual nodal loads to be controlled
by particular loading factors.

For the form {\tt PROP,,N1}, the specific proportional
loading is defined by specifying one set of records for each
of the 'n1' values up to a maximum of 10 (default for {\tt N1}
is 1, that is, {\tt PROP} alone inputs one set).  For the form
{\tt PROP,,N1,N2}, the specific data for proportional loadings
{\tt N1} to {\tt N2} are input.  Thus, {\tt PROP,,2,2} will assign the
input data set to proportional loading number 2.

Each set contains the following data:
\begin{verbatim}
       type, k, t-min, t-max, a(i),i=1,4
\end{verbatim}

The proportional loading may be specified as:
\begin{enumerate}
\item{
Type 1 is defined by:
\begin{equation}
Prop(t) = a_1 + a_2 \, (t - t_{min})
                 + a_3 \, \left(\sin( a_4 \, (t - t_{min}) \right)^{k}
\end{equation}
for all time values between {\tt t\_min} and {\tt t\_max}.  The value of
{\tt k} must be a positive integer all other parameters are real.

If a blank record is input the value of {\tt t\_min} is set to zero;
{\tt t\_max} to $10^8$; a(1), a(3), and a(4) are zero; and a(2) is
1.0 - this defines a ramp loading with unit slope.

Example: The following defines a linearly increasing
load to a maximum of 1.0 at time 10 and then a linearly
decreasing load to time 20, after which the loading is
zero:
\begin{verbatim}
       prop,,1,2
       1  0  0.0  20.  0.  0.1  0.0  0.0  ! Set 1
       1  0 10.0  20.  1. -0.2  0.0  0.0  ! Set 2
\end{verbatim}
Note that the negative slope is twice that of the increasing ramp.

Also, if individual nodal forced conditions (e.g.,
displacements or loads) have been assigned to proportional
load number 1 (using the mesh 'fpro' command),
the first input record result is used, whereas if
assigned to number 2 the second input record is used.
When no assignment is made or a zero is specified for
the dof using the {\tt FPRO}, {\tt EPRO}, and/or {\tt CPRO}
mesh commands the sum of the records is used.
}
\item{
Type 2 is a table input.  The input is as follows:
\begin{verbatim}
       prop,,3    ! Input proportional loading 3 only
       2,nn  (default nn is 1)
       t\_1   ,p\_1,   t\_2   ,p\_2   , ... ,t\_nn  ,p\_nn
       t\_nn+1,p\_nn+1,t\_nn+2,p\_nn+2, ... ,t\_2*nn,p\_2*nn
                  ! etc., terminate with blank record
\end{verbatim}
The time points must be in an increasing order.  After the
input of $t_1$, a zero time value terminates the input.
Linear interpolation is used between each pair of times, $t_i$
and $t_{i+1}$, for the two values, $p\_i$ and $p\_{i+1}$. This
option is particularly useful for specifying cyclic loadings.

Example:
\begin{verbatim}
       BATCH
         PROP,,3
       END
         2,4
         0.,0.  1.,1.  3.,-1.  5.,1.
         7.,-1.  8.,0.  0.,0.
                ! blank record
\end{verbatim}
gives a cyclic loading with linear behavior between the
times 0. and 8. and is zero thereafter.
}
\end{enumerate}
\vfill\eject
