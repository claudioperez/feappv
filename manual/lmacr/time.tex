\heada{SOLUTION}{TIME}
\hspace{1.2cm}{{time,,$<$t\_max$>$ \hfill}} \\{\smallskip}
\hspace{1.0cm}{{time,$<$set,t$>$ \hfill}}
\headb

The use of the {\tt TIME} command will increment the
current time by {\tt DT}, the current time increment.  In addition,
a new value of the proportional loading will be computed,
if necessary.  The value of the current time and proportional
loading are reported in the output (or to the
screen).  The time command also will perform the first
update for an active time integration algorithm of the equations
of motion (e.g., the Newmark-beta method) , as well
as, update the history data base for any elements with non-linear
constitutive equations (e.g., those which require
variables other than the displacement state to compute a
solution).  Accordingly, it is imperative to include a time
command for this class of problems.  Example: Time
dependent solution with loop control
\begin{verbatim}
       DT,,1.
       LOOP,,10
         TIME
          ..
         etc.
          ..
       NEXT
\end{verbatim}
Performs 10 time steps of a solution.

As an option, it is possible to specify the maximum
time that integration is to be performed.  Accordingly, when
a variable time step is employed the {\tt TMAX} parameter value may be
used as a convenient stop marker.  This also is essential if
an automatic time stepping algorithm is implemented. Example:
Time dependent solution  with  loop  control, terminate at specified time.
\begin{verbatim}
       DT,,1.
       LOOP,,10
         TIME,,5.0
          ..
         etc.
          ..
       NEXT
\end{verbatim}
Performs 10 time steps of a solution; however, if
the time reaches the value of 5.0 before the 10
steps terminate the execution.  This may happen if
the {\tt DT} value is automatically adjusted by 
another step in the solution process.

The current time may be set to a specified value, {\tt T},
using the command {\tt TIME,\-SET,\-T} (where {\tt T} is the value
desired).  No other action is taken.  This may be helpful in
certain steady state problems where solutions are desired
for certain specified times.
\vfill\eject
