\heada{SOLUTION}{VELOcity}
\hspace{1.2cm}{{ velo,,<n1,n2,n3>  \hfill}} \\{\smallskip}
\hspace{1.0cm}{{ velo,coor,idir,xi \hfill}} \\{\smallskip}
\hspace{1.0cm}{{ velo,all          \hfill}}
\headb

The command {\tt VELOcity} may be used to print the
current values of the velocity vector as follows:
\begin{enumerate}
\item{
Using the command:
\begin{verbatim}
       VELO,,n1,n2,n3
\end{verbatim}
prints out the current velocity vector for nodes
{\tt n1} to {\tt n2} at increments of {\tt n3} (default increment = 1).  If
{\tt n2} is not specified only the value of node {\tt n1} is
output.  If both {\tt n1} and {\tt n2} are not specified only
the first nodal velocity is reported.}
\item{
If the command is specified as:
\begin{verbatim}
       VELO,COOR,idir,xi
\end{verbatim}
prints all nodal quantities for the coordinate direction {\tt idir}
with value equal to {\tt xi}.

Example:
\begin{verbatim}
       VELO,COOR,1,3.5}
\end{verbatim}
prints all the nodal velocity which have $x_1$ = 3.5.
This is useful to find the nodal values along a particular
constant coordinate line.}
\item{
If the command is specified as:
\begin{verbatim}
       VELO,ALL
\end{verbatim}
all nodal velocities are output.}
\end{enumerate}

In order to output a velocity vector it is first necessary to
specify commands language instructions to compute the desired
values, e.g., for velocities perform a dynamic analysis.
\vfill\eject
