\heada{SOLUTION}{GEOMetric stiffness}
\hspace{1.2cm}{{geom \hfill}} \\{\smallskip}
\hspace{1.0cm}{{geom,on \hfill}} \\{\smallskip}
\hspace{1.0cm}{{geom,off \hfill}}
\headb

The command {\tt GEOM}etric stiffness command is used in two ways.
The first is to compute a {\it geometric} stiffness matrix for use
in linear buckling analysis.  This option is performfed when no
parameters are appended to the command. A parameter {\tt imtyp} is set
to 2 and each element then computes a contribution
to the geometric stiffness in the array {\tt S} when {\tt isw = 5}.

A geometric stiffness matrix may be used for eigen\-computations
(see solution command {\tt SUBS}pace).  Reported eigenpairs
correspond to linearized buckling for a loading multiplied
by the eigenvalue.  Not all elements have this feature.

The second use of the option is to enable and disable the geometric stiffnes
during {\tt TANG} and {\tt UTAN} computations.  For many problems the inclusion
of the geometric stiffness during early iterations of a Newton type solution
can lead to divergent results.  The geometric matrix may be disabled during
early iterations using the {\tt GEOM,OFF} command and then enabled for later
iterations using the {\tt GEOM,ON} command.  A typical example is:

\begin{verbatim}
       LOOP,time,nsteps
         TIME
         GEOM,OFF
         LOOP,newton,3
           TANG,,1
         NEXT
         GEOM,ON
         LOOP,newton,25
           TANG,,1
         NEXT
       NEXT,time
\end{verbatim}
where three iterations are performed with no geometric stiffness and, later,
additional iterations with the geometric stiffness.  At convergence each
loop can terminate before the number of specified iterations.  If this occurred
in the first loop one additional iteration would be made in the second loop.
\vfill\eject
