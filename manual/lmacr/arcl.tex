\heada{SOLUTION}{ARCLength}
\hspace{1.0cm}{{arcl,$<$xxxx,kfl,lfl$>$ \hfill}}
\headb

The {\tt ARCL}ength command computes a solution using an arclength
continuation method.

The {\tt kfl} options are defined as follows:

\begin{center}
\begin{tabular}{l l }
kfl = 0:& Normal plane, modified newton solution \\
        & N.B. kfl = 0: defaults to kfl = 2 \\
kfl = 1:& Updated normal plane, modified newton solution \\
kfl = 2:& Normal plane, full newton solution \\
kfl = 3:& Updated normal plane, full newton solution \\
kfl = 4:& Displacement control, modified newton solution \\
kfl = 5:& Displacement control, full newton solution
\end{tabular}
\end{center}

The {\tt lfl} options are defined as follows:
\begin{center}
\begin{tabular}{l l }
lfl = 0:& Use current values for arclength and load direction \\
        & (Initial default is calculated by first solution step). \\
lfl = 1:& Change current values for arclength and load direction
\end{tabular}
\end{center}

ARCL must be called once at the beginning of the
solution commands when a nonlinear problem is to be
solved using this method.  With this call all flags will
be set and retained to perform an arclength solution.
To turn arclength off after it has been activated issue {\tt ARCL,OFF}.

For the calculation of load deflection curves
specify {\tt PROP}\-ortional
load using default parameters;
the actual load level is computed by {\tt ARCL}.

\vfill\eject
