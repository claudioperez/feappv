\heada{SOLUTION}{TANGent}
\hspace{1.2cm}{{ tang,,<n1,v2> \hfill}} \\{\smallskip}
\hspace{1.0cm}{{ tang,line,<n1,v2,v3> \hfill}} \\{\smallskip}
\hspace{1.0cm}{{ tang,eigv,,n1 \hfill}}
\headb

The {\tt TANG}ent command computes a symmetric tangent stiffness matrix
about the current value of the solution state vector.  For
linear applications the current stiffness matrix is just the
normal {\it stiffness} matrix.

If the value of {\tt n1} is non-zero, a  force  vector  for
the  current residual is also computed (this is identical to
the {\tt FORM} command computation) - thus leading to greater
efficiency when both the tangent stiffness and a residual
force vector are needed.  The resulting equations are also solved
for the solution increment.  Thus, 
\begin{verbatim}
       TANGent,,1
\end{verbatim}
is equivalent to the set of commands
\begin{verbatim}
       TANGent
       FORM
       SOLVe
\end{verbatim}

If the value of {\tt v2} is non-zero a {\it shift} is applied
to the stiffness matrix in which the element mass matrix is
multiplied by {\tt v2} and subtracted from the stiffness matrix.
This option may be used with the {\tt SUBS}pace command to
compute the closest eigenvalues to the shift, {\tt v2}.   Alternatively,
the shift may be used to represent a forced vibration solution
in which all loads are assumed to be  harmonic
at a value of the square-root of {\tt v2} (rad/time-unit).

After the tangent matrix is computed, a triangular
decomposition is available for subsequent solutions using
{\tt FORM}, {\tt SOLV}e, {\tt BFGS}, etc.

In the solution of non-linear problems, using a full or
modified Newton method, convergence from any starting point
is not guaranteed.  Two options exist within available
commands to improve chances for convergence.  One is to use
a line search to prevent solutions from diverging rapidly.
Specification of the command {\tt TANG}ent,{\tt LINE} plus options
invokes the line search (it may also be used in  conjunction
with {\tt SOLV}e,{\tt LINE} in modified Newton schemes).  The
parameter {\tt v3} is typically chosen between 0.5 and 0.8
(default is 0.8).

The second option to improve convergence of non-linear
problems is to reduce the size of the load step increments.
The command {\tt BACK} may be used to {\it back-up} to the
beginning of the last time step (all data in the solution
vectors is reset and the history data base for inelastic
elements is restored to the initial state when the current
time is started). Repeated use of the back command may be
used. However, it applies only to the current time interval.
The loads may then be adjusted and a new solution with
smaller step sizes started.

The {\tt EIGV}alue option is used in transient algorithms to
compute eigenvalues of the (static) stiffness matrix.  If
{\tt IDEN}tity has been issued, then the shift given by non-zero
{\tt n1} is with respect to the identity otherwise the element
mass matrix is used.  Note, {\tt SUBS}pace is used to compute the
actual eigen-pairs.

The {\tt TANG}ent operation is normally the most time consuming
step in problem solutions - for large problems several
seconds are required - be patient!
\vfill\eject
