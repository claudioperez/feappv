\heada{SOLUTION}{ACCElerations}
\hspace{1.2cm}{{ acce,,n1,n2,n3    \hfill}}\\{\smallskip}
\hspace{1.0cm}{{ acce,coor,idir,xi \hfill}}\\{\smallskip}
\hspace{1.0cm}{{ acce,all          \hfill}}
\headb

The command {\tt ACCE}leration may be used to print the
current values of the acceleration vector as follows:

\begin{enumerate}
\item{
Using the command:
\begin{verbatim}
       acce,,n1,n2,n3
\end{verbatim}
prints out the current acceleration vector for nodes
{\tt n1} to {\tt n2} at increments of {\tt n3} (default increment = 1).  If
{\tt n2} is not specified only the value of node {\tt n1} is
output.  If both {\tt n1} and {\tt n2} are not specified only
the first nodal acceleration is reported.}

\item{
If the command is specified as:

\begin{verbatim}
       acce,coor,idir,xi
\end{verbatim}
prints all nodal quantities for the coordinate direction {\tt idir}

Example:
\begin{verbatim}
       acce,coor,1,3.5
\end{verbatim}
prints all the nodal accelerations which have $x_1$ = 3.5.

This is useful to find the nodal values along a particular
constant coordinate line.}

\item{
If the command is specified as:
\begin{verbatim}
       acce,all
\end{verbatim}
all nodal accelerations are output.}
\end{enumerate}

In order to output a acceleration vector it is first necessary to
specify commands language instructions to compute the desired
values, e.g., for accelerations perform a dynamic analysis.
\vfill\eject
