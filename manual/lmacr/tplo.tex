\heada{SOLUTION}{TPLOts}
\hspace{1.0cm}{{ tplo,,inc \hfill}} \\{\smallskip}
\hspace{1.4cm}{{ $<$ After end record give the data $>$}} \\{\smallskip}
\hspace{1.0cm}{{ disp,node,dof,x,y,z \hfill}} \\{\smallskip}
\hspace{1.0cm}{{ velo,node,dof,x,y,z \hfill}} \\{\smallskip}
\hspace{1.0cm}{{ acce,node,dof,x,y,z \hfill}} \\{\smallskip}
\hspace{1.0cm}{{ reac,node,dof,x,y,z \hfill}} \\{\smallskip}
\hspace{1.0cm}{{ arcl,node,dof \hfill}} \\{\smallskip}
\hspace{1.0cm}{{ stre,elmt,comp \hfill}} \\{\smallskip}
\hspace{1.0cm}{{ ener \hfill}} \\{\smallskip}
\hspace{1.0cm}{{ show \hfill}}
\headb

The {\tt TPLO}t command can be used to specify components of
displacement, velocity, acceleration, reaction,
arclength parameter, stress, and
energy which are to be saved to construct time history plots as a
post processing operation.  The command may be issued
several times; however, the total number of components to be
saved for each type of plot (time vs. displacement or time
vs. reaction, etc.) is limited to 20.  An exception is for stress
components where a maximum of 200 is permitted.  The {\tt inc} option
is used to specify the number of time steps between saving of information.
Each time the command
tplot is given components are added to the list.  The option
show may be used to echo the current list to the screen during
interactive executions.

Options which include both {\tt} and {\tt x,y,z} may be used in one
of two ways.  Giving the command as:
\begin{verbatim}
       xxxx,node,dof
\end{verbatim}
requires specific numbers to be provided for the {\tt node} and {\tt dof}
parameters.  The value of {\tt node} is an {\it active}
global node number of the mesh (i.e., one which has not been deleted by
at {\tt TIE} command).  Alternatively, the command may be given as:
\begin{verbatim}
       xxxx,,dof,x,y,z
\end{verbatim}
where {\tt x,y,z} are values for the necessary number of coordinates (ndm).
A search will be made to locate the node which is {\it closest} to the
coordinates given.

The {\tt DISP}lacement option will save the node and degree
of freedom value, together with the time in a file {\tt Pxxx.dis},
where {\tt xxx} is the name assigned for the input data file (with
the I stripped).  The components are on one record in the
order given during the tplot inputs.  Similarly for other
node based quantities.

The {\tt ENER}gy option maybe used to accumulate total
linear/angular momentum and kinetic/potential energy.

The {\tt ARCL}ength option output the arc-length load level
versus the selected nodal displacement dof.

The {\tt STRE}ss option will save the element and component
value, together with the time in a file {\tt Pxxxy.str}, where {\tt xxx}
is the name assigned for the input data file (with the I
stripped) and {\tt y} ranges between {\tt a} and {\tt j}.
The components are on one record in the order
given during the tplot inputs.  The meaning of components is
element dependent and each programmer must decide what is to
be saved.  Indeed the components need not be stresses, they
may be strains, internal variables, etc.

An example for the use of tplot is:
\begin{verbatim}
       BATCh
         TPLOt
       END
       stre,3,24
       stre,25,24
       stre,25,26
       disp,11,2
       disp,,2,5.2,4.3,-1.2
       show
                  ! blank termination record
\end{verbatim}
requests stress output for component 24 in element 3 and components 24
and 26 from element 25.  The program will also
output nodal displacement as requested by {\tt disp} for dof 2 at
node 11 and at the node located at the coordinates closest to
( 5.2, 4.3, -1.2).  Finally, the list will be echoed by the {\tt show} command.
\vfill\eject
