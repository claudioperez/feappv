\heada{SOLUTION}{STREss}
\hspace{1.2cm}{{ stre,,<n1,n2,n3> \hfill}}\\{\smallskip}
\hspace{1.0cm}{{ stre,all \hfill}}\\{\smallskip}
\hspace{1.0cm}{{ stre,coor,idir,xi \hfill}}\\{\smallskip}
\hspace{1.0cm}{{ stre.<node,n1,n2,n3> \hfill}}\\{\smallskip}
\hspace{1.0cm}{{ stre,erro \hfill}}
\headb

The {\tt STRE}ss command is used to output stress
results in elements {\tt n1} to {\tt n2} at increments of {\tt n2}
(default = 1), or at nodes using {\it projected} values.  Thus,
two options exist for reporting stress values. These are:
\begin{enumerate}
\item{
Stresses may be reported at selected points within each
element.  The specific values reported are described in
each element type.  In general elements report values
at gauss points.  The values at all points are reported
when the command {\tt STRE}ss,{\tt ALL} is used.}
\item{
For solid elements results may be reported at nodes
using the {\tt STRE}ss,{\tt NODE} option.  A projection method
using stresses at points in each element is used to
compute nodal values.  In general, nodal values are not
always as accurate as stresses within elements.  This
is especially true for reported {\it yield} stresses where
values in excess of the limit value result in the pro-
jection method employed.  For a mesh producing accurate
results inside elements this degradation should not be
significant.}
\item{
The command specified as:

\begin{verbatim}
       stre,coor,idir,xi
\end{verbatim}
prints all nodal stresses for the coordinate  direction
{\tt idir} with value equal to {\tt xi}. Example:

\begin{verbatim}
       stre,coor,1,3.5
\end{verbatim}
will print all the nodal stresses which have $x_1$ = 3.5.
This is useful in finding the nodal values along a par-
ticular constant coordinate line.}
\end{enumerate}

With the {\tt ERRO}r option {\tt STRE}ss computes element sizes
for adaptive mesh refinement. N.B. The error option does not function with
all  elements.
\vfill\eject
