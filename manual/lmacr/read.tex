\heada{SOLUTION}{READ}
\hspace{1.0cm}{{read,xxxx \hfill}}
\headb

The {\tt READ} command may be used to input the
values of displacements and nodal stresses previously com-
puted and saved using the {\tt WRIT}e command - it is primarily
used for plots related to deformations or nodal
stresses.  It is not intended for a restart option (see
{\tt REST}art) but may be used to restore displacement states of
linear and non-linear elastic elements (or other elements
with no data base requirements) for which reactions,
stresses, etc. may then be computed.

The values of {\tt xxxx} are used to specify the file name
(4-characters only), manipulate the file, and read displacements
and nodal stresses.  The values permitted are:

\begin{verbatim}
       xxxx = wind: Rewind current file.
       xxxx = back: Backspace current file.
       xxxx = clos: Close current file.
       xxxx = disp: Read displacement state from current file.
       xxxx = stre: Read nodal stress state from current file.
       xxxx = Anything else will set current filename.
\end{verbatim}                   
Only four characters are permitted and only one
file may be opened at any time.  Files may be
opened and closed several times during any run
to permit the use of more than one file name.

A {\tt READ} input is created using the {\tt WRIT}e command
which has identical options for {\tt xxxx} except for the backspace option.
\vfill\eject
