\heada{SOLUTION}{ITERative}
\hspace{1.2cm}{{iter \hfill}} \\{\smallskip}
\headb

The {\tt ITER}ative command sets the mode of solution to iterative
for the linear algebraic
equations generated by a {\tt TANG}ent.  Currently, iterative options
exist only for symmetric, positive definite
tangent arrays, consequently the use of the
{\tt UTAN}gent command must be avoided.  An
iterative solution requires a sparse matrix form of the tangent matrix to fit
within the blank common array dimensioned in the main program (see
\textsl{FEAP} programmer manual for procedures to reset the size of the
blank common array).

The symmetric equations are solved by a preconditioned conjugate gradient
method where the preconditioner is taken as the diagonal of
the tangent matrix.

The iterative solution option currently available is not very effective
for poorly conditioned problems.  Poor conditioning occurs when the material
model is highly non-linear (e.g., plasticity); the model has a long thin
structure (like a beam); or when structural elements such as frame, plate,
or shell elements are employed.  For compact three dimensional bodies with
linear elastic material behavior the iterative solution is often very
effective.

Another option is to solve the equations using a direct
method (see, the {\tt DIRE}ct command language manual page).
\vfill\eject
