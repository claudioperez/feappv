\heada{SOLUTION}{REACtions}
\hspace{1.2cm}{{ reac,,<n1,n2,n3> \hfill}} \\{\smallskip}
\hspace{1.0cm}{{ reac,coor,idir,xi \hfill}} \\{\smallskip}
\hspace{1.0cm}{{ reac,all \hfill}} \\{\smallskip}
\hspace{1.0cm}{{ reac,file \hfill}}
\headb

Nodal reactions may be computed for all nodes in the
problem and reported for nodes {\tt n1} to {\tt n2} at increments
of {\tt n3} (default increment = 1).  If {\tt n2} is not specified then only
the values for node {\tt n1} are output.  When both {\tt n1} and {\tt n2}
are not specified only total sum information is reported.

If the command is specified as:
\begin{verbatim}
       reac,coor,idir,xi
\end{verbatim}
prints all nodal reactions for the coordinate direction
{\tt idir} with value equal to {\tt xi}.
This option is useful in finding the nodal values along a particular
constant coordinate line.

Example:
\begin{verbatim}
       reac,coor,1,3.5
\end{verbatim}
will print all the nodal reactions which have x-1 = 3.5.

All reactions may be output using the {\tt REAC,ALL} command as:
\begin{verbatim}
       reac,all
\end{verbatim}

In addition to computing the reaction at each degree of freedom
an equilibrium check is performed by summing the values for
each degree of freedom over all nodes in the analysis.  The
sum of the absolute value of the reaction at each degree of
freedom is also reported to indicate the accuracy to which
equilibrium is attained.  It should be noted that problems
with rotational degrees of freedom or in curvilinear coordinates
may not satisfy an equilibrium check of this type.
For example, the sum for the radial direction in an axisymmetric
analysis will not be zero due to the influence of the
{\it hoop stresses}.

In addition to sums over all the nodes a sum is computed
for only the nodes output.  This permits the check of
equilibrium on specified series of nodes, or the computation
of the applied load on a set of nodes in which motions or
restraints are specified.

The {\tt FILE} option outputs reactions to the restart save
file with the extender {\tt .ren} (starting from {\tt re0}).  These
maybe used as input in Mesh (see Mesh {\tt REAC}tion command).
\vfill\eject
