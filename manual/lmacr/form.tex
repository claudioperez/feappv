\heada{SOLUTION}{FORM}
\hspace{1.2cm}{{form \hfill}} \\{\smallskip}
\hspace{1.0cm}{{form,acce \hfill}} \\{\smallskip}
\hspace{1.0cm}{{form,expl \hfill}}
\headb

The {\tt FORM} command computes the residual for the  current
time and iteration of a solution.  {\sl FEAPpv} is a general nonlinear
program and computes a residual for each solution by subtracting
from any applied loads: (1) The force computed for
the stresses in each element, often called the {\it stress divergence}
or {\it internal force} term; (2) If the problem is dynamic the
inertia forces.

At the end of each computation {\sl FEAPpv} reports the value
of the current residual in terms of its Euclidean norm,
which is the square root of the sum of squares of each component
of force.

If the {\tt ACCEleration} option is present an acceleration is computed by
solving the equation:
\begin{equation}
\Bf{M} \, \Bf{a} \ = \ \Bf{R}
\end{equation}
where $\Bf{M}$ is a consistent mass or a lumped
mass computed by the {\tt MASS} command
and must be computed before the specification
of the {\tt FORM} command.  This option may be used to compute
consistent accelerations for starting a transient
analysis using the Newmark type integration algorithms when
initial forces or initial displacements are specified.

If the {\tt EXPL}icit option is present {\sl FEAPpv} computes a solution to
the equations of motion (momentum equations) using an explicit
solution option.  Prior to using the {\tt FORM,\-EXPL}icit command
it is necessary to specify the explicit solution option using
the {\tt TRAN}sient,\-{\tt EXPL}icit command.
Explicit solutions are conditionally
stable, thus, a critical time step must be estimated
before attempting a solution.  An estimate to the critical
time step may be obtained using the maximum wave speed in
the material, $c$, and the closest spacing between nodes, $h$.
The maximum time step used must be less or equal to $h/c$.
\vfill\eject
