\heada{SOLUTION}{DATA}
\hspace{1.0cm}{{data,xxxx \hfill}}
\headb

During command language execution it is sometimes desirable to
progressively change parameters, e.g., the time step size or
the solution tolerance accuracy.  This could become cumber-
some and require an excessive number of commands if
implemented directly.  Accordingly, the {\tt DATA} command may
be used in instances when the time step or tolerance is to
be varied during a {\tt LOOP} execution.  The permissible values
for {\tt xxxx} are {\tt TOL} and {\tt DT}.
The actual values of the tolerance or time step size are
given after the {\tt END} statement using the data inputs
specified in the {\tt TOL} or {\tt DT} manuals.  For example, to
vary time steps during a loop the commands:

\begin{verbatim}
       LOOP,time,3
         DATA,DT
         TIME
         ...
         ...
       NEXT,time
       ....
       ...
       END
       DT,,0.1
       DT,,0.2
       DT,,0.4
\end{verbatim}
could be given to indicate three time steps with dt = 0.1,
0.2, and 0.4 respectively.
\vfill\eject
