\heada{SOLUTION}{NEWForce}
\hspace{1.2cm}{{newf \hfill}} \\{\smallskip}
\hspace{1.0cm}{{newf,zero \hfill}}
\headb

The use of the {\tt NEWF}orce command will set a fixed
pattern of nodal forces and displacements to the values of
the current pattern in boundary force and displacements plus the
previous "fixed" pattern.  That is:

\begin{enumerate}
\item{
For degree-of-freedoms where forces (loads) are  specified:
\begin{equation}
{\tt f0(i,1) = f(i,1)*prop(t) + f0(i,1)}
\end{equation}
}
\item{
For degree-of-freedoms where displacements  are  specified:
\begin{equation}
{\tt f0(i,2) = u(i)}
\end{equation}
where {\tt f0(i,n)} is the {\it fixed} pattern forces and displacements,
{\tt f(i,1)} is the pattern specified in force boundary loads,
{\tt prop(t)} is the current value of the proportional loading at the current
time $t$, and {\tt u(i)} is the current displacement value.
}
\end{enumerate}

When execution is initiated the values in {\tt f0(i,n)} are all
zero.  NOTE at restart they again will become all zero so
that caution must be exercised at any restart where {\tt NEWF}orce
had been used in generating the results.

The {\tt f0(i,n)} may be reset to zero using the {\tt NEWF}orce,{\tt zero}
command (values are not updated).
N.B. This only affects the current partition degree of freedoms.
\vfill\eject
