\chapter[Command Language Programs]{COMMAND LANGUAGE PROGRAMS}
\label{command}

{\sl FEAPpv} performs solution steps based upon user specified {\it command
language statements}.  The program provides commands which can be used to solve
problems using standard algorithms, such as linear static and transient
methods and Newton's method to solve non-linear problems.
Appendix \ref{appxc} describes all the programming
commands which are included in the current system.  These commands are
combined to define the solution algorithm desired.

To enter the solution command language part of {\sl FEAPpv}
the user issues the command BATCh or for an
interactive execution mode the command INTEractive.  A solution
is terminated by the command END (QUIT or just Q
also may be used in interactive mode).

Thus, the input file must contain at least one set of
\begin{verbatim}
       BATCh
        ....  ! Solution specification steps
       END
\end{verbatim}
or
\begin{verbatim}
       INTEractive
\end{verbatim}
for any solution process to be possible.

More than one BATCh-END and/or
INTE\-rac\-tive-END sequence may be used during the solution process.

All commands available in an installed program may be displayed during
an interactive mode of solution by issuing
the command {\tt MANUal,,3} followed by a {\tt HELP} command.
However, with a basic
set of commands quite sophisticated solution algorithms may
be constructed.  Each of the commands may be issued in a
lower or upper case mode.  For example, a command which
always should be issued when first solving a problem
is the {\tt CHECk} command.  In either a batch or interactive mode,
the command is issued as:
\begin{verbatim}
       CHECk       !perform check of mesh correctness
\end{verbatim}
This command instructs {\sl FEAPpv} to make
basic checks for correctness of the mesh data prepared by the user{\footnote
{The check part of user developed elements must be implemented for the
check command to work properly}}.
One of the basic checks is an assessment of the element volume (or area)
at each node based on the specified sequence of
element nodes.  If the volume Jacobian of an element is negative or zero
at a node a diagnostic will be written to the output file.  If
all the volumes (or areas) are negative most of the system element routines
will perform a resequencing of the nodes and repeat the check.
If the resequencing gives no negative results the mesh will be
accepted as correct.

A check
also may reveal and report element nodes which have {\it zero} volume.
This may be an error or may result from merging nodes on quadrilaterals
to form triangles.  This is an acceptable way to make 3-node triangular
elements from 4-node quadrilateral elements, but in other cases may not
produce elements preserving the order of interpolation of the quadrilateral.
{\it It is the responsibility of the analyst to check correctness of finite
element solution software.} One good procedure is the patch test in which
basic polynomial solutions, for which the user can compute exactly the correct
solution (by hand), can be checked (see Chapter 9 in
\textit{The Finite Element Method: Its Basis \& Fundamentals} by
Zienkiewicz, Taylor and Zhu\scite{zt1n6} for a description of the patch test).

The {\tt CHECk} command should always be used in situations where
either a new mesh
has been constructed or modifications to the element connection lists have
been made. {\it No analysis should be attempted for a mesh with}
{\it negative volumes as incorrect results will result}.  Note,
however, that if a correct mesh is produced after
the {\tt CHECk} command resequences nodes,
the data in the input file is {\it not corrected}, consequently, it
will be necessary to always use a {\tt CHECk} command when solving a
problem with this data input file.
Since the amount of output from a {\tt CHECk} can be quite large, it
is recommended that the user correct the mesh for subsequent
solutions.

\section{Problem Solving}

Each problem is solved by using a set of the command language statements
which together form the {\it algorithm} defining the particular solution
method employed.  The commands to solve a linear static problem are:
\begin{verbatim}
       BATCh               !initiate batch execution
         TANG              !form tangent matrix
         FORM              !form residual
         SOLVe             !solve equations
         DISPlacement,ALL  !output all displacements
         STREss,ALL        !output all element stresses
         REACtion,ALL      !output all nodal reactions.
       END                 !end of batch program
\end{verbatim}
The command sequence
\begin{verbatim}
       TANG
       FORM
       SOLVe
\end{verbatim}
is the basic solution step in {\sl FEAPpv} and for simplicity (and efficiency)
may be replaced by the single command
\begin{verbatim}
       TANG,,1
\end{verbatim}
This single statement is more efficient in numerical operations
since it involves only a single
process to compute all the finite element tangent and residual arrays,
whereas the three
statement form requires one for {\tt TANG} and a second for {\tt FORM}.
Thus,
\begin{verbatim}
       BATCh               !initiate batch execution
         TANG,,1           !form and solve
         DISPlacement,ALL  !output all displacements
         STREss,ALL        !output all element stresses
         REACtion,ALL      !output all nodal reactions.
       END                 !end of batch program
\end{verbatim}
is the preferred solution form.
Some problems have tangent matrices which are unsymmetric.  For these situations
the {\tt TANG}ent command should be replaced by the {\tt UTAN}gent command.
The statements {\tt DISP}lacement, {\tt STRE}ss, and {\tt REAC}tion control
information which is written to the output file and to the screen.
The commands {\tt PRIN}t and {\tt NOPR}int may be used to control or prevent
information appearing on the screen - information always goes to the
output file.  Printing to the screen is the default mode.  See Appendix \ref{appxc}
for the options to control the displacement, stress, and reaction outputs.

Additional commands may be added to the program given above.  For example,
inserting the following command after the solution step (i.e., the
{\tt TANG,,1} command) will produce a screen plot of the mesh:
\begin{verbatim}
       PLOT,MESH           !plot mesh
\end{verbatim}
Further discussion for plotting is given in Chapter \ref{plot}.

\subsection{Solution of Non-linear Problems}
\label{nonlin}

The solution of non-linear problems is often performed using Newton's
method which solves the problem
\begin{equation}
\B{R}(\B{u}) = \B{0}
\end{equation}
using the iterative algorithm

\begin{enumerate}
\item Set initial solution
\begin{equation}
\B{u}^0 = \B{0}
\end{equation}
\item Solve the set of equations
\begin{equation}
\B{K} \, \Delta\B{u}^i = \B{R}(\B{u}^i)
\end{equation}
where
\begin{equation}
\B{K} = - \, \frac{\partial \B{R}}{\partial \B{u}}
\end{equation}
\item
Update the solution iterate
\begin{equation}
\B{u}^{i+1} = \B{u}^i + \Delta\B{u}^i
\end{equation}
\end{enumerate}
The steps are repeated until a norm of the solution is less than some tolerance.

{\sl FEAPpv} implements the Newton algorithm using the
following commands:
\begin{verbatim}
       LOOP,iter,10     !perform up to 10 Newton iterations
         TANG,,1        !form tangent, residual and solve
       NEXT,iter        !proceed to next iteration
\end{verbatim}
The tolerance used for controlling the solution is
\begin{equation}
E^i = \Delta{\B{u}^i} \cdot \B{R}^i
\end{equation}
with convergence assumed when
\begin{equation}
E^i < tol \, E^0
\end{equation}
The value of the tolerance is set using the {\tt TOL} command (default
is $10^{-12}$).

While the sample above specifies 10 iterations, fewer will be used if
convergence is achieved.  Convergence is tested during the {\tt TANG,,1}
command.  If convergence is achieved, {\sl FEAPpv} transfers to the
statement following the {\tt NEXT} command.
If convergence is not achieved in 10 iterations, {\sl FEAPpv}
exits the loop, prints a {\tt NO CONVERGENCE} warning,
and continues with the next statement.
For the algorithm given above, the only difference between a converged
and non-converged exit from the loop is the number of iterations
used.  However, if there are commands inserted between the {\tt TANG} and
{\tt NEXT} statements they are not processed for the iteration in which
convergence is achieved.
Obviously, solutions which do not converge during a time step may produce
inaccurate results in the later solution steps.
Consequently, users should check the output log of non-linear solutions
for any {\tt NO CONVERGENCE} records.

Remarks:
\begin{enumerate}
\item
Blank characters before the first character in a command are ignored
by {\sl FEAPpv}, thus,
the indenting of statements shown is optional but provides for clarification
of key parts in the algorithm.

\item
In the above loop command the {\tt ITER} in the
second field is given to provide clarity.  This is
optional; the field may be left blank.
\end{enumerate}

By replacing the Newton steps
\begin{verbatim}
       LOOP,iter,10
         TANG,,1
       NEXT,iter
\end{verbatim}
with
\begin{verbatim}
       TANG             !form tangent only
       LOOP,iter,10     !perform 10 modified Newton iterations
         FORM           !form residual
         SOLVe          !solve linearized equations
       NEXT,iter        !proceed to next iteration
\end{verbatim}
a modified Newton algorithm results.  The modified Newton method
forms only one tangent
and each iteration is performed by computing and solving the residual equation
with the same tangent.  When {\sl FEAPpv} forms the tangent while in a
direct solution
of equations mode the triangular factors are also computed
so that the {\tt SOLVe} only performs re-solutions during
each iteration.  While a modified Newton method involves fewer computations
during each iteration it often
requires substantially more iterations to achieve a
converged solution.  Indeed, if the tangent matrix is an accurate linearization
for the non-linear equations, the asymptotic rate of convergence for a
Newton method is quadratic, whereas a modified Newton method is often
only linear (if the residual equation set
is linear the tangent matrix is constant
and both the Newton and modified Newton methods should converge after one
iteration, that is, iteration two should produce a residual which is
zero to within the computer precision).

The {\sl FEAPpv} command language is
capable of defining a large number of standard algorithms.  Each user is urged
to carefully study the complete set of available commands and the options
available for each command.
In order to experiment with the capabilities of the language, it is suggested
that small problems be set up to test any proposed command language program
and to ensure that the desired result is obtained.

\subsection{Solution of linear equations}
\label{eqsoln}

The use of Newton's method results in a set of linear algebraic equations
which are solved to give the incremental displacements.  {\sl FEAPpv} includes
several options for solving linear equations.  The default solution scheme
is the variable band, profile scheme discussed in
Chapter 16 of \textit{The Finite Element Method, Vol 1},
5th edition.\scite{zt1n}
This solution scheme may be used to solve problems where the incremental
displacements are either in real arithmetic or in complex arithmetic.  The
coefficient matrix of the linear equations results in large storage requirements
within the computer memory.

An iterative option is available based on a preconditioned conjugate
gradient scheme (PCG method).  The PCG method is applicable to symmetric,
positive definite coefficient arrays only.  Thus, only the {\tt TANG}ent
command may be used.  The PCG with diagonal preconditioner is requested
by the command {\tt ITER}\-a\-tion before the first {\tt TANG}\-ent
command.  Experience to date suggests the iteration
method is effective and efficient only for three dimensional linear
elastic solids problems.  Success has been achieved when the solids
are directly connected to shells and beam; however, use with thin
shells has resulted in very slow convergence - rendering the method
ineffective.  Use with non-linear material models (e.g., plasticity)
has not been successful in static problem applications.  Use of the PCG
method in dynamics improves the situation if a mass term is available for
each degree of freedom (i.e., lumped mass on frames with no rotational
mass will probably not be efficient).

\section{Transient Solutions}
\label{trans}

{\sl FEAPpv} provides several alternatives to construct transient solutions.
To solve a non-linear time dependent problem
using Newton's method with a time integration method
the following commands may be issued:
\begin{verbatim}
       DT,,0.01           !set time increment to 0.01
       TRANsient,method   !specify "method" for time stepping
       LOOP,time,12       !perform 12 time steps
         TIME             !advance time by 'dt' (i.e., 0.01)
         LOOP,iter,10     !perform up to 10 Newton iterations
           TANG,,1        !form tangent, residual and solve
         NEXT,iter        !proceed to next iteration
         DISP,,1,12       !report displacements at nodes 1-12
         STRE,NODE,1,12   !report stresses at nodes 1-12
       NEXT,time          !proceed to next time step
\end{verbatim}
In addition to output for {\tt DISP}lacement, transient algorithms permit
the output of {\tt VELO}\-city and {\tt ACCE}l\-er\-a\-tion (see Appendix \ref{appxc}).

{\sl FEAPpv} provides several alternatives to construct transient solutions.
A transient solution is performed by giving the solution command language
statement
\begin{verbatim}
       TRANsient,method
\end{verbatim}
The type of transient solution to be performed
depends on the {\it method} option specified.
{\sl FEAPpv} solves three basic types of transient formulations:

\subsection{Quasi-static solutions}
\label{static}

The governing equation to be solved by the quasi-static option is expressed
as:
\begin{equation}
\B{R} (t) ~=~ \B{F}(t) ~-~ \B{P}( \B{u}(t) ) ~=~ {\bf 0}
\end{equation}
where, for example, the $\B{P}$ vector is given by the stress divergence
term of a solid mechanics problem as:
\begin{equation}
\B{P}( \B{u}(t) ) ~~=~~ \B{P}_{\sigma} ~=~
\int_{\Omega} \B{B}^T \, \Bf{\sigma} \, dV
\end{equation}
The solution options for this form are:

\begin{enumerate}
\item
The default algorithm which solves
\begin{equation}
\B{R} (t_{n+1}) ~=~ \B{F}(t_{n+1}) ~-~ \B{P}( \B{u}(t_{n+1}) ) ~=~ {\bf 0}
\end{equation}
using the commands
\begin{verbatim}
       LOOP,time,nstep
         TIME
         LOOP,Newton,niters
           TANG,,1
         NEXT
         ... Outputs
       NEXT
\end{verbatim}
The default option does not require a {\tt TRAN}sient command; however it
may also be specified using the command
\begin{verbatim}
       TRANsient,OFF
\end{verbatim}

\item
Quasi-static solutions may also be solved using a generalized
mid\-point con\-fig\-uration for the residual equation.  This option
is specified by the command
\begin{verbatim}
       TRANsient,STATic,alpha
\end{verbatim}
and solves the equation
\begin{equation}
\B{R} (t_{n+\alpha}) ~=~ \B{F}(t_{n+\alpha}) ~-~ \B{P}( \B{u}(t_{n+\alpha}) )
~=~ {\bf 0}
\end{equation}
where
\begin{equation}
\B{u}(t_{n+\alpha}) ~=~
\B{u}_{n+\alpha} ~=~ (1 - \alpha) \, \B{u}_n \,+\, \alpha \, \B{u}_{n+1}
\end{equation}
and
\begin{equation}
\B{F}(t_{n+\alpha}) ~=~
\B{F}_{n+\alpha} ~=~ (1 - \alpha) \, \B{F}_n \,+\, \alpha \, \B{F}_{n+1}
\end{equation}

The parameter $\alpha$ must be greater than zero; the default value is 0.5.
Setting $\alpha$ to 1 should produce answers identical to those from
option 1.
The transient option to be used must be given prior to specifying the time loop
and solution commands shown above.
\end{enumerate}

\subsection{First order transient solutions}
\label{trans1}

The governing equation to be solved for first order transient solutions
is expressed as:
\begin{equation}
\B{R} (t) ~=~ \B{F}(t) ~-~ \B{P}( \B{u}(t) , \dot{\B{u}} (t) ) ~=~ {\bf 0}
\end{equation}
where, for example, $\B{u}$ are the nodal temperatures $\B{T}$ and the $\B{P}$
vector is given by:
\begin{equation}
\B{P}~~=~~ \int_{\Omega} (\nabla N)^T \, \B{q} \, dV
~+~ \B{C} \, \dot{\B{T}}
\end{equation}
with $\B{C}$ the heat capacity matrix.

The solution options for this form are:

\begin{enumerate}
\item
{A backward Euler method which solves the problem
\begin{equation}
\B{R} (t_{n+1})
~=~ \B{F}(t_{n+1}) ~-~ \B{P}( \B{u}(t_{n+1}) , \dot{\B{u}}_{n+1} (t) )
~=~ {\bf 0}
\end{equation}
where
\begin{equation}
\dot{\B{u}}_{n+1} ~=~ \frac{1}{\Delta t} [ \B{u}_{n+1} \,-\, \B{u}_n ]
\end{equation}
The command:
\begin{verbatim}
       TRANsient,BACK
\end{verbatim}
is used to specify this solution option.}

\item
{A generalized midpoint method which solves the problem
\begin{equation}
\B{R} (t_{n+\alpha}) ~=~ \B{F}(t_{n+\alpha})
~-~ \B{P}( \B{u}(t_{n+\alpha}) , \dot{\B{u}}_{n+\alpha} (t) ) ~=~ {\bf 0}
\end{equation}
where
\begin{equation}
\dot{\B{u}}_{n+\alpha} ~=~ \frac{1}{\Delta t} [ \B{u}_{n+1} \,-\, \B{u}_n ]
\end{equation}
This solution option is selected using the command
\begin{verbatim}
       TRANsient,GN11,alpha
\end{verbatim}
where $0 < \alpha \le 1$ (default is 0.5); $\alpha = 1$ should produce answers
identical to those from the backward Euler option.}
Alternatively, the \texttt{SS11} algorithm may be used.\scite{zt1n}
\end{enumerate}

\subsection{Second order transient solutions}
\label{trans2}

The governing equation to be solved for second order transient solutions
is expressed as:
\begin{equation}
\B{R} (t) ~=~
\B{F}(t) ~-~ \B{P}( \B{u}(t) , \dot{\B{u}} (t) , \ddot{\B{u}}(t) )
~=~ {\bf 0}
\end{equation}
where, for example, the $\B{P}$ vector is given by:
\begin{equation}
\B{P}~~=~~ \B{P}_{\sigma} ~+~ \B{C} \, \dot{\B{u}} ~+~ \B{M} \, \ddot{\B{u}}
\end{equation}
with $\B{C}$ the damping and $\B{M}$ the mass matrix.

The solution options for second order problems are:

\begin{enumerate}
\item
{A Newmark method\scite{newmark} which solves the problem
\begin{equation}
\B{R} (t_{n+1}) ~=~ \B{F}(t_{n+1})
~-~ \B{P}( \B{u}_{n+1} , \B{v}_{n+1} , \B{a}_{n+1}) ~=~ {\bf 0}
\end{equation}
where
\begin{equation}
\B{v}_n ~=~ \dot{\B{u}}_n ~~~~~;~~~~~ \B{a}_n ~=~ \ddot{\B{u}}_n
\end{equation}
with updates computed as:
\begin{equation}
\B{u}_{n+1} ~=~ \B{u}_n \,+\, {\Delta t} \, \B{v}_n \,+\,
{\Delta t}^2 \, [ \, (0.5 - \beta ) \, \B{a}_n
\,+\, \beta \, \B{a}_{n+1} ,\ ]
\end{equation}
and
\begin{equation}
\B{v}_{n+1} ~=~ \B{v}_n \,+\, {\Delta t} \,[\, (1 - \gamma) \, \B{a}_n
\,+\, \gamma \, \B{a}_{n+1} \, ]
\end{equation}
in which $\beta$ and $\gamma$ are parameters controlling stability and
numerical dissipation.  The command
\begin{verbatim}
       TRANsient,NEWMark
\end{verbatim}
is used to select this integration scheme.  Optionally, the command
\begin{verbatim}
       TRANsient
\end{verbatim}
or
\begin{verbatim}
       TRANsient,GN22
\end{verbatim}
or
\begin{verbatim}
       TRANsient,SS22
\end{verbatim}
also selects the Newmark algorithm.\scite{zt1n}
Default values for classical Newmark are $\beta = 0.25$ and $\gamma = 0.5$.  
Default values for the \texttt{GN22} or \texttt{SS22} algorithm
are $\theta_1 = \theta_2 = 0.5$.

The second order problem using the Newmark method may require special
care in computing the initial state if non-zero initial conditions
or loading terms exist.  To compute the initial state it is necessary to
first compute a mass matrix and then the initial accelerations.  The
commands are
\begin{verbatim}
       TRANsient,NEWMark
       INITial (DISPlacements and/or RATEs)
       FORM,ACCEleration
       LOOP,time,nstep
         TIME
         LOOP,Newton,niters
           TANG,,1
         NEXT
         ... Outputs
       NEXT
\end{verbatim}
It is also necessary to use this sequence for the following method.
If $\B{F} (0)$, $\B{u} (0)$, and $\B{v} (0)$ are zero, the
{\tt FORM}, {\tt ACCE}leration command should be omitted to conserve
memory resources.}  The \texttt{GN22} and \texttt{SS22} algorithms
are self starting and do not require the above special starting condition.

In the above the setting of any non-zero initial displacements or rates
may be specified using the {\tt INIT}ial command.  The initial command
requires additional data which in a {\tt BATC}h solution option appears
immediately after the {\tt END} command.  In an interactive mode a user
receives a prompt to specify the data.
\end{enumerate}

{\sl FEAPpv} permits the type of transient problem
to be changed during the solution phase.  Thus, it is possible to
compute a configuration using a quasi-static option and then change
to a solution mode which includes the effects of rate terms (e.g.,
inertial effects).

\section{Transient Solution of Linear Problems}
\label{tranlin}

The solution of second order linear equations by the finite element method
leads to the set of equations
\begin{equation}
\label{eqm1}
\B{M} \, \ddot{\B{u}}(t) ~+~ \B{C} \,\dot{\B{u}}(t) ~+~ \B{K} \,\B{u}(t)
~=~ \B{F}(t)
\end{equation}
In structural dynamics,
the matrices $\B{M}$, $\B{C}$, and $\B{K}$
denote {\it mass}, {\it damping}, and {\it stiffness}, respectively.  The
vector $\B{F}$ is a force vector.
For the case where $\B{M}$, $\B{C}$, and $\B{K}$ are constant {\it symmetric}
matrices a
solution to Eq. \ref{eqm1} may be constructed
by partitioning the solution into the parts
\begin{equation}
\label{eqm2}
\B{u} ~=~ \left[
\begin{matrix} \B{u}_u \\ \B{u}_s \\ \end{matrix}
\right]
\end{equation}
where $( \cdot )_u$ denotes an unknown part and $( \cdot )_s$ a specified
part.
With this division, the equations are then written in the
form:
\begin{equation}
\label{eqm3}
\left[
\begin{matrix}
\B{M}_{uu} & \B{M}_{us} \\
\B{M}_{su} & \B{M}_{ss} \\
\end{matrix}
\right] \, \left[
\begin{matrix}
\ddot{\B{u}}_u \\ \ddot{\B{u}}_s \\
\end{matrix}
\right] + \left[
\begin{matrix}
\B{C}_{uu} & \B{C}_{us} \\
\B{C}_{su} & \B{C}_{ss} \\
\end{matrix}
\right] \, \left[
\begin{matrix}
\dot{\B{u}}_u \\ \dot{\B{u}}_s \\
\end{matrix}
\right] + \left[
\begin{matrix}
\B{K}_{uu} & \B{K}_{us} \\
\B{K}_{su} & \B{K}_{ss} \\
\end{matrix}
\right] \, \left[
\begin{matrix}
\B{u}_u \\ \B{u}_s \\
\end{matrix}
\right] ~=~ \left[
\begin{matrix}
\B{F}_u \\ \B{F}_s \\
\end{matrix}
\right]
\end{equation}

A solution is then constructed by first solving the first row of these
equations. The value of the reactions (i.e., $\B{F}_s$) associated
with the known part of the solution $\B{u}_s$ may be computed later if it is
needed.  The solution of the first row of these equations may be
constructed by several approaches.  \textsl{FEAPpv} uses only an integration in
time directly using a numerical step-by-step procedure (e.g., the Newmark
method).
