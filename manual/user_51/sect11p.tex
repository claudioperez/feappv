\subsection{Periodic boundary conditions}
\label{periodicbc}

\subsubsection{Specified displacement conditions}

For some problems in stress and/or thermal analysis of solids
it is desirable to specify non-zero nodal displacements
that result from applying a homogenious non-zero displacement or
thermal gradient.  For such situations
the displacement of the node is computed from 
\begin{displaymath}
\B{u}_a = \B{G} \, \B{x}_a
\end{displaymath}
where $\B{u}_a$ and $\B{x}_a$ are the displacement and coordinate of node $a$;
and $\B{G}$ is a \textit{specified} displacement gradient computed from
\begin{displaymath}
\B{G} = \begin{bmatrix}
\dpdif{u_1}{x_1} & \dpdif{u_1}{x_2} & \dpdif{u_1}{x_3} \\
\dpdif{u_2}{x_1} & \dpdif{u_2}{x_2} & \dpdif{u_2}{x_3} \\
\dpdif{u_3}{x_1} & \dpdif{u_3}{x_2} & \dpdif{u_3}{x_3} \end{bmatrix}
= \begin{bmatrix}
G_{11} & G_{12} & G_{13} \\[+6pt]
G_{21} & G_{22} & G_{23} \\[+6pt]
G_{31} & G_{32} & G_{33} \end{bmatrix}
\end{displaymath}
and the thermal gradient from
\begin{displaymath}
T_a = \B{G} \, \B{x}_a
\end{displaymath}
where $T_a$ are nodal temperatures and $\B{G}$ is a \textit{specified}
thermal gradient computed from
\begin{displaymath}
\B{G} = \begin{bmatrix}
\dpdif{T}{x_1} & \dpdif{T}{x_2} & \dpdif{T}{x_3} \end{bmatrix} 
\end{displaymath}

\subsubsection{Data input for periodic conditions}

\textsl{FEAPpv} provides for a treatment for periodic behavior on
a representative volume element (RVE) using an energy balance based on
Hill-Mandel theory\scite{kouznetsova01a,kouznetsova02a,zt2n7}.
For small strain problems the data input is given as
\begin{verbatim}
       PERIODIC HILL
         MECHanical <PROP n_u>
           eps_11 eps_12 eps_13
           eps_21 eps_22 eps_23
           eps_31 eps_32 eps_33
\end{verbatim}
which are components of the \textit{tensor} strains, i.e. $\epsilon_{ij} = 
\gamma_{ij}/2$;
and for finite deformation by
a \textit{displacement gradient} specified by
\begin{verbatim}
       PERIODIC HILL
         MECHanical <PROP n>
           G_11 G_12 G_13
           G_21 G_22 G_23
           G_31 G_32 G_33
\end{verbatim}
and for the thermal problem by
\begin{verbatim}
       PERIODIC HILL
         THERmal <PROP n>
           G_1 G_2 G_3
\end{verbatim}
In the above the \texttt{PROP n} permits subjecting the boundaries
to a time dependent behavior, but at the same spatial gradient distribution.
