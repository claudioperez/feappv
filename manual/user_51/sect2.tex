\chapter[Problem Definition]{PROBLEM DEFINITION}
\label{def}

To perform an analysis using the finite element method the first step is
to subdivide the region of interest into elements and nodes.  In this
process the analyst must make a choice on: (a) the type of elements to
use, (b) where to place nodes, (c) how to apply the loading and
boundary restraints, (d) the appropriate material model
and values of its parameters in each element, and (e) any other aspects
relating to the particular problem.  The specification of the node and
element data defines what we will subsequently refer to as the
{\it finite element mesh} or, for short, the {\it mesh} of the problem.  
In order to complete a problem specification it is necessary also to
specify additional data, e.g., boundary conditions, loads, etc..

Once the analyst has defined a model of the problem to be solved it is 
necessary to  define the nodal and element data in a form
which may be interpreted by {\sl FEAPpv}.  The steps to define 
a mesh for {\sl FEAPpv}
are contained in Chapters \ref{meshin} to \ref{manip}
and the input data for several
example problems is described in the {\sl FEAP Example Manual}.\scite{feape}
Each of the commands available for constructing mesh data for \textsl{FEAPpv}
is described in Appendix A.

The second phase of a finite element analysis is to specify the 
solution algorithm for the problem.  This may range from a simple
linear static (steady state)
analysis for one loading condition to a more complicated transient
non-linear analysis subjected to a variety of loading conditions.  {\sl FEAPpv}
permits the user to specify the solution algorithm utilizing
a solution command language which is described in Chapter \ref{command} and
also illustrated in the book.\scite{zt1n,zt2n}
Each solution command is also described in Appendix B of this report.

\section{Execution of FEAPpv and Input/Output Files.}

The execution of {\sl FEAPpv} is initiated by issuing the command:\footnote{Users
may give a different name for the executable during the compilation of the full
system -- it is then necessary to specify this name instead of the one stated above.}
\vskip 0.1in \par\noindent
\begin{verbatim}
            FEAPpv
\end{verbatim}
In PC use it is possible to execute the program using standard windows options
or to open an MS-DOS window and execute with the above command.
If this is a first execution of the program it is necessary to
provide names for the file containing
the input data and those to receive output information.  Upon a successful first
execution of the program a file \texttt{feappvname} will be written to disk to
preserve the name for each of the input and output disk files.
If it is desired to reinstall the program the \texttt{feappvname}
file should be deleted and the \texttt{FEAPpv} command then reissued.

For each subsequent execution of the program using the \texttt{FEAPpv} command,
the analyst receives prompts for a new input data filename,
as well as for the filenames which are to contain the
output of results and diagnostics, and
restart files (used if subsequent analyses are desired starting 
with the final results of a previous execution).
The name of a default selection will also be indicated and may be accepted
by pressing the return (enter) key without specifying any other data.
Prior to running {\sl FEAPpv} it is necessary to create the input data file
using a standard text editor or word processing system. The other
files are created by {\sl FEAPpv}.  A large part of the remainder of
this manual is directed to defining
the steps needed to create a valid input data file and to describe the
command language instructions needed to solve and output results for
a problem.

\section{Modification of Default Options}

At the time that the executable version of {\sl FEAPpv} is created
default values for several parameters may be set in file {\tt feappv.f}.
These default parameters may be changed without recompilation
by creating a file named {\tt feap.ins} which contains the new values for
specific parameters.  This file must be placed in each directory where
problems are to be solved.  The {\tt feap.ins} file contains separate records
which define the default parameters to be employed during any solution.  The
current options are given in Table \ref{tab22}.

\begin{table}
\begin{center}
\begin{tabular}{l l l |  l}
Option & Parameter 1 & Parameter 2 & Description \\ \hline
manfile & mesh & path & Path to locate MESH \\
 & & & COMMAND manual pages \\
        & macr & path & Path to locate SOLUTION \\
 & & & COMMAND manual pages \\
        & plot & path & Path to locate PLOT \\
 & & & COMMAND manual pages \\
        & elem & path & Path to locate USER \\
 & & & ELEMENT manual pages \\ \hline
noparse & & & Assumes input data is mostly \\
 & & & numeric  \\
parse & & & Assumes input data contains \\
 & & & parameters  \\ \hline
graphic & prompt & off & Turns off contour prompts \\
        &        & on  & Turns on contour prompts \\
        & default & off & Turns off graphics defaults \\
        &         & on  & Turns on graphics defaults \\ \hline
postscr & color & reverse & Makes color PostScript files \\
 & & & with reversed order. \\
        &       & normal  & Makes color PostScript files \\
 & & & with normal order. \\ \hline
helplev & basic & & Default level for commands \\
 & & & Same as: MANU,0 \\
        & interm & & Default level for commands \\
 & & & Same as: MANU,1 \\
        & advance & & Default level for commands \\
 & & & Same as: MANU,2 \\
        & expert & & Default level for commands \\
 & & & Same as: MANU,3 \\
increment & value & & Set increment value change to \\
 & & & force reduction in array size. \\
\end{tabular}
\caption{Options for Changing Default Parameters}
\label{tab22}
\end{center}
\end{table}
