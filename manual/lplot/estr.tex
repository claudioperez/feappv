\heada{PLOT}{ESTRess}
\hspace{1.0cm}{{estr,n1,n2,n3 \hfill}}
\headb

This command functions exactly like {\tt STRE}ss, except
that the quantities plotted are done without inter-element smoothing.

The command plots contours
of stresses (or other element variables), where {\tt n1} is the component to be
plotted and {\tt n2} is the number of contours (same as for {\tt CONT}our
including shading options).  The definitions of {\tt n1} for 2 and 3
dimensional elasticity problems are:

\begin{center}
\begin{tabular}{c | l}
  n1 & Component \\ \hline
   1 & 11-stress \\
   2 & 22-stress \\
   3 & 33-stress \\
   4 & 12-stress \\
   5 & 23-stress \\
   6 & 31-stress \\
   7 & 1-heat flux \\
   8 & 2-heat flux \\
   9 & 3-heat flux
\end{tabular}
\end{center}

The {\tt n3} parameter is used for filled (solid color) stress
plots as follows:

\begin{center}
\begin{tabular}{r | l}
    n3 & Action \\ \hline
     0 & superpose mesh on plot \\
     1 & suppress showing mesh \\
    -1 & suppress showing mesh and \\
       & uses previously set contour values
\end{tabular}
\end{center}

For contour line plots ({\tt n2} $>$ 0), a zero {\tt n3} value will
suppress numbers near each contour line (same as
{\tt CONT}our). Default: n3 = 0.
\vfill
\eject
