\heada{PLOT}{CONTour}
\hspace{1.0cm}{{cont,n1,n2,n3 \hfill}}
\headb

Plot contours for solution degree of freedom {\tt n1}
(default is 1).
Two options are available to construct contouring:
\begin{enumerate}
\item{
If {\tt n2} is zero or negative, areas between contours
will be shaded in colors or grayscale.  For this case, only
the minimum and maximum contour values are specified - by
default (enter return) the program constructs 7 evenly
spaced intervals for the shading.

If {\tt n3} is positive, plotting of the mesh is suppressed.
If {\tt n3} is negative, plotting
of the mesh is suppressed and the previously existing
contour values are used.  Note the that the contours must
have been already set by a previous call to {\tt CONT}our for this
option to function properly.}
\item{
If {\tt n2} is a positive number specific contour lines may be designated
and plotted as lines. It is necessary to define the value for each of the
{\tt n2} contour lines.
If {\tt n3} is non-zero a numerical label will
be added near each contour indicating the relationship to a
value table given on the screen.}
\end{enumerate}

In interactive mode, after the {\tt cont,n1,n2,n3} command
is given prompts for additional data will appear.  For each
contour line the values to be plotted should be entered
(maximum of 8 items per record).  Maximum and minimum existing values
are indicated on the screen.
For shaded plots only a lower and an upper value separating the smallest
and largest shading from their adjacent ones are input.
\vfill\eject
