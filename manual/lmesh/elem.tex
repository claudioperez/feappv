\heada{MESH}{ELEMent}
\hspace{1.0cm}{{ elem \hfill}}
\hspace{1.4cm}{{ nelm1,ngen1,matl1,(ix(i,nelm1),i=1,nen) \hfill}} \\{\smallskip}
\hspace{1.4cm}{{ nelm2,ngen2,matl2,(ix(i,nelm2),i=1,nen) \hfill}} \\{\smallskip}
\hspace{1.4cm}{{ <etc.,terminate on blank record> \hfill}} \\{\smallskip}
\hspace{1.0cm}{{ elem,old \hfill}} \\{\smallskip}
\hspace{1.4cm}{{ nelm1,matl1,(ix(i,nelm1),i=1,nen),ngen1 \hfill}} \\{\smallskip}
\hspace{1.4cm}{{ nelm2,matl2,(ix(i,nelm2),i=1,nen),ngen2 \hfill}} \\{\smallskip}
\hspace{1.4cm}{{ <etc.,terminate on blank record> \hfill}}
\headb

The {\tt ELEM}ent command is used to specify values of
nodal numbers which are attached to an element.  The command may appear
more than once during mesh inputs.  It may also be combined with {\tt BLOC}k
and {\tt BLEN}d inputs to generate elements in a mesh.  For each
element to be specified by an {\tt ELEM}ent command,
a record is entered with the following information:

\begin{center}
\begin{tabular}{r l}
\it nelm        &-- Number of the element to be specified \\
\it ngen        &-- Value to increment each node-i value \\
                &\quad when generation is used (default = 1). \\
\it matl        &-- Material identifier for the element, \\
                &\quad this will determine the element type. \\
\it ix(1,nelm)  &-- Node-1 number attached to element. \\
\it ix(2,nelm)  &-- Node-2 number attached to element. \\
                &\quad  etc., to {\it nen} nodes. \\
\end{tabular}
\end{center}
Element inputs must be in increasing values for {\it nelm}.
If gaps occur in the input order 
generation is performed, the element number
sequence will be in increments of 1 from {\it nelm1} to {\it nelm2};
the nodes which are generated for each intermediate element
will be as follows:

\begin{verbatim}
       ix(i,nelm1+1) = ix(i,nelm1) + ngen1
\end{verbatim}
except
\begin{verbatim}
       ix(i,nelm1+1) = 0  whenever ix(i,nelm1) = 0}
\end{verbatim}

The program assumes that any zero value of an {\it ix(i,nelm)}
indicates that no node is attached at that point.

Input terminates whenever a blank record is encountered.

ADVICE: When the number of elements on the control record is input as
zero {\sl FEAPpv} attempts to compute the number of elements in the mesh.
The number computed is the largest number input by an {\tt ELEM}ent
input or during a {\tt BLOC}k and {\tt BLEN}d
generation.  During {\tt ELEM}ent input
it is necessary to input the last element in generation sequences.
\vfil\eject
