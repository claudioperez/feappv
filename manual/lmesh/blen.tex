\heada{MESH}{BLENd}
\hspace{1.0cm}{{\quad blen (Surface in 2 or 3-D) \hfill}} \\{\smallskip}
\hspace{1.4cm}{{\quad surf,r-inc,s-inc,[node1,elmt1,mat],b-type \hfill}}
\\{\smallskip}
\hspace{1.8cm}{{\quad (snode(i),i=1,4) \hfill}} \\{\smallskip}
\hspace{1.0cm}{{\quad blen (3-D Solid) \hfill}} \\{\smallskip}
\hspace{1.4cm}{{\quad soli,r-inc,s-inc,t-inc,[node1,elmt1,mat],b-type \hfill}}
\\{\smallskip}
\hspace{1.8cm}{{\quad (snode(i),i=1,8) \hfill}}
\headb

{\sl FEAPpv} can generate patches of a mesh using the {\tt BLEN}ding
function mesh command.  Blending functions are briefly discussed in the
Zienkiewicz \& Taylor finite element book, volume 1 pp 181 ff.  Each
super node is defined by an input of the following information:

The {\tt BLEN}d data input segment may be used to generate:
\begin{enumerate}
\item 4-node quadrilateral elements in 2 or 3-D.
\item 8-node bricks in 3-D.
\end{enumerate}

For surface patches
the nodes and quadrilateral elements
defined by {\tt BLEN}d command is developed from a master
element which is defined by an isoparametric
mapping function in terms of the two natural coordinates,
r (or $\xi_1$) and s (or $\xi_2$), respectively.
The node numbers on the master element of each
patch defined by {\tt BLEN}d are defined from the values of the
four super-nodes used to define the vertices of the blend patch region.
The four vertex super-nodes are numbered in any right-hand rule sequence.
The r-direction ($\xi_1$) is defined along the direction of the first
two super-nodes and the s-direction ($\xi_2$) along the direction of the
first and fourth super-nodes.
The vertex super-nodes are used as the end nodes which define the four edges
of the blend patch. {\sl FEAPpv} searches the list of edges defined by the
the {\tt SIDE} command.  If a match is found it is used as the patch edge.
If no match is found {\sl FEAPpv} will define a straight edge with linear
equal increment interpolation used to define the spacing of
nodes in the finite element mesh.  Care must be used in defining any
specified sides in order to avoid errors from this automatic generation.

For three dimensional solid patches the same technique is used; however, now
it is necessary to define eight vertex super-nodes to define the blend
patch.  The eight nodes are numbered by any right-hand rule sequence.
The r-direction and s-direction are defined in the same way as for the
surface patch.  The third t-direction ${\xi_3}$ is along the direction defined
by the first to fifth vertex super-node.

The r-, s-, and t-increments are used in the same manner as for the {\tt BLOC}k
command.  Care must be used in defining the increments along any direction
which involves a multi-segment interpolation to ensure that the total number
of intervals from the side definition for the mult-segment agrees with the
number of increments specified with the {\tt BLEN}d command.

Examples for two and three dimensional blends are illustrated in the {\sl FEAPpv}
User Manual.

Since the description of the {\tt BLEN}d command depends on existence of 
{\tt SNOD}e and {\tt SIDE} command data, the actual generation of nodes and
elements is deferred until the entire mesh data has been defined.  Thus,
errors may not appear in the output file in the order data was placed in the
input file.
\vfil\eject
