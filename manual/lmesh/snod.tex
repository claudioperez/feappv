\heada{MESH}{SNODes}
\hspace{1.0cm}{{ snod                  \hfill}} \\{\smallskip}
\hspace{1.4cm}{{ snode1,(x(i,snode1),i=1,ndm) \hfill}} \\{\smallskip}
\hspace{1.4cm}{{ snode2,(x(i,snode2),i=1,ndm) \hfill}} \\{\smallskip}
\hspace{1.4cm}{{ <etc.,terminate with blank record> \hfill}}
\headb

The {\tt SNOD}e command is used to specify the values for
nodal coordinates of {\it super nodes}.  Currently, {\sl FEAPpv} uses
super nodes to generate patches of a mesh using the {\it blending
function} option and to determine contact surfaces.
Blending functions are briefly discussed in the
Zienkiewicz \& Taylor finite element book, volume 1 pp 181 ff.  Each
super node is defined by an input with the following information:

\begin{center}
\begin{tabular}{r l}
\it snode     & - Number of super node to be specified \\
\it x(1,snode)& - Value of coordinate in 1-direction \\
\it x(2,snode)& - Value of coordinate in 2-direction \\
             &\quad etc., to 'ndm' directions \\
\end{tabular}
\end{center}
Super nodes must be numbered from 1 to the number needed to describe
the {\it sides} and {\it blend patches}.
The position of each super node is specified in cartesian coordinate components.
No generation is performed for missing node numbers.
Location of all super nodes may be graphically displayed using the
{\tt PLOT,SNOD}e command.

In addition to the supernodes it may be necessay to define the sides of
blend patches using the mesh command {\tt SIDE}.  Also, the mesh command
{\tt BLEN}d must be given for each mesh patch to be created.
\vfil\eject
