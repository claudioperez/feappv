\heada{MESH}{DISPlacements}
\hspace{1.0cm}{{ disp                  \hfill}} \\{\smallskip}
\hspace{1.4cm}{{ node1,ngen1,(d(i,node1),i=1,ndf) \hfill}} \\{\smallskip}
\hspace{1.4cm}{{ node2,ngen2,(d(i,node2),i=1,ndf) \hfill}} \\{\smallskip}
\hspace{1.4cm}{{ <etc.,terminate with blank record> \hfill}}
\headb

The {\tt DISP}lacement command is used to specify the values for
nodal boundary displacements.
For each node to be specified a record is entered with the
following information:

\begin{center}
\begin{tabular}{r l}
\it node      &-- Number of node to be specified \\
\it ngen      &-- Increment to next node, if generation \\
              &\quad is used, otherwise 0. \\
\it d(1,node) &-- Value of displacement for 1-dof \\
\it d(2,node) &-- Value of displacement for 2-dof \\
              &\quad etc., to {\it ndf} directions \\
\end{tabular}
\end{center}
When generation is performed, the node number sequence will be
(for {\it node1-node2} sequence shown at top):

\begin{center}
{\it node1, node1+ngen1, node1+2$\times$ngen1, .... , node2}
\end{center}

The values for each displacement will be a linear
interpolation between the {\it node1} and {\it node2} values for
each degree-of-freedom.

While it is possible to specify both the force and the displacement applied
to a node, only one can be active during a solution step.  The determination
of the active value is determined from the boundary
restraint condition value.  If the boundary restraint value is zero
and you use one of the force-commands a force
value is imposed, whereas, if the boundary restraint value is non-zero
and you use one of the displacement-commands a
displacement value is imposed. (See {\tt BOUN}dary, {\tt CBOU}ndary,
or {\tt EBOU}ndary pages for setting boundary
conditions.).  It is possible to change the type of boundary restraint
during execution by resetting the boundary restraint value. It is not
possible, to specify a displacement by using the combination of a 
force-command with a non-zero boundary restraint value, as it was
in the last releases of {\sl FEAPpv}. For further information see the
{\tt CDIS}placement page.

Displacement conditions may also be specified using the {\tt EDIS}
and {\tt CDIS} commands.
\vfil\eject
