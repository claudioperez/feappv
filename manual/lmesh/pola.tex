\heada{MESH}{POLAr}
\hspace{1.0cm}{{ pola                  \hfill}} \\{\smallskip}
\hspace{1.4cm}{{ node,node1,node2,inc \hfill}} \\{\smallskip}
\hspace{1.4cm}{{ all \hfill}} \\{\smallskip}
\hspace{1.4cm}{{ <terminate with blank record> \hfill}}
\headb

The {\tt POLA}r command may be used to convert  any  coordinates
which have been specified in polar (or cylindrical)
form, to cartesian coordinates.  The conversion is performed
using the following relations:

\begin{center}
\begin{tabular}{l l l}
radius    & = x(1,node) &  -- input value \\
theta     & = x(2,node) &  -- input value in degrees \\
x(1,node) & = $x_0$ + radius $\times$ cos(theta) \\
x(2,node) & = $y_0$ + radius $\times$ sin(theta) \\
x(3,node) & = $z_0$ + x(3,node) &  -- 3-D only \\
\end{tabular}
\end{center}
The values for $x_0$, $y_0$, and $z_0$ are specified using the {\tt SHIF}t
command (default values are zero).
A sequence of nodes may be converted by specifying  non-zero
values  for  {\it node1},  {\it node2}, and {\it inc}.  The sequence generated
will be:

\begin{center}
{\it node1, node1+inc, node1+2$\times$inc,  ...  , node2}
\end{center}

Several records may follow the {\tt POLA}r command.   Execution
terminates with a blank record.

The option {\it all} perform the operation on all currently defined nodes.
\vfil\eject
