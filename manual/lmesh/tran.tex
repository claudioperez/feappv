\heada{MESH}{TRANsformation}
\hspace{1.0cm}{{ tran,<inc>                  \hfill}} \\{\smallskip}
\hspace{1.4cm}{{ $T_{11}$,$T_{12}$,$T_{13}$  \hfill}} \\{\smallskip}
\hspace{1.4cm}{{ $T_{21}$,$T_{22}$,$T_{23}$  \hfill}} \\{\smallskip}
\hspace{1.4cm}{{ $T_{31}$,$T_{32}$,$T_{33}$  \hfill}} \\{\smallskip}
\hspace{1.4cm}{{ $x_0$,$y_0$,$z_0$           \hfill}}
\headb

The {\tt TRAN}sformation commmand defines a coordinate transformation
to be applied to input values. After specification of the command
the input nodal coordinates $\Bf{x}_{input}$ are transformed to global
nodal coordinates, $\Bf{x}$ using
\begin{displaymath}
\B{x} = \B{T} \, \B{x}_{input} + \B{x}_0
\end{displaymath}
Thus the $\Bf{x}$ correspond to the nodal values after applying
the transformation and become the values used for the analysis.

Two options exist to define the transformation: (a) a direct specification of
the translation and rotation transformation arrays; (b) an incremental
specification of the arrays. 

\noindent{\bf{Example: Direct specification TRANsformation}}

A rectangular block of nodes and elements of size $20 \times 20$ units
is to be generated in two dimensions
in a rotated coordinate frame ($30^o$ relative to $x_1$ axis).  The commands
may be given as
\begin{verbatim}
       TRANsform
         cosd(30)  sind(30)  0
        -sind(30)  cosd(30)  0
            0         0      1
            0         0      0

       BLOCk
         CARTesian n1 n2
           1  0  0
           2 20  0
           3 20 10
           4  0 10
\end{verbatim}
After the generation it is best to enter an identity transformation 
to prevent any spurrious later effects.  That is enter the set
\begin{verbatim}
       TRANsform
            1  0  0
            0  1  0
            0  0  1
            0  0  0
\end{verbatim}
before ending the mesh generations.

\noindent{\bf{Example: Incremental specification TRANsformation}}

Another block of elements may be input in which the transformation
is given as:
\begin{eqnarray*}
  \B{T}_{new} &=& \B{T}_{inc} \, \B{T}_{old} \\
  \B{x}_{0,new} &=& \B{T}_{inc} \, \B{x}_{0,old} + \B{x}_{inc}
\end{eqnarray*}
In this case the new coordinates are given as:
\begin{displaymath}
\B{x} = \B{T}_{new} \, \B{x}_{input} + \B{x}_{0,new}
\end{displaymath}
Thus specification of a second block of nodes as:
\begin{verbatim}
       TRANsform,INCrement
         cosd(30)  sind(30)  0
        -sind(30)  cosd(30)  0
            0         0      1
            0         0      0

       BLOCk
         CARTesian n1 n2
           1  0  0
           2 20  0
           3 20 10
           4  0 10
\end{verbatim}
after the first block given above will create a block rotated by $60^o$.
This option may be used very conveniently with \texttt{LOOP-NEXT} commands to
replicate a block of nodes and elements, each rotated and/or translated by
an equal incremental amount.
\vfil\eject
