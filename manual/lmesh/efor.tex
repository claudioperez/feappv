\heada{MESH}{EFORce}
\hspace{1.0cm}{{ efor,[set,add]                 \hfill}} \\{\smallskip}
\hspace{1.4cm}{{ i-coor,xi-value,(f(j),j=1,ndf) \hfill}} \\{\smallskip}
\hspace{1.4cm}{{ <etc.,terminate with a blank record> \hfill}}
\headb

The values of boundary force conditions may be set
along any set of nodes which has a constant value of the
{\it i-coordinate direction} (e.g., 1-direction (or x), 2-direction
(or y), etc.).  The data to be supplied during the
definition of the mesh consists of:

\begin{center}
\begin{tabular}{r l}
\it i-coor   &-- Direction of coordinate (i.e., 1 = x, 2 = y, etc.) \\
\it xi-value &-- Value of i-direction coordinate to be used during  \\
             &\quad search (a tolerance of 1/1000 of mesh size is used  \\
             &\quad during search, any coordinate within the gap is  \\
             &\quad assumed to have the specified value).  \\
\it f(1)     &-- Value of force for dof's   \\
\it f(2)     &   \\
\it ...      &   \\
\it f(ndf)   &   \\
\end{tabular}
\end{center}
While it is possible to specify both the force and the displacement applied
to a node, only one can be active during a solution step.  The determination
of the active value is determined from the boundary
restraint condition value.  If the boundary restraint value is zero
and you use one of the force-commands a force
value is imposed, whereas, if the boundary restraint value is non-zero
and you use one of the displacement-commands a
displacement value is imposed. (See {\tt BOUN}dary, {\tt CBOU}ndary,
or {\tt EBOU}ndary pages for setting boundary
conditions.).  It is possible to change the type of boundary restraint
during execution by resetting the boundary restraint value.

The {\tt EFOR} command may be used with two options.  Using the {\tt EFOR,SET}
option replaces previously defined forces at a node by the
pattern specified.  Using the {\tt EFOR,ADD} option
accumulates the forces with previously defined values.
The default mode is {\tt ADD}.

Force conditions may also be specified using the {\tt FORC}e and {\tt CFOR}ce
commands.  The data is order dependent with data
defined by {\tt FORC}e processed first, {\tt EFOR}ce processed second and
the {\tt CFOR}ce data processed last.  The value defined last is used for
any analysis.
\pagebreak

\noindent{\bf{Example: EFORce}}

All the nodes located on the $x_3 = z = 10$ plane are to have a common specified
horizontal
force value. (Note that this is not a common case as end nodes on equally
spaced intervals would have different values from other nodes.)
This may be specified using the command set:
\begin{verbatim}
       EFORce
         3 10.0 -12.5
\end{verbatim}
where -12.5 is the value of each force
\vfil\eject
