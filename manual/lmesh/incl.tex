\heada{MESH}{INCLude}
\hspace{1cm}{{incl,filename \hfill}}
\headb

The {\tt INCL}ude command may be used to access data contained
in a file called {\it filename}.  This permits the data to be
separated into groups which may be combined to form the problem
data.  Thus, if all the coordinate numerical data is in a file
called {\tt COOR.DAT} it may be combined into the mesh by using the
command sequence:

\begin{verbatim}
        COORdinates
          INCLude,COOR.DAT
                 !blank terminator}
\end{verbatim}

This is particularly useful when data is generated by another program.

Another use is for cases in which multiple executions are to be performed
using a different value for some parameter.  Placing the problem data
in a file named {\tt Example.prb} (without the definition for the parameter)
and using the sequence:
\begin{verbatim}
       PARAmeter
         n=2
                !blank terminator
       INCLude,Example.prb
                !blank terminator
       PARAmeter
         n=4
                !blank terminator
       INCLude,Example.prb
                !blank terminator}
\end{verbatim}
permits two executions for different values of the parameter {\it n}.
\vfil\eject
