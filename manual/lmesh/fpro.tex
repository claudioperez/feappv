\heada{MESH}{FPROportional factors}
\hspace{1.0cm}{{ fpro                  \hfill}} \\{\smallskip}
\hspace{1.4cm}{{ node1,ng1,(pnum(i,node1),i=1,ndf) \hfill}} \\{\smallskip}
\hspace{1.4cm}{{ node2,ng2,(pnum(i,node2),i=1,ndf) \hfill}} \\{\smallskip}
\hspace{1.4cm}{{ <etc.,terminate with blank record> \hfill}}
\headb

The {\tt FPRO}portional factors command is used to specify the proportional
load numbers for forced nodal conditions.  For each node a
record is entered with the following information:

\begin{center}
\begin{tabular}{r l}
\it node         &-- Number of node \\
\it ng           &-- Generator increment to next node \\
\it pnum(1,node) &-- Proportional load number of dof-1 \\
\it pnum(2,node) &-- Proportional load number of dof-2 \\
                 &\quad etc., to {\it ndf} directions \\
\end{tabular}
\end{center}
The proportional load numbers are interpretted as follows:

\begin{center}
\begin{tabular}{r l}
\it pnum(i,node)  =  0  & dof-i uses sum of specified proportional \\
                        & load factors \\
\it pnum(i,node) not 0  & dof-i uses specified proportional load \\
                        & based on order of solution inputs \\
                        & prop (default = 1. if prop not used). \\
\end{tabular}
\end{center}
As a default all {\it pnum} values are set to zero (0) and individual
proportional load factors to 1.

Generation is performed similar to {\tt FORCe} input.  Thus
\begin{verbatim}
      FPROportional
         1 5 0 1
        21 0 1 2

\end{verbatim}
would generate nodes {\tt 6, 11, 16} with proportional load number {\tt 1}
assigned to the second degree of freedom; node {\tt 21} would have
proportional load {\tt 1} for the first degree of freedom and {\tt 2} for
the second degree of freedom.

Proportional loading numbers may also be specified using the {\tt EPRO}p
and {\tt CPRO}p commands.  The data is order dependent with data
defined by {\tt FPRO}p processed first, {\tt EPRO}p processed second and
the {\tt CPRO}p data processed last.  The value defined last is used for
any analysis.
\vfil\eject
