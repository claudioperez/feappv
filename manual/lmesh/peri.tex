\heada{MESH}{PERIodic}
\hspace{1cm}{{peri,$<$ther,cauc,ither,icau,    $>$ \hfill}}
\headb

\index{Mesh command!PERIodic}

For some problems in stress analysis of solids
it is desirable to specify non-zero nodal displacements
that result from applying a non-zero displacement gradient.  For such situations
the displacement of the node is computed from
\begin{displaymath}
\B{u}_a = \B{G} \, \B{x}_a
\end{displaymath}
where $\B{u}_a$ and $\B{x}_a$ are the displacement and coordinate of node $a$;
and $\B{G}$ is a \textit{specified} gradient computed
\begin{displaymath}
\B{G} = \begin{bmatrix}
\dpdif{u_1}{x_1} & \dpdif{u_1}{x_2} & \dpdif{u_1}{x_3} \\
\dpdif{u_2}{x_1} & \dpdif{u_2}{x_2} & \dpdif{u_2}{x_3} \\
\dpdif{u_3}{x_1} & \dpdif{u_3}{x_2} & \dpdif{u_3}{x_3} \end{bmatrix}
= \begin{bmatrix}
G_{11} & G_{12} & G_{13} \\[+6pt]
G_{21} & G_{22} & G_{23} \\[+6pt]
G_{31} & G_{32} & G_{33} \end{bmatrix}
\end{displaymath}
The displacement gradient is input using the commands
\begin{verbatim}
       PERIodic <CAUChy ,     >
         G_11    G_12    G_13
         G_21    G_22    G_23
         G_31    G_32    G_33
\end{verbatim}

Another useful application is to impose \textit{periodic} boundary conditions
in which the constraint for two nodes is given as
\begin{displaymath}
u_i^+ = u_i^- + \sum_{j=1}^d G_ij ( x_j^+ - x_j^-)
~~\mbox{with}~~ x_j^+ = x_j^- ~~ j \ne i
\end{displaymath}
where $d$ is the mesh dimension.  Such periodic cases are useful in evaluating
the behavior of micro-scale models under specified strain histories.  Generally,
the constraint can only be used for meshes in which the geometry is rectangular and the boundary nodes satisfy the above constraint.  The boundary
constraints for the problem can then be specified by specifying fixed
displacement conditions for the corner nodes and \texttt{ELINk} conditions
for the parallel boundaries (see Sec. \ref{link}).

A similar behavior may be imposed for thermal problems where temperature $T$
has the condition
\begin{displaymath}
T^+ = T^- + \sum_{j=1}^d  G_j ( x_j^+ - x_j^-)
~~\mbox{with}~~ x_j^+ = x_j^- ~~ j \ne i
\end{displaymath}
in which $G_j = \partial T/\partial x_j$ is a constant thermal gradientl..
The input is specified as
\begin{verbatim}
       PERIodic THERmal
         G_1    G_2    G_3
\end{verbatim}

It is also possible to solve the problem using an incremental form in which
temperature, strain and deformation gradient are defined by
\begin{displaymath}
\nabla T = \pdif{T}{\B{x}} + \bar{G}
\end{displaymath}
\begin{displaymath}
\bepsilon = \left( \dpdif{\B{u}}{\B{x}} \right)^{(s)} + \bar{\B{G}}^{(s)}
\end{displaymath}
and
\begin{displaymath}
\B{F} = \pdif{\B{u}}{\B{x}} + \bar{\B{G}}
\end{displaymath}
respectively. In the above the barred quantities are the imposed gradients.

The above are input as
\begin{verbatim}
       PERIodic ITHErmal
         G_1    G_2    G_3
\end{verbatim}
or
\begin{verbatim}
       PERIodic ICAUchy
         G_11    G_12    G_13
         G_21    G_22    G_23
         G_31    G_32    G_33
\end{verbatim}
In the latter, the distinction between small and finite deformation is
based on the constitutive model used.

\vfil\eject
