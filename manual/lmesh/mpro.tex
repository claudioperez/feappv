\heada{MESH}{MPROportional factors}
\hspace{1.0cm}{{ fpro                  \hfill}} \\{\smallskip}
\hspace{1.4cm}{{ node1,ng1,(pnum(i,node1),i=1,ndf) \hfill}} \\{\smallskip}
\hspace{1.4cm}{{ node2,ng1,(pnum(i,node2),i=1,ndf) \hfill}} \\{\smallskip}
\hspace{1.4cm}{{ <etc.,terminate with blank record> \hfill}}
\headb

The {\tt MPRO}portional factors command is used to specify the proportional
load numbers for nodal mass conditions specified by {\tt MASS} inputs.
For each node a record is entered with the following information:

\begin{center}
\begin{tabular}{r l}
\it node         &-- Number of node \\
\it ng           &-- Generator increment to next node \\
\it pnum(1,node) &-- Proportional load number of dof-1 \\
\it pnum(2,node) &-- Proportional load number of dof-2 \\
                 &\quad etc., to {\it ndf} directions \\
\end{tabular}
\end{center}
The proportional load numbers are interpretted as follows:

\begin{center}
\begin{tabular}{r l}
\it pnum(i,node)  =  0  & dof-i uses sum of specified proportional \\
                        & load factors \\
\it pnum(i,node) not 0  & dof-i uses specified proportional load \\
                        & based on order of solution inputs \\
                        & prop (default = 1. if prop not used). \\
\end{tabular}
\end{center}
As a default all {\it pnum} values are set to zero (0) and individual
proportional load factors to 1.

Generation is performed similar to {\tt FORCe} input.  Thus
\begin{verbatim}
      MPROportional
         1 5 0 1
        21 0 1 2

\end{verbatim}
would generate nodes {\tt 6, 11, 16} with proportional load number {\tt 1}
assigned to the second degree of freedom; node {\tt 21} would have
proportional load {\tt 1} for the first degree of freedom and {\tt 2} for
the second degree of freedom.

In addition to specifying the mass proportional load numbers it is necessary
to specify a factor to multiply the time values.  This is given as a
{\tt GROUnd} acceleration option using a {\tt GLOBal} command.  For example,
the commands:
\begin{verbatim}
      GLOBAL
        GROUND factor v-1 v-2 ... v-ndf
\end{verbatim}
speifies the factors {\tt v-i} for each discrete mass degree of freedom.
Using this option it is possible to specify different acceleration effects
for the various degrees of freedom.  Similar options exist for many of the
elements included with {\sl FEAPpv} and for these it is necessary also to
include the {\tt GROUnd} record with each affected material set.
\vfil\eject
