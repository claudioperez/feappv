\heada{MESH}{COORdinates}
\hspace{1.0cm}{{ coor                  \hfill}} \\{\smallskip}
\hspace{1.4cm}{{ node1,ngen1,(x(i,node1),i=1,ndm) \hfill}} \\{\smallskip}
\hspace{1.4cm}{{ node2,ngen2,(x(i,node2),i=1,ndm) \hfill}} \\{\smallskip}
\hspace{1.4cm}{{ <etc.,terminate with blank record> \hfill}}
\headb

The {\tt COOR}dinate command is used to specify the values for
nodal coordinates.  For each node to be specified a record
is entered with the following information:

\begin{center}
\begin{tabular}{r l}
\it node     &-- Number of node to be specified \\
\it ngen     &-- Increment to next node, if generation \\
             &\quad is used, otherwise 0. \\
\it x(1,node)&-- Value of coordinate in 1-direction \\
\it x(2,node)&-- Value of coordinate in 2-direction \\
             &\quad etc., to 'ndm' directions \\
\end{tabular}
\end{center}
When generation is performed, the node number sequence
will be (for {\it node1-node2} sequence shown above):

\begin{center}
{\it node1, node1+ngen1, node1+2$\times$ngen1, .... , node2}
\end{center}

The values generated for each coordinate will be a linear interpolation
between {\it node1} and {\it node2}.

The {\tt COOR}dinate values may be input in a polar or spherical
coordinate system and converted to cartesian values later
using the {\tt POLA}r or {\tt SPHE}rical commands.

Nodal coordinates may also be generated using the {\tt BLOC}k and the
{\tt BLEN}d commands.

\noindent{\bf{Example: COORdinate}}

The set of commands:
\begin{verbatim}
       COORdinates
         1 1  0.0  0.0
        11 0 10.0  5.0

\end{verbatim}
will generate 11 nodes equally spaced along the straight line connecting the
points (0, 0) and (10, 5).  The nodes will be numbered from 1 to 11.
\vfil\eject

