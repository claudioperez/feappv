\heada{MESH}{BOUNdary}
\hspace{1.0cm} {{ boun \hfill}} \\{\smallskip}
\hspace{1.4cm} {{ node1,ngen1,(id(i,node1),i=1,ndf) \hfill}} \\{\smallskip}
\hspace{1.4cm} {{ node2,ngen2,(id(i,node2),i=1,ndf) \hfill}} \\{\smallskip}
\hspace{1.4cm} {{ $<$etc.,terminate with blank record$>$ \hfill}}
\headb

The {\tt BOUN}dary command is used to specify the values for
the boundary restraint conditions.  For each node to be
specified a record is entered with the following information:

\begin{center}
\begin{tabular}{r l}
\it node      &-- the number of the node to be specified \\
\it ngen      &-- the increment to the next node, if \\
              &  \quad generation is used, otherwise 0. \\
\it id(1,node)&-- value of 1-dof boundary restraint for {\it node} \\
\it id(2,node)&-- value of 2-dof boundary restraint for {\it node} \\
              &  \quad etc., to {\it ndf} direction.
\end{tabular}
\end{center}
The boundary restraint codes are interpretted as follows:

\begin{center}
\begin{tabular}{r l}
\it id(i,node)$~= ~0$   & a force will be an applied load to dof (default). \\
\it id(i,node)$~\ne ~0$ & a displacement will be imposed to dof.
\end{tabular}
\end{center}
When generation is performed, the node number sequence
will be (for {\it node1-node2} sequence shown at top):

\begin{center}
{\it node1, node1+ngen1, node1+2$\times$ngen1, .... , node2}
\end{center}
The values for each boundary restraint will be as follows:

\begin{center}
\begin{tabular}{rc l}
\it id(i,node1) =& 0 or positive &$\rightarrow$ id(i,node1+ngen1) =
\phantom{$-$}0 \\
\it id(i,node1) =& negative      &$\rightarrow$ id(i,node1+ngen1) = $-1$
\end{tabular}
\end{center}
With this convention the value of a zero {\it id(i,node2)} will be set
negative whenever the value of {\it id(i,node1)} starts negative.
Accordingly, it is necessary to assign a positive value for
the restraint code to terminate a generation sequence (e.g.,
when it is no longer desired to set a dof to be restrained).
Alternatively, an i-dof may be eliminated for all nodes by
using the generation sequence:

\begin{center}
\begin{tabular}{c|c|ccccc}
node  & ngen &   &     & dofs &     &  \\
      &      & 1 & ... &  i   & ... & ndf \\ \hline
1     & 1    & 0 & ... & -1   & ... & 0 \\
numnp & 0    & 0 & ... & +1   & ... & 0 
\end{tabular}
\end{center}
Subsequent records may then be used to assign values to
other degree-of-freedoms.

Boundary condition restraints may also be specified using the {\tt EBOU}nd
or {\tt CBOU}nd commands.

\noindent{\bf{Example: BOUNdary}}

Consider a problem which has 3 degrees of freedom at each node.  The
sequence of records:
\begin{verbatim}
       BOUNdary conditions
         1 4 1 -1 0
        13 0 0  1 1

\end{verbatim}
will define boundary conditions for nodes 1, 5, 9 and 13 and the restraint
codes will have the following values
\begin{center}
\begin{tabular}{r|ccc}
node  &    & DOFS & \\
      &  1 &  2 & 3  \\ \hline
 1    &  1 & -1 & 0 \\
 5    &  0 & -1 & 0 \\
 9    &  0 & -1 & 0 \\
13    &  0 &  1 & 1 \\
\end{tabular}
\end{center}

Any degree of freedom with a non-zero bounday code will be \textit{restrained},
whereas a degree of freedom with a zero boundary code will be
\textit{unrestrained}.  Restrained degrees of freedom may have specified
non-zero (generalized) \textit{displacements} whereas unrestrained ones may
have specified non-zero (generalized) \textit{forces}.
\vfil\eject
