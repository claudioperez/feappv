\heada{MESH}{EANGle}
\hspace{1.0cm}{{ eang                  \hfill}} \\{\smallskip}
\hspace{1.4cm}{{ i-coor,xi-value,angle \hfill}} \\{\smallskip}
\hspace{1.4cm}{{ <etc.,terminate with a blank record> \hfill}}
\headb

The sloping boundary condition angle may be set along
any set of nodes which has a constant value of the {\it i-coordinate
direction} (e.g., 1-direction (or x), 2-direction
(or  y),  etc.).  The data to be supplied during the definition of
the mesh consists of:

\begin{center}
\begin{tabular}{r l}
\it i-coor   &-- Direction of coordinate (i.e., 1 = x, 2 = y, etc.) \\
\it xi-value &-- Value of i-direction coordinate to be used during  \\
             &\quad search (a tolerance of 1/1000 of mesh size is used  \\
             &\quad during search, any coordinate within the gap is  \\
             &\quad assumed to have the specified value).  \\
\it angle    &-- Value 1-direction makes with x1-direction in degrees. \\
\end{tabular}
\end{center}
For nodes with sloping conditions, the degrees-of-freedom
are expressed with respect to the rotated frame 1-2 instead
of the global frame x1-x2 (x-y).  For three dimensional problem
the 3-direction coincides with the x3-direction (z).

Angle conditions may also be specified using the {\tt EANG}le and {\tt CANG}le
commands.  The data is order dependent with data
defined by {\tt ANGL}e processed first, {\tt EANG}le processed second and
the {\tt CANG}le data processed last.  The value defined last is used for
any analysis.

\noindent{\bf{Example: EANGle}}

All the nodes located on the $x_3 = z = 0$ plane are to have degrees of freedom
specified relative to a rotated coordinate system (about the $x_3$-axis).  This
is not a common case but may be specified using the command set:
\begin{verbatim}
       EANGle
         3 0.0 40.0
\end{verbatim}
where 40.0 is the angle (in degrees) of the rotation.  Rotation is defined by
right-hand screw rule.
\vfil\eject
