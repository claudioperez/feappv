\heada{MESH}{TEMPerature}
\hspace{1.0cm}{{ temp                 \hfill}} \\{\smallskip}
\hspace{1.4cm}{{ node1,ngen1,t(node1) \hfill}} \\{\smallskip}
\hspace{1.4cm}{{ node2,ngen2,t(node2) \hfill}} \\{\smallskip}
\hspace{1.4cm}{{ <etc.,terminate with blank record> \hfill}}
\headb

The {\tt TEMP}erature command is used to specify the values for
nodal temperatures.  For each node to be specified a record
is entered with the following information:

\begin{center}
\begin{tabular}{r l}
\it node    &-- Number of the node to be specified \\
\it ngen    &-- Increment to the next node, if generation \\
            &\quad is used, otherwise 0. \\
\it t(node) &-- Value of temperature for node. \\
\end{tabular}
\end{center}
When generation is performed, the node number  sequence
will be (for {\it node1-node2} sequence shown at top):

\begin{center}
{\it node1, node1+ngen1, node1+2$\times$ngen1, .... , node2 }
\end{center}

The values for each temperature will be a linear  interpolation
between {\it node1} and {\it node2}. 
\vfil\eject
