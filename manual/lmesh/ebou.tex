\heada{MESH}{EBOUndary}
\hspace{1.0cm}{{ ebou,[set,add]        \hfill}} \\{\smallskip}
\hspace{1.4cm}{{ i-coor,xi-value,(ibc(j),j=1,ndf) \hfill}} \\{\smallskip}
\hspace{1.4cm}{{ <etc.,terminate with a blank record> \hfill}}
\headb

The boundary restraint conditions may be set along any
set of nodes which has a constant value of the {\it i-coordinate
direction} (e.g., 1-direction (or x), 2-direction (or y),
etc.).  The data to be supplied during the definition of the
mesh consists of:

\begin{center}
\begin{tabular}{r l}
\it i-coor   &-- Direction of coordinate (i.e., 1 = x, 2 = y, etc.) \\
\it xi-value &-- Value of i-direction coordinate to be used during  \\
             &\quad search (a tolerance of 1/1000 of mesh size is used  \\
             &\quad during search, any coordinate within the gap is  \\
             &\quad assumed to have the specified value).  \\
\it ibc(1)   &-- Restraint conditions for all nodes with value of \\
\it ibc(2)   &\quad search.(0 = boundary code remains as previously set \\
\it ...      &\quad $>0$ denotes a fixed dof, $<0$ resets previously \\
\it ibc(ndf) &\quad defined boundary codes to 0.) \\
\end{tabular}
\end{center}

The {\tt EBOU} command may be used with two options.  Using the {\tt EBOU,SET}
option replaces previously defined conditions at any node by the
pattern specified.  Using the {\tt EBOU,ADD} option
accumulates the specified boundary conditions with previously defined
restraints.  The default mode is {\tt ADD}.
Boundary restraint conditions may also be specified using the {\tt BOUN}
and {\tt CBOU} commands.
The data is order dependent with data
defined by {\tt DISP} processed first, {\tt EDIS} processed second and
the {\tt CDIS} data processed last.  The value defined last is used for
any analysis.

\noindent{\bf{Example: EBOUndary}}

All the nodes located on the $x_3 = z = 0$ plane are to have
restraints on the $3^{rd}$ and $6^{th}$ degrees of freedom.
This may be specified using the command set:
\begin{verbatim}
       EBOUndaray
         3 0.0 0 0 1 0 0 1
\end{verbatim}
where non-zero values indicate a \textit{restrained} degree of freedom and
a \textit{zero} an unrestrained degree of freedom.  Non-zero displacements
may be specified for restrained dof's and non-zero forces for unrestrained
dof's.
\vfil\eject
