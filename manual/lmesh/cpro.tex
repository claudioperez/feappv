\heada{MESH}{CPROp}
\hspace{1.0cm}{{ cpro                  \hfill}}    \\{\smallskip}
\hspace{1.4cm}{{ node,(x(i),i=1,ndm),(pnum(i),i=1,ndf) \hfill}}\\{\smallskip}
\hspace{1.4cm}{{ linear (pnum(i),i=1,ndf) \hfill}} \\{\smallskip}
\hspace{1.8cm}{{ 1,x1,y1    \hfill}} \\{\smallskip}
\hspace{1.8cm}{{ 2,x2,y2    \hfill}} \\{\smallskip}
\hspace{1.4cm}{{ quadratic (pnum(i),i=1,ndf) \hfill}} \\{\smallskip}
\hspace{1.8cm}{{ 1,x1,y1    \hfill}} \\{\smallskip}
\hspace{1.8cm}{{ 2,x2,y2    \hfill}} \\{\smallskip}
\hspace{1.8cm}{{ 3,x3,y3    \hfill}} \\{\smallskip}
\hspace{1.0cm}{{ surface (pnum(i),i=1,ndf) \hfill}} \\{\smallskip}
\hspace{1.8cm}{{ 1,x1,y1,z1 \hfill}} \\{\smallskip}
\hspace{1.8cm}{{ 2,x2,y2,z2 \hfill}} \\{\smallskip}
\hspace{1.8cm}{{ 3,x3,y3,z3 \hfill}} \\{\smallskip}
\hspace{1.8cm}{{ 4,x4,y4,z4 \hfill}} \\{\smallskip}
\hspace{1.4cm}{{ cartesian  \hfill}} \\{\smallskip}
\hspace{1.4cm}{{ pola,x0,y0 \hfill}} \\{\smallskip}
\hspace{1.4cm}{{ gap,value \hfill}} \\{\smallskip}
\hspace{1.4cm}{{ <etc.,terminate with a blank record> \hfill}}
\headb

The proportional loading number to be appled to nodal forces and displacments
may be input using this command.
The input values are saved in a file(s) and searched
after the entire mesh is specified.
The data is order dependent with data
defined by {\tt FPRO}p processed first, {\tt EPRO}p processed second and
the {\tt CPRO}p data processed last.  The value defined last is used for
any analysis.

For a single {\it node}, the data to be supplied during
the definition of the mesh consists of:

\begin{center}
\begin{tabular}{r l}
\it node   &-- Defines inputs to be for a {\it node} \\
\it x(1)   &-- Value of coordinates to be used during search \\
\it ...    &\quad (a tolerance of about 1/1000 of mesh size is \\
\it x(ndm) &\quad used during search, coordinate with smallest \\
           &\quad distance within tolerance is assumed to have \\
           &\quad specified value). \\
\it pnum(1,node) &-- Proportional load number of dof-1 \\
\it pnum(2,node) &-- Proportional load number of dof-2 \\
                 &\quad etc., to {\it ndf} directions \\
\end{tabular}
\end{center}
At execution, the node(s) within the tolerance will have
their values set to the proportional load numbers given.

For two dimensional problems it is possible to
specify a segment to which the proportional load numbers are to be
applied.  The segment may be specified as a {\it linear}
or a {\it quadratic} line. For the {\it linear} segment the angle
is given together with the coordinates of the ends.
These are specified as:

\begin{verbatim}
       LINEar (pnum(i),i=1,ndf)
         1,x1,y1
         2,x2,y2}
\end{verbatim}

For {\it quadratic} segments the ends {\it (x1,y1)} and {\it (x2,y2)}
together with an intermediate point {\it (x3,y3)} are used.
The quadratic segment is given as:

\begin{verbatim}
       QUADratic (pnum(i),i=1,ndf)
         1,x1,y1
         2,x2,y2
         3,x3,y3}
\end{verbatim}

For three dimensional problems it is possible to
specify the segment to which the proportional load numbers
are applied.  The segment is specified as a {\it surface}.
The data is specified as:

\begin{verbatim}
       SURFace (pnum(i),i=1,ndf)
         1,x1,y1,z1
         2,x2,y2,z2
         3,x3,y3,z3
         4,x4,y4,z4}
\end{verbatim}

The program assigns a search region and attempts to
find the elements and the nodes to which the specified
segments are associated.  It is possible that no segment
is located (an error message will appear in the output file).  To
expand the search region a {\it gap} can be specified as:

\begin{verbatim}
       GAP,value}
\end{verbatim}
The {\it gap-value} is a coordinate distance within which
nodes are assumed to lie on the specified segment. The
value should be less than dimensions of typical elements
or erroneous nodes will be found by the search.
It is suggested that the computed boundary conditions
be checked graphically to ensure that they are
correctly identified (e.g., use {\tt PLOT,MESH} and {\tt PLOT,BOUN}dary
to show the locations of conditions).

The {\it polar} option may be used to set the origin {\it (x0,y0)} of a
polar coordinate system. Coordinates entered after
{\it polar} will be assumed to be radius and angle.  The
{\it cartesian} option resets the coordinate system to a
cartesian frame.

\noindent{\bf{Example: CFORce}}

In a two dimensional problem it is desired to have a time variation for
the force applied to the node nearest to the coordinates $x_1 = 10$ and
$x_2 = 5$ which is different in the two directions.  To prescribe the data
it is necessary to define three different command sets.  The first defines
the \textit{magnitude} of the two forces at the node.  This may be given
as:
\begin{verbatim}
       CFORce
         NODE 10 5  8.5  -6.25

\end{verbatim}
in which $F_1 = 8.5$ and $F_2 = -6.25$.  The second command set describes the
\textit{numbers} for proportional loading factors which will multiply each of
the forces.  These may be given as:
\begin{verbatim}
       CPROportional
         NODE 10 5  2  3

\end{verbatim}
where $2$ is the proportional loading number 2 and $3$ that for 3.  Finally,
during solution mode the proportional loads must be given.  This is best
included in a \texttt{BATCh} solution mode as:
\begin{verbatim}
       BATCh
         PROP,,2
       END
        data for proportional load 2 (see PROP in solution commands)
\end{verbatim}
and
\begin{verbatim}
       BATCh
         PROP,,3
       END
        data for proportional load 3 (see PROP in solution commands)
\end{verbatim}
Failure to specify correctly any of the above will usually result in an error.
\vfil\eject
