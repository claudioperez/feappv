\heada{MESH}{CDISplacement}
\hspace{1.0cm}{{ cdis,[set,add]        \hfill}} \\{\smallskip}
\hspace{1.4cm}{{ gap,value \hfill}} \\{\smallskip}
\hspace{1.4cm}{{ node,(x(i),i=1,ndm),(d(j),j=1,ndf) \hfill}} \\{\smallskip}
\hspace{1.4cm}{{ <etc.,terminate with a blank record> \hfill}}
\headb

The specified displacement boundary conditions may be set
using the reference coordinates for a {\it node}.  The input values are
saved in files and searched after the entire mesh is
specified.  After use files are deleted.
The data is order dependent with data
defined by {\tt DISP} processed first, {\tt EDIS} processed second and
the {\tt CDIS} data processed third and data specified by the {\tt CSUR}f
processed last.  The value defined last is used for any analysis.

The {\tt CDIS} command may be used with two options.  Using the {\tt CDIS,SET}
option replaces all previously defined values at any node by the
pattern specified. This is the default mode.  Using the {\tt CDIS,ADD} option
accumulates the specified value with previously defined values.

For a {\it node}, the data to be supplied during
the definition of the mesh consists of:

\begin{center}
\begin{tabular}{r l}
\it node   &-- Defines inputs to be for a {\it node}. \\
\it x(1)   &-- value of coordinates to be used during search \\
\it ...    &\quad (a gap of 1/1000 of mesh size is used during \\
\it x(ndm) &\quad search, coordinate with smallest distance within \\
           &\quad gap is assumed to have specified value). \\
\it d(1)   &-- displacement on 1-dof \\
\it d(2)   &-- displacement on 2-dof \\
\it ...    & \\
\it d(ndf) &
\end{tabular}
\end{center}
To expand the search region a {\it gap-value} can be specified as:

\begin{verbatim}
       GAP,value}
\end{verbatim}
The {\it gap-value} is a coordinate distance within which
nodes are assumed to lie on the specified segment.  The
value should be less than dimensions of typical elements
or erroneous nodes will be found by the search.
It is suggested that the computed boundary conditions
be checked graphically to ensure that they are
correctly identified (e.g., use {\tt PLOT,MESH} and {\tt PLOT,BOUN}
to show the locations of conditions).

While it is possible to specify both the force and the displacement applied
to a node, only one can be active during a solution step.  The determination
of the active value is determined from the boundary
restraint condition value.  If the boundary restraint value is zero
and you use one of the force-commands a force
value is imposed, whereas, if the boundary restraint value is non-zero
and you use one of the displacement-commands a
displacement value is imposed. (See {\tt BOUN}dary, {\tt CBOU}ndary,
or {\tt EBOU}ndary pages for setting boundary
conditions.).  It is possible to change the type of boundary restraint
during execution by resetting the boundary restraint value. It is not
possible, to specify a displacement by using the combination of a 
force-command with a non-zero boundary restraint value, as it was
in the last releases of FEAPpv.

Only those values of the {\tt CDIS}placement-command are regarded whose
directions have a non-zero boundary restraint value. All other displacement
values are variable.

For Example:

\begin{verbatim}
       cang
         node,1.0,1.0,30.0
                  ! end with blank record
       cbou
         node,1.0,1.0,0,1
                  ! end with blank record
       cdis
         node,1.0,1.0,0.1,0.1
                  ! end with blank record
\end{verbatim}
Here the first displacement value is not considered. There is a
displacement of the node with the coordinates (1.0,1.0). The direction
of the displacement is 120 degrees and the value is 0.1. The displacement
in the 30 degree direction is variable.
\vfil\eject

