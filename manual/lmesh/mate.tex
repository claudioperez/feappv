\heada{MESH}{MATErial}
\hspace{1.0cm}{{ mate,ma,<output label> \hfill}} \\{\smallskip}
\hspace{1.4cm}{{ type,iel,<id,(idf(i),i=1,ndf)> \hfill}} \\{\smallskip}
\hspace{1.4cm}{{ <parameters element type> \hfill}}
\headb

The {\tt MATE}rial set command is used to specify the parameters
for each unique material set number {\it ma} in the analysis, as well as to
specify the element type associated with the material set parameters.

The parameter {\it type} denotes the element formulation to be employed.
{\sl FEAPpv} includes a library of elements for thermo-mechanical analyses.
The included types are:

\begin{center}
\begin{tabular}{l l}
\it SOLId   & Continuum solid mechanics element (2 or 3-D). \\
\it THERmal & Continuum thermal element (2 or 3-D). \\
\it FRAMe   & 2-Node frame element (2 or 3-D). \\
\it TRUSs   & 2-Node truss element (1, 2, or 3-D). \\
\it PLATe   & 2-d Plate bending element. \\
\it SHELl   & 3-d Shell element. \\
\it MEMBrane & 3-d Membrane element. \\
\it GAP     & n-d Gap element. \\
\it PRESsure & 3-d Pressure load element (dead or follower). \\
\end{tabular}
\end{center}
Users
may also add their own elements and access by setting {\it type} to {\tt USER}
and the parameter {\it iel} to the number of the element module added (between
1 and 50).

The parameter {\it id} is the material identifier.  Defined during
element generation using {\tt ELEM}ent or {\tt BLOC}k commands.  If {\it id}
is less than or equal to zero it defaults to the value of the {\it ma}
parameter.  Material sets with the same {\it id} number are associated to each
element which designate this {\it id} number, thus, an element can be associated
with more than one material set.

The {\it idf} parameters are used to assign active degrees of
freedom.  Default: idf(i) = i, i=1,ndf.

The {\it MATE}rial command may also be used to provide a material
identification label for the {\sl FEAPpv} output file.
\pagebreak

\noindent{\bf{Example: MATErial}}

\begin{verbatim}
       MATE,1,Cam shaft material model: Aluminum mechanical
         SOLId,,1,1,2,3    ! properties for solid analysis
           ELAStic,,200.0d09,0.3
                     ! terminate set 1
       MATE,,2,Cam shaft material model: Aluminum thermal
         THERmal,,1,3,0,0  ! properties for thermal analysis
           FOURier,,50
                     ! terminate set 2}
\end{verbatim}

The {\it Cam shaft material model: Aluminum mechanical}
will appear in the output file before the first
material parameter values printed from the element routine.
Note, that two material sets have the same
material identifier, consequently the element connection
list belonging to this identifier will be processed
twice - once for the mechanical and once for the
thermal.  For the mechanical element the local dofs 1, 2,
and 3 will map to global dofs 1, 2, and 3; for the thermal
element local dof 1 will map to global dof 3.
The mechanical element will not form residual or tangent terms for the 3-dof;
however, it is used to extract the temperature used to calculate the thermal
strains.  This temperature degree of freedom must be designated for the
material set using a {\tt TEMP}erature command (or globally, using the
{\tt GLOB}al,{\tt TEMP}erature command).

The specific parameters to be input are described in the user manual
for the elements included with {\sl FEAPpv}.  For {\tt USER} elements the
data is set by the programmer of each module.
\vfil\eject
