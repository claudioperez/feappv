\heada{MESH}{PARAmeter}
\hspace{1.0cm}{{ para \hfill}} \\{\smallskip}
\hspace{1.4cm}{{ x = expression \hfill}}
\headb

The use of the {\tt PARA}meter command may be used to
assign values to letter parameters.  A letter parameter is
defined immediately following the {\tt PARA}meter command (several may
follow terminating with a blank record) according to the
following:

\begin{verbatim}
       x = expression
\end{verbatim}
where {\it x} may be any of the single letters ({\it a}-{\it z}), any group
of two letters ({\it aa}-{\it zz}), or any letter and a
numeral ({\it a0}-{\it z9})
followed by the equal sign.  The expression may be any set of
numbers (floating point numbers should contain an {\it E} or a {\it D}
exponent format so they will not be interpretted as integer
constants!) or one or two letter constants together with any of the
arithmetic operations +, -, *, /, or $\hat {~}$.
The expression is processed left to right and can contain one set of
parentheses to force groupings.  Examples are:

\begin{verbatim}
       a  = 3.
       bb = 14/3.45
       f  = a + 3.23/bb
       c  = f + 1.03e-04*a/bb
       d1 = (f + 1.03e-04)*a/bb
                      ! blank terminator
\end{verbatim}
In interactive mode of execution, the current set of paramater
values may be output by entering {\it list} while in
{\tt PARA}ameter input mode.  After listing, input of additional
parameters may be continued.  It is possible to use expressions
containing the parameters while in any input mode.

An input file may contain multiple {\tt PARA}meter commands.  The
values for parameters may be reset as needed.  If an
expression requires more than one set of parentheses a parameter
may be used to temporarily hold the value for one set of parentheses
and then reset.  For example,

\begin{verbatim}
       a = cos( (2*n-1)*p/l )
\end{verbatim}
is not legal because of the nested parenthese, but may be replaced by

\begin{verbatim}
       a = 2*n-1
       a = cos(a*p/l)
\end{verbatim}
which is legal.  Note the reuse and replacement of the {\it a} parameter.
The list of functions permitted in expressions is defined in the user
manual.
\vfil\eject
