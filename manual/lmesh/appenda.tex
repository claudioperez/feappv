%\documentclass[12pt]{article}

%\setlength{\textwidth}{15.5cm}
%\setlength{\oddsidemargin}{+0.4cm}
%\setlength{\evensidemargin}{+0.4cm}
%\setlength{\textheight}{21.7cm}
%\setlength{\parindent}{0pt}
%\setlength{\parskip}{3mm}
%\setlength{\parindent}{0pt}

%\pagenumbering{arabic}
%\setcounter{page}{1}

%\input emlines2.sty

%\usepackage{amsmath}
%\usepackage{latexsym}
%\usepackage{psfig}
%\usepackage{epsf}

%\newcommand{\Bf}[1]{\boldsymbol{#1}}
%\newcommand{\B}[1]{\mathbf{#1}}

\input ../lmesh/feap_in

%\begin{document}
\chapter[Mesh Manual]{Mesh Manual Pages}

{\sl FEAPpv} has several options which may be used to input data
to analyize a wide range of finite element problems in 1 to 3 dimensions.
The following
pages summarze the commands which are available to input specific parts
of the mesh data.  Provisions are also available for users to include their
own input routines through use of {\tt UMESHn} subprograms.  Methods
to write and interface user routines to the program are described in the
{\sl FEAP} Programmers Manual.
\vfill\eject

\input ../lmesh/alfeap.tex
\input ../lmesh/starele.tex
\input ../lmesh/starnod.tex
\input ../lmesh/angl.tex
\input ../lmesh/blen.tex
\input ../lmesh/bloc.tex
\input ../lmesh/boun.tex
\input ../lmesh/btem.tex
\input ../lmesh/cang.tex
\input ../lmesh/cbou.tex
\input ../lmesh/cdis.tex
\input ../lmesh/cfor.tex
\input ../lmesh/cpro.tex
\input ../lmesh/coor.tex
\input ../lmesh/csur.tex
\input ../lmesh/debu.tex
\input ../lmesh/disp.tex
\input ../lmesh/eang.tex
\input ../lmesh/ebou.tex
\input ../lmesh/edis.tex
\input ../lmesh/efor.tex
\input ../lmesh/elem.tex
\input ../lmesh/end.tex
\input ../lmesh/epro.tex
\input ../lmesh/ereg.tex
\input ../lmesh/forc.tex
\input ../lmesh/fpro.tex
\input ../lmesh/glob.tex
\input ../lmesh/incl.tex
\input ../lmesh/loop.tex
\input ../lmesh/manu.tex
\input ../lmesh/mate.tex
\input ../lmesh/next.tex
\input ../lmesh/nopa.tex
\input ../lmesh/nopr.tex
\input ../lmesh/para.tex
\input ../lmesh/pars.tex
\input ../lmesh/peri.tex
\input ../lmesh/pola.tex
\input ../lmesh/prin.tex
\input ../lmesh/regi.tex
\input ../lmesh/rese.tex
\input ../lmesh/shif.tex
\input ../lmesh/side.tex
\input ../lmesh/snod.tex
\input ../lmesh/temp.tex
\input ../lmesh/titl.tex
\input ../lmesh/tran.tex

%\end{document}
