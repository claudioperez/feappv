\heada{MESH}{*NODe}
\hspace{1.2cm}{{*nod = number \hfill}}
\headb

\index{Mesh command!*NODe}
The \texttt{*NODe} command is used to specify a base value for all
subsequently input nodal quantities.  The value may be reset as many
times as needed to define a complete mesh.  The default value is zero (0).
The command often will be used in conjunction with a \texttt{*ELEment}
command.

It is sometimes necessary to combine mesh data generated in two parts,
each of which may number nodes and elements starting with 1,2,3, etc.
In this case it will be necessary to use the \texttt{*NODe} command
to increment the values of node numbers on each record.

Consider two parts of a mesh which have been created with node
numbers \texttt{1} to \texttt{N1} for mesh 1 and 
numbers \texttt{1} to \texttt{N2} for mesh 2.  These are to be combined
to form a mesh containing \texttt{N1 + N2} nodes.  The structure for
the mesh input would be
\begin{verbatim}
       FEAP * * COMBINE
         ...
       INCLude MESH-1 (file with mesh data)

       *NODe    = N1 (value for max node    in MESH-1) 
       *ELEment = E1 (value for max element in MESH-1) 

       INCLude MESH-2 (file with mesh data)
         ...
\end{verbatim}

During the input of the second mesh \textsl{FEAP} will add the
value of \texttt{*NODe} to each value corresponding to a node number.
This could be nodal forces or the specification of node numbers on an
element record.  Note that the \texttt{*ELEment} command is necessary to
add offsets to element related numbers.
\vfill\eject
