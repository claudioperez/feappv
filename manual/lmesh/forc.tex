\heada{MESH}{FORCe}
\hspace{1.0cm}{{ forc                  \hfill}} \\{\smallskip}
\hspace{1.4cm}{{ node1,ngen1,(f(i,node1),i=1,ndf) \hfill}} \\{\smallskip}
\hspace{1.4cm}{{ node2,ngen2,(f(i,node2),i=1,ndf) \hfill}} \\{\smallskip}
\hspace{1.4cm}{{ <etc.,terminate with blank record> \hfill}}
\headb

The {\tt FORC}e command is used to specify the values for
nodal boundary forces.
For each node to be specified a record is entered with the
following information:

\begin{center}
\begin{tabular}{r l}
\it node      &-- Number of node to be specified \\
\it ngen      &-- Increment to next node, if generation \\
              &\quad is used, otherwise 0. \\
\it f(1,node) &-- Value of force for 1-dof \\
\it f(2,node) &-- Value of force for 2-dof \\
              &\quad etc., to {\it ndf} directions \\
\end{tabular}
\end{center}
When generation is performed, the node number  sequence
for {\it node1-node2} sequence shown at top will be:

\begin{center}
{\it node1, node1+ngen1, node1+2$\times$ngen1, .... , node2}
\end{center}

The values for each force will be a linear
interpolation between the {\it node1} and {\it node2} values for
each degree-of-freedom.

While it is possible to specify both the force and the displacement applied
to a node, only one can be active during a solution step.  The determination
of the active value is determined from the boundary
restraint condition value.  If the boundary restraint value is zero
and you use one of the force-commands a force
value is imposed, whereas, if the boundary restraint value is non-zero
and you use one of the displacement-commands a
displacement value is imposed. (See {\tt BOUN}dary, {\tt CBOU}ndary,
or {\tt EBOU}ndary pages for setting boundary
conditions.).  It is possible to change the type of boundary restraint
during execution by resetting the boundary restraint value.

Force conditions may also be specified using the {\tt EFOR}ce and {\tt CFOR}ce
commands.  The data is order dependent with data
defined by {\tt FORC}e processed first, {\tt EFOR}ce processed second and
the {\tt CFOR}ce data processed last.  The value defined last is used for
any analysis.

\noindent{\bf{Example: FORCe}}

A concentrated force is to be applied to nodes 10 and 15.  The force at node
10 has values of 100.0 in the horizontal direction and 0 in the vertical
direction; whereas the force at node 15 has a magnitude of 200 and makes an
angle of $60^o$ with the horizonatal axis.  These two forces may be specified
using the command set:
\begin{verbatim}
       FORCe
         10  0  100.0          0.0
         15  0  200*cosd(60) 200*sind(60)
\end{verbatim}
Note the use of the built-in functions available in \textsl{FEAPpv} to compute
the horizontal and vertical components.

\vfil\eject
