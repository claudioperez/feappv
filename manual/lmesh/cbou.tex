\heada{MESH}{CBOUndary}
\hspace{1.0cm}{{ cbou,[set,add]          \hfill}} \\{\smallskip}
\hspace{1.4cm}{{ node,(x(i),i=1,ndm),(ibc(j),j=1,ndf) \hfill}} \\{\smallskip}
\hspace{1.4cm}{{ linear,(ibc(j),j=1,ndf) \hfill}} \\{\smallskip}
\hspace{1.8cm}{{ 1,x1,y1 \hfill}} \\{\smallskip}
\hspace{1.8cm}{{ 2,x2,y2 \hfill}} \\{\smallskip}
\hspace{1.4cm}{{ quadratic,(ibc(j),j=1,ndf) \hfill}} \\{\smallskip}
\hspace{1.8cm}{{ 1,x1,y1 \hfill}} \\{\smallskip}
\hspace{1.8cm}{{ 2,x2,y2 \hfill}} \\{\smallskip}
\hspace{1.8cm}{{ 3,x3,y3 \hfill}} \\{\smallskip}
\hspace{1.4cm}{{ surface,(ibc(j),j=1,ndf) \hfill}} \\{\smallskip}
\hspace{1.8cm}{{ 1,x1,y1,z1 \hfill}} \\{\smallskip}
\hspace{1.8cm}{{ 2,x2,y2,z2 \hfill}} \\{\smallskip}
\hspace{1.8cm}{{ 3,x3,y3,z3 \hfill}} \\{\smallskip}
\hspace{1.8cm}{{ 4,x4,y4,z4 \hfill}} \\{\smallskip}
\hspace{1.4cm}{{ cartesian \hfill}} \\{\smallskip}
\hspace{1.4cm}{{ pola,x0,y0 \hfill}} \\{\smallskip}
\hspace{1.4cm}{{ gap,value \hfill}} \\{\smallskip}
\hspace{1.4cm}{{ <etc.,terminate with a blank record> \hfill}}
\headb

The boundary restraint conditions may be set using
the reference coordinates for a single {\it node}, a {\it linear}
line or a {\it quadratic} line.  The input values are saved
in files and searched after the entire mesh is specified.
The data is order dependent with data
defined by {\tt BOUN}e processed first, {\tt EBOU}le processed second and
the {\tt CBOU}le data processed last.  The value defined last is used for
any analysis.
After use files are deleted automatically.

The {\tt CBOU} command may be used with two options.  Using the {\tt CBOU,SET}
option replaces all previously defined conditions at any node by the
pattern specified. This is the default mode.  Using the {\tt CBOU,ADD} option
accumulates the specified boundary conditions with previously defined
restraints.

For a single {\it node}, the data to be supplied during
the definition of the mesh consists of:

\begin{center}
\begin{tabular}{r l}
\it node   &-- Defines inputs to be for a {\it node} \\
\it x(1)   &-- Value of coordinates to be used during search \\
\it ...    &\quad (a tolerance of about 1/1000 of mesh size is \\
\it x(ndm) &\quad used during search, coordinate with smallest \\
           &\quad distance within tolerance is assumed to have \\
           &\quad specified value). \\
\it ibc(1) &-- Restraint conditions for all nodes with value \\
\it ibc(2) &\quad of search. (0 = active dof, $>$0 or $<$0 denotes \\
\it ...    &\quad a fixed dof \\
\it ibc(ndf) &
\end{tabular}
\end{center}
For two dimensional problems it is possible to
specify the segment to which the boundary conditions
are applied.  The segment may be specified as a {\it linear}
or a {\it quadratic} line.  For the {\it linear} segment the boundary
condition pattern are given together with the
coordinates of the ends.  These are specified as:

\begin{verbatim}
       LINEar,(ibc(i),i=1,ndf)
         1,x1,y1
         2,x2,y2
\end{verbatim}

For {\it quadratic} segments the ends {\it (x1,y1)} and {\it (x2,y2)}
together with an intermediate point {\it (x3,y3)} are used.
The {\it quadratic} segment is given as:

\begin{verbatim}
       QUADratic,(ibc(i),i=1,ndf)
         1,x1,y1
         2,x2,y2
         3,x3,y3
\end{verbatim}

For three dimensional problems it is possible to
specify the segment to which the boundary conditions
are applied.  The segment is specified as a {\it surface}.
The data is specified as:

\begin{verbatim}
       SURFace,(ibc(i),i=1,ndf)
         1,x1,y1,z1
         2,x2,y2,z2
         3,x3,y3,z3
         4,x4,y4,z4
\end{verbatim}

The program assigns a search region and attempts to
find the elements and the nodes to which the specified
segments are associated.  It is possible that no segment
is located (an error message will appear in the output file).  To
expand the search region a {\it gap} can be specified as:

\begin{verbatim}
       GAP,value
\end{verbatim}

\vspace{3mm}

\noindent
The {\it gap-value} is a coordinate distance within which
nodes are assumed to lie on the specified segment.  The
value should be less than dimensions of typical elements
or erroneous nodes will be found by the search.
It is suggested that the computed boundary conditions
be checked graphically to ensure that they are
correctly identified (e.g., use {\tt PLOT,MESH} and {\tt PLOT,BOUN}
to show the locations of conditions).

The {\it polar} option may be used to set the origin of a
polar coordinate system.  Coordinates entered after
{\it polar} will be assumed to be radius and angle. The angles must
be input in degrees. The
{\it cartesian} option resets the coordinate system to a
cartesian frame.
\vfil\eject
