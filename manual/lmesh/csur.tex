\heada{MESH}{CSURface}
\hspace{1.0cm}{{ csur       \hfill}} \\{\smallskip}
\hspace{1.4cm}{{ linear     \hfill}} \\{\smallskip}
\hspace{1.8cm}{{ 1,x1,y1,p1 \hfill}} \\{\smallskip}
\hspace{1.8cm}{{ 2,x2,y2,p2 \hfill}} \\{\smallskip}
\hspace{1.4cm}{{ quadratic  \hfill}} \\{\smallskip}
\hspace{1.8cm}{{ 1,x1,y1,p1 \hfill}} \\{\smallskip}
\hspace{1.8cm}{{ 2,x2,y2,p2 \hfill}} \\{\smallskip}
\hspace{1.8cm}{{ 3,x3,y3,p3 \hfill}} \\{\smallskip}
\hspace{1.4cm}{{ surface    \hfill}}     \\{\smallskip}
\hspace{1.8cm}{{ 1,x1,y1,z1,p1  \hfill}} \\{\smallskip}
\hspace{1.8cm}{{ 2,x2,y2,z2,p2  \hfill}} \\{\smallskip}
\hspace{1.8cm}{{ 3,x3,y3,z3,p3  \hfill}} \\{\smallskip}
\hspace{1.8cm}{{ 4,x4,y4,z4,p4  \hfill}} \\{\smallskip}
\hspace{1.4cm}{{ disp,component \hfill}} \\{\smallskip}
\hspace{1.4cm}{{ normal      \hfill}} \\{\smallskip}
\hspace{1.4cm}{{ tangential  \hfill}} \\{\smallskip}
\hspace{1.4cm}{{ flux        \hfill}} \\{\smallskip}
\hspace{1.4cm}{{ polar,x0,y0 \hfill}} \\{\smallskip}
\hspace{1.4cm}{{ cartesian   \hfill}} \\{\smallskip}
\hspace{1.4cm}{{ gap,value   \hfill}} \\{\smallskip}
\hspace{1.4cm}{{ axisymetric \hfill}} \\{\smallskip}
\hspace{1.4cm}{{ plane       \hfill}} \\{\smallskip}
\hspace{1.4cm}{{ <terminate with a blank record> \hfill}}
\headb

A mesh may be generated in {\sl FEAPpv} in which it is
desired to specify distributed loading or displacements on parts of the
body. Distributed loading may be  \textit{traction} for solid mechanics
problems or \textit{flux} for thermal problema.

For two dimensional problems it is possible to
specify the surface to which the boundary condition is applied
using the {\tt CSUR}face command (The command is for Coordinate
specified SURfaces.).
The input values are saved in files and searched
after the entire mesh is input (i.e., after the {\tt END} mesh command.
After use the temporary files are deleted.
The data is order dependent with data
defined by other input options.  Surface data is always generated last.

The type of input to be generated is set using the \textit{displacement},
\textit{normal}, \textit{tangential}, or \textit{flux}  options.  These specify that inputs
will be a specific
displacement component, normal tractions (pressures), tangential
tractions (shears), or thermal fluxes, respectively.  The default is \textit{normal} loadings.
For displacement inputs the component to be generated is specified
immediately after the {\it displacement} command.

A two-dimensional surface may be specified as a
{\it linear} or a {\it quadratic} line.  For the linear surface the
values at the ends {\it p1}, {\it p2} are given together with 
the end coordinates {\it (x1,y1)} and {\it (x2,y2)}.
These are specified as:

\begin{verbatim}
       type data
       LINEar
         1,x1,y1,p1
         2,x2,y2,p2}
\end{verbatim}
For traction data of 2-dimensional axisymmetric problems the \texttt{type data} should use
the \texttt{AXiSYmmetric} option to ensure the specified traction 
is multiplied by the radius.
`

For quadratic line surfaces the ends (nodes 1 and 2)
together with an intermediate point are used.  Thus it
is possible to have quadratic variation of the values.
The quadratic surface is given as:

\begin{verbatim}
       type data
       QUADratic
         1,x1,y1,p1
         2,x2,y2,p2
         3,x3,y3,p3
\end{verbatim}

For three dimensional problems it is possible to
specify the segment to which the quantities
are applied.  The segment is specified as a {\it surface}.
The data is specified as:

\begin{verbatim}
       type data
       SURFace
         1,x1,y1,z1,p1
         2,x2,y2,z2,p2
         3,x3,y3,z3,p3
         4,x4,y4,z4,p4
\end{verbatim}

The program assigns a search region and attempts to
find the elements and the nodes to which the specified
surfaces are associated.  It is possible that no surface
is located (an error message will appear in the output file).  To
expand the search region a {\it gap} can be specified as:

\begin{verbatim}
       GAP,value
\end{verbatim}

The {\it gap-value} is a coordinate distance within which
nodes are assumed to lie on the specified surface.  The
value should be less than dimensions of typical element
or erroneous surfaces will be found by the
search.  It is suggested that the computed loads be
checked graphically to ensure that they are correctly
identified (e.g., use {\tt PLOT,MESH} and {\tt PLOT,LOAD} to show
the locations of computed loads).

The {\it polar} option may be used to set the origin of a
polar coordinate system.  Coordinates entered after
{\it polar} will be assumed to be radius and angle.  The
{\it cartesian} option resets the coordinate system to a
cartesian frame.  The default mode is {\it cartesian}.

The nodes 1, 2 (and 3 and 4 if required) must be input in the right order. The
normal vector of the surface has to point outward from the surface as defined
by a right-hand rule.
\vfil\eject
