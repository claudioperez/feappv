\heada{MESH}{SIDE}
\hspace{1.0cm}{{ side                  \hfill}} \\{\smallskip}
\hspace{1.4cm}{{ type1,(is(i,side1),i=1,nn) \hfill}} \\{\smallskip}
\hspace{1.4cm}{{ type2,(is(i,side2),i=1,nn) \hfill}} \\{\smallskip}
\hspace{1.4cm}{{ <etc.,terminate with blank record> \hfill}}
\headb

Currently, {\sl FEAPpv} uses the
{\tt SIDE} command to generate patches of a mesh using the {\it blending
function} option and to determine contact surfaces.
Blending functions are briefly discussed in the
Zienkiewicz \& Taylor finite element book, volume 1 pp 181 ff.  Each
super node is defined by an input of the following information:

It is necessary to define only those edges which are not straight or which
have interpolations which generate non-equal spacing on a straight edge.
There are four options for generating the side description as indicated
in the following table:

\begin{center}
\begin{tabular}{l l}
\it type     & \qquad Type of interpolation \\ \hline
    cart     &  - Lagrange interpolation in cartesian coordinates \\
    pola     &  - Lagrange interpolation in polar coordinates \\
    segm     &  - Straight multi-segment interpolation \\
    elli     &  - Lagrange interpolation in elliptical coordinates
\end{tabular}
\end{center}
For Lagrange interpolation in cartesian coordinates the list of values
defining the connected super nodes are given according to the following:

\begin{center}
\begin{tabular}{r l}
\it  is     & \qquad Type of interpolation \\ \hline
      1     &  - End 1 super-node number \\
      2     &  - End 2 super-node number \\
      3     &  - Intermediate node nearest End 1 \\
  $\cdots$  &  - etc. for remaining internal nodes
\end{tabular}
\end{center}
For Lagrange interpolation in polar or elliptical coordinates the list of values
is input as above, followed by the super-node number defining the
location of the origin for the polar radius.

For straight multi-segment interpolations the inputs are given as:

\begin{center}
\begin{tabular}{c l}
\it  is     & \qquad Type of interpolation \\
      1     &  - End 1 super-node number \\
      2     &  - Number of equal increments to next node \\
      3     &  - Intermediate node nearest End 1 \\
      4     &  - Number of equal increments to next node \\
      5     &  - Next intermediate node \\
  $\cdots$  &  - etc. for remaining internal nodes \\
     nn     &  - End 2 super-node number
\end{tabular}
\end{center}
In addition to the side definitions it is necessay to define the super-node
locations using the mesh command {\tt SNOD}e.  Finally, the mesh command
{\tt BLEN}d must be specified for each mesh patch to be created.
\vfil\eject
