\heada{MESH}{EDISplacement}
\hspace{1.0cm}{{ edis                            \hfill}} \\{\smallskip}
\hspace{1.4cm}{{ i-coor,xi-value,(d(j),j=1,ndf) \hfill}} \\{\smallskip}
\hspace{1.4cm}{{ <etc.,terminate with a blank record> \hfill}}
\headb

The values of boundary displacment conditions may be set
along any set of nodes which has a constant value of the
{\it i-coordinate direction} (e.g., 1-direction (or x), 2-direction
(or y), etc.).  The data to be supplied during the
definition of the mesh consists of:

\begin{center}
\begin{tabular}{r l}
\it i-coor   &-- Direction of coordinate (i.e., 1 = x, 2 = y, etc.) \\
\it xi-value &-- Value of i-direction coordinate to be used during  \\
             &\quad search (a tolerance of 1/1000 of mesh size is used  \\
             &\quad during search, any coordinate within the gap is  \\
             &\quad assumed to have the specified value).  \\
\it d(1)     &-- Value of displacement for dof's  \\
\it d(2)     &   \\
\it ...      &   \\
\it d(ndf)   &   \\
\end{tabular}
\end{center}
While it is possible to specify both the force and the displacement applied
to a node, only one can be active during a solution step.  The determination
of the active value is determined from the boundary
restraint condition value.  If the boundary restraint value is zero
and you use one of the force-commands a force
value is imposed, whereas, if the boundary restraint value is non-zero
and you use one of the displacement-commands a
displacement value is imposed. (See {\tt BOUN}dary, {\tt CBOU}ndary,
or {\tt EBOU}ndary pages for setting boundary
conditions.).  It is possible to change the type of boundary restraint
during execution by resetting the boundary restraint value. It is not
possible, to specify a displacement by using the combination of a 
force-command with a non-zero boundary restraint value, as it was
in the last releases of {\sl FEAPpv}. For further information see the
{\tt CDIS}placement page.

Displacement conditions may also be specified using the {\tt DISP}
and {\tt CDIS} commands.
The data is order dependent with data
defined by {\tt DISP} processed first, {\tt EDIS} processed second and
the {\tt CDIS} data processed last.  The value defined last is used for
any analysis.
\pagebreak

\noindent{\bf{Example: EDISplacement}}

All the nodes located on the $x_3 = z = 0$ plane are to have a vertical
displacement ($2^{nd}$ dof) of -0.25 units.  This may be set using the commands
\begin{verbatim}
       EDISplacement
         3 0.0  0.0 -0.25
\end{verbatim}
In addition it is necessary to specify boundary restraint codes for the nodes
to which the condition is to be applied.  A simple way to do this is to use
the command set:
\begin{verbatim}
       EBOUndary
         3 0.0  0 1
\end{verbatim}
Of course the horizontal ($1^{st}$) dof could be restrained for any
of the nodes also.
\vfil\eject
