\headm{MESH}{TIE}
\hspace{1.0cm}{{ tie            \hfill}} \\{\smallskip}
\hspace{0.8cm}{{ tie,line,n1    \hfill}} \\{\smallskip}
\hspace{0.8cm}{{ tie,node,n1,n2 \hfill}} \\{\smallskip}
\hspace{0.8cm}{{ tie,regi,n1,n2 \hfill}} \\{\smallskip}
\hspace{0.8cm}{{ tie,mate,n1,n2 \hfill}} \\{\smallskip}
\hspace{0.8cm}{{ tie,,dir,x-dir \hfill}} \\{\smallskip}
\headb

A mesh may be generated by {\sl FEAPpv} in which there is more
than one node with the same coordinates.  The {\tt TIE} command
may be used after the mesh {\tt END} command
to {\it merge} these nodes so that the same values
of the solution will be produced at specified nodes which have the same
initial coordinates. Current options include:

\begin{center}
\begin{tabular}{r l}
\it line &-- [Currently not documented] \\
\it node &-- Search node list between nodes {\it n1} and {\it n2} \\
\it regi &-- Search regions {\it n1} and {\it n2} ({\it n1}
can equal {\it n2}) \\
\it mate &-- Search material identifiers {\it n1} and {\it n2} ({\it n1}
can equal {\it n2}) \\
\end{tabular}
\end{center}
\noindent
To use the {\tt TIE} option the complete mesh must first be
defined.  After the {\tt END} command for the mesh definition
and before the {\tt BATC}h or {\tt INTE}ractive command
for defining a solution algorithm, use of a {\tt TIE} statement
will cause the program to search for all coordinates
that are to be connected together.  Use of the {\tt TIE} command
without additional parameters will search all nodes and join
those which have coordinates with the same values (to within a
small tolerance).  Use of {\tt TIE},,{\it i,value} (with i = 1,..,ndm)
will tie nodes with common coordinates which are on the
plane defined with an $x_i$ coordinate equal to {\it value}.
Similarly, the use of the {\it region} or {\it material} parameters
will result in searches based on these identifiers.

When nodes are connected any specified, restrained
boundary condition will be assigned to all interconnected
nodes.  Thus, it is only necessary to specify restrained
boundary conditions and loadings for one of the nodes.
\vfil\eject
