\section{User modules}

The capability to perform isogeometric analysis has largely been incorporated
into \textsl{FEAP} through user modules.  These modules permit the refinement
of original model descriptions into analysis suitable form.  This is accomplished
by degree elevation of the NURBS and by knot insertion to create additional
subdivisions that are used to describe \textit{elements}.  The various
options that have been developed to date are described in Table \ref{tab1iga}.
In addition to the solution commands a number of user elements have been
added as described in Table \ref{tab2iga}.

\input{iga_table.tex}

\section{Elevation of order and adding knot values}

Generally, the initial input data defining the geometry of the problem is not
sufficient for an accurate analysis.  It may be necessary to \textit{elevate}
the order of the NURBS approximation for the patches or to \textit{insert}
additional knot values in the knot vectors.  To perform these steps by hand
is a tedious and error prone process.  The current implementation in
\textsl{FEAP} permits both steps to be performed using solution `Command
Language' statements.

\subsection{Degree ELEVation}

To raise the polynomial order of one B-spline defining a NURBS the
solution command set
\begin{verbatim}
       ELEVate BLOCk blk dir incr
\end{verbatim}
may be used.  The parameters are: \texttt{blk} is the block number;
\texttt{dir} is the direction in the block to elevate; and \texttt{incr} is
the order increment to increase.  Execution of these commands creates a
file named: \texttt{NURB\_mesh} that contains the new set of control points,
side lists, knot vectors and nurb blocks.  At this stage only one direction
in one block has been elevated and it is necessary to repeat the process for
other blocks and directions.  The process of repeating the process can be
performed using the \textsl{FEAP} input \texttt{LOOP-NEXT} commands.

\subsubsection{Multiple elevations}

The process of performing several elevations can be accommodated easily by
preparing a mesh input file with the structure
\begin{verbatim}
       FEAP <optional title information>
         0 0 0 ndm ndf nel

       MATE <all material properties included in original mesh

       INCLude NURB_mesh

       END
       STOP
\end{verbatim}
Then prepare a second input file that has the form:
\begin{verbatim}
       INCLude I<original mesh>
       BATCH
         ELEVate BLOCk blk1 dir1 incr1
       END

       INCLude I<NURB_mesh>
       BATCh
         ELEVate BLOCk blk2 dir2 incr2
       END

       etc.
\end{verbatim}
where \texttt{I<NURB\_mesh>} is the name of the input file containing the
\texttt{INCLude NURB\_mesh} statement.
After the processing of the original mesh the subsequent processes may all
use the file containing the \texttt{NURB\_mesh} include.  Using parameters
and looping structures such as
\begin{verbatim}
       PARAmeter
         d = 1

       LOOP,2
         PARAmeter
           d = d + 1

         INCLude I<NURB_mesh>
         BATCh
           ELEVate BLOCk 1 d 2
         END
       NEXT
\end{verbatim}
would elevate the second and third directions of block 1 by two orders.

\subsection{KNOT insertion}

To insert knots the command set
\begin{verbatim}
       INSErt BLOCk blk dir value rr
\end{verbatim}
may be use.  The parameters are: \texttt{blk} is the block number;
\texttt{dir} the knot direction in the block; \texttt{value} the location
in the knot vector to perform an insertion; and \texttt{rr} the number of
times to repeat the insertion.
Alternatively, knot insertions may be specified by the knot number using the
command set
\begin{verbatim}
       INSErt KNOT knum value rr
\end{verbatim}
where \texttt{knum} is the knot number and \texttt{value} and \texttt{rr} are
as described above.

Each use of an \texttt{INSErt} set again
results in a new \texttt{NURB\_mesh}.  Multiple insertions can again be 
performed using the above loop structure.  For meshes that perform several
knot insertions it may take several seconds before the final \texttt{NURB\_mesh}
file is created.

After several elevations and insertions, the final \texttt{NURB\_mesh} file
contains the isogeometric problem description suitable for analysis.  The
analysis can be performed using an input file containing the
\texttt{INCLude NURB\_mesh} along with boundary conditions and loading
specification.  This is now a standard \textsl{FEAP} solution process and
any of the solution options described in the User Manual may be used.
Recall that the activation of a NURBS analysis is controlled by a statement
in the \texttt{MATErial} set data.
