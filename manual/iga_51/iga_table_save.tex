
\begin{table}[h]
\begin{center}
\begin{tabular}{| c | r | p{11cm}|} \hline
\texttt{UMACR} & uct & Description \\ \hline
 2 & \texttt{nzzp} & ZZ projection of stress for 2-d T-splines \\
 3 & \texttt{bdis} & Calls  \texttt{pdblock(3)} \\
 4 & \texttt{xloc} & Locate knot coordinate to insert boundary coordinate \\
& & \texttt{BDIS,, x, y, z, blk\_no} \\
 5 & \texttt{bezi} & Bezier extraction from NURBS curve to generate power matrix form\\
& & \texttt{BEZIer,, x, y, z} \\
 6 & \texttt{extr} & Compute extraction operator for 1-d forms. \\
& & \texttt{EXTRact BEZIer n\_side} \\
 7 & \texttt{curv} & Elevation and knot insertion of curve \\
& & \texttt{CURVe,, no, deg, ord }\\
 8 & \texttt{elev} & Elevate individual directions of NURBS block. \\
& & \texttt{ELEVate block blk\_num dir inc\_order} \\
 9 & \texttt{inse} & Insert knot in individual directs of NURBS block. \\
& & \texttt{INSERT block u\_knot times} \\
 0 & \texttt{eiga} & Determine elements on edges of NURBS blocks or T-splines.\\
& & \texttt{EIGA,,dir xi} \\ \hline
\end{tabular}
\caption{Commands for NURBS/T-spline solutions. \label{tab1iga} }
\end{center}
\end{table}

\begin{table}[t]
\begin{center}
\begin{tabular}{| r | p{11cm}|} \hline
\texttt{ELMT} & Description \\ \hline
 1~ & NURBS Euler-Bernoulli beam \\
 2~ & NURBS 1-d rod. \\
 3~ & NURBS 1-d displacement boundary condition \\ 
 4~ & NURBS traction loading for tension strip \\
 5~ & NURBS \& T-spline thin $C^1$ plate \\
 6~ & NURBS \& T-spline thin membrane \\
 7~ & NURBS \& T-spline derivative boundary condition. \\
 8~ & Bending patch for Euler-Bernoulli beam ties. \\
18~ & Global least squares boundary fit for NURBS quadrilateral region \\
19~ & Local least squares boundary fit for NURBS quadrilateral region \\
20~ & Follower couple to load thin shell element \\
21~ & Thick shell ??? \\
22~ & Reissner-Mindlin thick  isogeometric shell (Benson et al.) \\
23~ & Thin isogeometric shell (Benson et al.) \\
24~ & Thin isogeometric non-linear shell (Kiendl et al.) \\
25~ & Thin isogeometric linear shell (Kiendl et al.) \\ \hline
\end{tabular}
\caption{User elements for NURBS/T-spline solutions. \label{tab2iga} }
\end{center}
\end{table}
