\section{Isogeometric elements}

The selection of an element formulation based on NURBS or T-spline interpolation is controlled in the \texttt{MATErial} data input by the command
\begin{verbatim}
       NURBs,,q1 q2 q3
\end{verbatim}
for block NURBS forms or
\begin{verbatim}
       TSPLine,,q1 q2 q3
\end{verbatim}
for T-spline forms using the mesh output from the refinement program of Mike
Scott.  In the above $qi$ is the quadrature order for each knot direction in an
element.

\subsection{Solid elements}

An isogeometric analysis may be performed using the small and finite deformation
solid elements formulated in terms of displacements only.  The basic
material input for an analysis is given by the usual material set commands
described in the \textsl{FEAP} User Manual\scite{feapu}.  For example
an analysis using small displacement model may be given as  
\begin{verbatim}
       MATErial ma
         SOLID
           ELAStic ISOTropic 200e9 0.3
           DENSity mass      8.9
           NURBS,,q1 q2 q3  ! or TSPLine,, q1 q2 q3
\end{verbatim}
A finite deformation model is used by adding the statement \texttt{FINIte}
to the above or by specifying a specific finite deformation material model
(e.g., \texttt{ELASTic NEOHook}).

The standard implementations for mixed and enhanced forms does not work
for an isogeometric analysis.  For meshes described by NURBS blocks a
special form of the mixed formulation is available.  The formulation has
nodal values for the displacements $\B{u}$ and the mean stress pressure $p$
and a discontinuous approximation for the volume behavior in each element.
For this formulation it is necessary to explicitly set the maximum number of
nodal parameters on the control record.  Thus for a 2-d problem one starts
an analysis by the statements
\begin{verbatim}
       FEAP * * <title information for mixed analysis>
         0 0 0 2 3 0
\end{verbatim}
and for 3-d the statements
\begin{verbatim}
       FEAP * * <title information for mixed analysis>
         0 0 0 3 4 0
\end{verbatim}

The material records are then given by
\begin{verbatim}
       MATErial ma
         SOLID
           ELAStic NEOHook  200e9 0.3
           DENSity mass      8.9
           NURBS,,q1 q2 q3  ! or TSPLine,, q1 q2 q3
           MIXEd
\end{verbatim}

\subsection{Thin shell elements}

The thin shell Kirchhoff-Love formulation by
Kiendl \textit{et al.}\scite{kiendl09a} may be specified using the
material commands
\begin{verbatim}
       MATErial ma
         USER <24,25> 
           ELAStic ISOTropic E nu
           DENSity mass      ro
           THICkness shell   h
           <NURBs,TSPLine>,, q1 q2
\end{verbatim}
Element type 24 uses large displacement kinematics and
element type 25 uses linear kinematics.
The element has no rotation parameters and thus the control data may be set
as
\begin{verbatim}
       FEAP * * <title information for shell analysis>
         ndm = 3
         ndf = 3
\end{verbatim}

