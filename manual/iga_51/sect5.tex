\section{Solutions using T-splines}

\textsl{FEAP} permits the calculation of isogeometric objects represented by
\textit{T-splines}.  The solution is obtained using an \textit{extraction
operator} form in which the element shape functions are expressed in terms
of shape functions given as
\begin{displaymath}
\B{N}^e = \B{C}^e \, \B{R}^e
\end{displaymath}
where $\B{R}^e$ are a Bezier representation of NURBS, $\B{C}^e$ is the element
extraction operator and $\B{N}^e$ are the T-spline shape functions.

The data input is provided by an output from the refinement program
developed at The University of Texas by Mike Scott\scite{scott10a}
and included as an extension
of the T-Splines\scite{tsplines} plug-in to Rhino\scite{rhino}.  

Only surface data is provided and thus analyses are restricted to bodies that
are represented by surfaces (e.g., 2-d solid bodies, membranes and shells).  A
typical input file is given as:
\begin{verbatim}
       FEAP * * Title information
         ndm = <2,3>     ! 2-d solids or 3-d surfaces, respectively
         ndf = <1,2,...> ! Describes number of dof at a control point

       MATErial
         <SOLId, MEMBrane, SHELl>
         elastic isotropic E nu
         NURBs,TSPLine,q1,q2,q3

       T-SPline
         PLOT INTErval <1 to 7>    ! Number of subdivision of surface
         FILE = "filename of data"

       .... ! Loads, B.C., etc.
       END              ! End of data inputs
       INTEractive      ! Interactive solution commands
       STOP             ! End of data file
\end{verbatim}

Standard solution commands may be used.  Graphics is available in a manner
similar to that for NURBS problems.
