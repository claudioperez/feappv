\section{NURBS mesh refinement}

Generally, the initial input data defining the geometry of the problem is not
sufficient for an accurate analysis.  It may be necessary to \textit{elevate}
the order of the NURBS approximation for the patches or to \textit{insert}
additional knot values in the knot vectors.  To perform these steps by hand
is a tedious and error prone process.  The current implementation in
\textsl{FEAP} permits both steps to be performed using solution `Command
Language' statements.
These are generally given in a batch solution.

\subsection{Degree ELEVation}

To raise the polynomial order of one B-spline defining a NURBS the
solution command set
\begin{verbatim}
       BATCh
         ELEVate PATCh blk dir incr
       END
\end{verbatim}
may be used.  The parameters are: \texttt{blk} is the patch number;
\texttt{dir} is the direction in the patch to elevate; and \texttt{incr} is
the order increment to increase.  Execution of these commands creates a
file named: \texttt{NURBS\_mesh} that contains the new set of control points,
side lists, knot vectors and patches.  At this stage only one direction
in one patch has been elevated and it is necessary to repeat the process for
other patches and directions.  The process of repeating the process can be
performed using the \textsl{FEAP} input \texttt{LOOP-NEXT} commands.

\subsubsection{\ul{Multiple elevations}}

In addition to preparing the input file for the original problem
description,
the process of performing several elevations can be accommodated easily by
preparing an additional mesh input file with the structure
\begin{verbatim}
       FEAP <optional title information>
         0 0 0 ndm ndf nel

       MATE <all material properties included in original mesh

       INCLude NURBS_mesh

       END
       STOP
\end{verbatim}
Then prepare a third input file that has the form:
\begin{verbatim}
       INCLude I<original mesh>
       BATCH
         ELEVate PATCh pat1 dir1 incr1
       END

       INCLude I<NURBS_mesh>
       BATCh
         ELEVate PATCh pat2 dir2 incr2
       END

       etc. for other patches/directions
\end{verbatim}
where \texttt{I<NURBS\_mesh>} is the name of the second input file containing the
\texttt{INCLude NURBS\_mesh} statement.
The analysis is initiated by specifying the third file as the solution
input file.
After the processing of the original mesh the subsequent processes may all
use the file containing the \texttt{NURBS\_mesh} include.  Using parameters
and looping structures such as
\begin{verbatim}
       PARAmeter
         d = 1

       LOOP,2
         PARAmeter
           d = d + 1

         INCLude I<NURBS_mesh>  ! File with INClude NURBS_mesh
         BATCh
           ELEVate PATCh 1 d 2
         END
       NEXT
\end{verbatim}
would elevate the second and third directions of three-dimensional
patch 1 by two orders.

\subsection{KNOT insertion}

To insert knots the command set
\begin{verbatim}
       BATCh
         INSErt PATCh pat dir value rr
       END
\end{verbatim}
may be use.  The parameters are: \texttt{pat} is the patch number;
\texttt{dir} the knot direction in the patch; \texttt{value} the location
in the knot vector to perform an insertion; and \texttt{rr} the number of
times to repeat the insertion.

Each use of an \texttt{INSErt} set again
results in a new \texttt{NURBS\_mesh}.  Multiple insertions can again be 
performed using the above loop structure.  For meshes that perform several
knot insertions some time may elapse before the final \texttt{NURBS\_mesh}
file is created.

After several elevations and insertions, the final \texttt{NURBS\_mesh} file
contains an isogeometric problem description suitable for analysis.  The
analysis can be performed using an input file containing the
\texttt{INCLude NURBS\_mesh} along with boundary conditions and loading
specification.  This is now a standard \textsl{FEAP} solution process and
any of the solution options described in the User Manual may be used.
Recall that the activation of a NURBS analysis in an element is controlled by a
statement in the \texttt{MATErial} set data:
\begin{verbatim}
       MATE ma
          ....
         NURBs,,q1,q2,q3
          ....
\end{verbatim}
